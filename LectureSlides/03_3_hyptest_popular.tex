\documentclass[10pt,t,english]{beamer}
\usepackage{fontawesome}
\usepackage{graphicx}
\usepackage{array}
\usepackage[normalem]{ulem}
\usepackage{amsfonts,amsmath,amssymb,bm,bbm}
\usepackage{mathrsfs}
\usepackage{sgame}
\usepackage{graphicx,pstricks}
\usepackage{xcolor}
\usepackage{colortbl}
\usepackage{makecell}
\usepackage{tikz,tikzsymbols,gnuplottex}
\usetikzlibrary{decorations.pathreplacing,shapes}
\usepackage[english]{babel}
\usepackage[utf8]{inputenc}
\usepackage{appendixnumberbeamer}
\usepackage{datetime2}
\usepackage{booktabs}
\usepackage{setspace}
\usepackage{rotating}
\usepackage{natbib}
\usepackage{listings}

%\usepackage{matlab-prettifier} % For enhanced MATLAB highlighting
\lstset{
    language=matlab,
    % Other options for styling (optional)
    basicstyle=\ttfamily\footnotesize, % Font style
    keywordstyle=\color{blue}, % Keyword color
    commentstyle=\color{green!50!black}, % Comment color
    stringstyle=\color{red!70!black}, % String color
    numbers=left, % Line numbers on the left
    numberstyle=\tiny\color{gray}, % Style for line numbers
    frame=single, % Frame around the listing
    breaklines=true, % Allow line breaking
    captionpos=b, % Caption at the bottom
    tabsize=4 % Tab size
}

% ------------------------------------------------------------------------------
% Use the beautiful metropolis beamer template
% ------------------------------------------------------------------------------
\usepackage[T1]{fontenc}
\usepackage[utf8]{inputenc}
\usepackage{fontawesome}
\usepackage{FiraSans} 
\mode<presentation>
{
  \usetheme[progressbar=foot,background=light]{metropolis} 
  \usecolortheme{default} % or try albatross, beaver, crane, ...
  \usefonttheme{default}  % or try default, serif, structurebold, ...
  \setbeamertemplate{navigation symbols}{}
  \setbeamertemplate{caption}[numbered]
  %\setbeamertemplate{frame footer}{My custom footer}
} 

\newenvironment{stepenumerate}{\begin{enumerate}[<+->]}{\end{enumerate}}
\newenvironment{stepitemize}{\begin{itemize}[<+->]}{\end{itemize} }
\newenvironment{stepenumeratewithalert}{\begin{enumerate}[<+-| alert@+>]}{\end{enumerate}}
\newenvironment{stepitemizewithalert}{\begin{itemize}[<+-| alert@+>]}{\end{itemize} }

\newtheorem{question}{Question}
\newtheorem{claim}{Claim}
\newtheorem{proposition}{Proposition}
\newtheorem{remark}{Remark}
\newtheorem{conjecture}{Conjecture}

\definecolor{metrop}{RGB}{29, 44, 44}
\colorlet{GrayLight}{black!15}
\colorlet{GrayMedium}{black!30}
\colorlet{ForestGreen}{green!60!black}

\newenvironment{transitionframe}{
  \setbeamercolor{background canvas}{bg=black!80}
  \begin{frame}}{
    \end{frame}
}

\newcommand{\br}{

\bigskip

}

\newcommand{\pd}{\partial}
\newcommand{\RR}{\mathbb{R}}

\newcommand*\hugme[1]{\tikz[baseline=(char.base)]{\node[shape=ellipse,draw,inner sep=0pt] (char) {#1};}}

\newcounter{saveenumi}
\newcommand{\seti}{\setcounter{saveenumi}{\value{enumi}}}
\newcommand{\conti}{\setcounter{enumi}{\value{saveenumi}}}
\resetcounteronoverlays{saveenumi}



\newcommand\dotprod[2]{\langle #1 , #2 \rangle}
\newcommand{\ft}[1]{\widehat #1}
\newcommand{\qabove}[1]{\overset{\text{\large \textbf ?}}{#1}}
\newcommand{\eqae}{\overset{\text{a.e.}}{=}}
\newcommand{\calp}{\mathcal{P}}
\newcommand{\calg}{\mathcal{G}}
\newcommand{\calb}{\mathcal{B}}
\newcommand{\textd}{\text{d}}
\newcommand{\bbr}{\mathbb{R}}
\newcommand{\binm}{\mathbin{M}}
\newcommand{\binc}{\mathbin{C}}
\newcommand{\binb}{\mathbin{B}}
\newcommand{\calc}{\mathcal{C}}
\newcommand{\calh}{\mathcal{H}}
\newcommand{\bfone}{\mathbf{1}}
\newcommand{\bbe}{\mathbb{E}}
\newcommand{\bfle}{\mathbf{e}}
\newcommand{\calf}{\mathcal{F}}
\newcommand{\cala}{\mathcal{A}}
\newcommand{\cale}{\mathcal{E}}
\newcommand{\bbn}{\mathbb{N}}
\newcommand{\cantor}{\calc}
\newcommand{\calY}{\mathcal{Y}}
\newcommand{\textb}{\text{B}}
\newcommand{\calm}{\mathcal{M}}
\newcommand{\bint}{\mathbin{T}}
\newcommand{\ep}{\epsilon}
\newcommand{\bbq}{\mathbb{Q}}
\newcommand{\bbp}{\mathbb{P}}
\newcommand{\cals}{\mathcal{S}}
\newcommand{\emptysequence}{e}
\newcommand{\bbz}{\mathbb{Z}}
\newcommand{\fraka}{\frak{A}}
\newcommand{\frakb}{\frak{B}}
\newcommand{\length}{\text{length}}
\newcommand{\bfn}{\mathbf{N}}
\newcommand{\support}{\text{support}}
\DeclareMathOperator*{\argmax}{arg\,max}
\newcommand{\dom}{\mbox{dom}}
\def\ut{\underline t}
\def\um{\underline m}
\def\PP{\mathbb{P}}
\def\EE{\mathbb{E}}
\def\RR{\mathbb{R}}
\usepackage{copyrightbox}


% Title page info
\title[Popular Tests]{ExpEcon Methods:\\Popular Hypothesis Tests}
\author[ECON 8877]{ECON 8877\\P.J. Healy\\First version of distributional test slides thanks to Floyd Carey} \color{metrop}
\institute[OSU]{}
\date[]{\vfill {\tiny Updated \today}}

\begin{document}

\frame{\maketitle}

% \begin{frame}{To Add}
% Stuff to add:
% \begin{enumerate}
%     \item You can't test the hypothesis that a proportion is significantly different from zero.
% \end{enumerate}
    
% \end{frame}

\begin{frame}{Siegel \& Castellan's Book}
\vfill
Siegel \& Castellan (1988) \textit{Nonparametric Statistics for the Behavioral Sciences}, 2nd ed.
\vfill
\end{frame}

\begin{frame}{Siegel \& Castellan's Book}
The back jacket of 2nd Edition:
\br

What do your data look like?
\begin{enumerate}
    \item Nominal/Categorical
    \begin{itemize}
        \item Pass/fail, gender, race...
    \end{itemize}
    \item Ordinal
    \begin{itemize}
        \item Type/ability
    \end{itemize}
    \item Interval
    \begin{itemize}
        \item Score on a test
    \end{itemize}
\end{enumerate}
\end{frame}

\begin{frame}{Siegel \& Castellan's Book}
The back jacket of 2nd Edition:
\br

What do you want to test?
\begin{enumerate}
    \item One Sample
    \begin{itemize}
        \item Value of a statistic ($\mu=0$)
        \item Fit of a sample to a distribution ($X\sim N(0,1)$)
        \item Properties of a sample (runs test, symmetry test)
    \end{itemize}
    \item Comparing Two or More Samples
    \begin{enumerate}
        \item Matched samples
        \begin{itemize}
            \item ``Sample of differences'' ($\mu_{diff}=0$)
        \end{itemize}
        \item Independent samples
        \begin{itemize}
            \item Comparing statistics ($\mu_1 = \mu_2$)
        \end{itemize}
    \end{enumerate}
    \item Measuring Association Between Two Samples
    \begin{enumerate}
        \item Various notions of ``correlation''
    \end{enumerate}
\end{enumerate}
\end{frame}

\begin{frame}{Contingency Tables}
Comparing samples with categorical data\\
(or ordinal, discarding order info)

\begin{center}
    \begin{tabular}{c|c|c|}
         Category & Control & Treatment \\   
         \cline{2-3}
         High & A & B  \\
         \cline{2-3}
         Low & C & D \\
         \cline{2-3}
    \end{tabular}
\end{center}
\br
Let $P_1=A/C$ and $P_2=B/D$.\\
$H_0$: $P_1 = P_2$ (category is independent of treatment)\\
\textbf{Fisher's Exact Test:}\\
Prob of $(A,B,C,D)$ under $H_0$: Hypergeometric dist'n\\
1-Tail: Calculate prob of all tables with $P_1-P_2$ bigger than observed\\
2-Tail: Calculate prob of all tables with $|P_1-P_2|$ bigger than observed\\
\emph{among tables with the same row \& column sums}.\\
Exact test since sampling distribution is known for any $n$
\br
Problem: calculation intensive! Only for small tables.
\end{frame}

\begin{frame}{Contingency Tables}
The Chi-Squared Test (for contingency tables):\\
\begin{center}
    \begin{tabular}{c|c|c||c|}
         OBSERVED & Control & Trt & Pooled\\   
         \cline{2-4}
         Bet A & 166 & 107 & 273 (66\%) \\
         \cline{2-4}
         Bet B & 91 & 49 & 140 (34\%) \\
         \cline{2-4}
         TOTAL & 257 & 156 & 413\\
         \cline{2-4}
    \end{tabular}
    \br
    \begin{tabular}{c|c|c|}
         EXPECTED & Control & Trt \\   
         \cline{2-3}
         Bet A & 169.9 & 103.1  \\
         \cline{2-3}
         Bet B & 87.1 & 52.9 \\
         \cline{2-3}
    \end{tabular}
    \br
    \begin{tabular}{c|c|c|c}
         $(O-E)^2/E$ & Control & Trt &  \\   
         \cline{2-3}
         Bet A & 0.09 & 0.15 & Sum = 0.693 \\
         \cline{2-3}
         Bet B & 0.17 & 0.28 & $p$-val = 0.405\\
         \cline{2-3}         
    \end{tabular}
\end{center}
\br
Test statistic $T = \sum \frac{(O-E)^2}{E}$. As $n\rightarrow \infty$ we have $T\sim \chi^2_{(r-1)(c-1)}$
    
\end{frame}

\begin{frame}{Contingency Tables}
Partitioning the D.O.F.\\
\begin{center}
    \begin{tabular}{c|c|c|c|}
         OBSERVED & Black & White & Asian \\   
         \cline{2-4}
         Pass & 70 & 70 & 30 \\
         \cline{2-4}
         Fail & 30 & 30 & 70 \\
         \cline{2-4}
    \end{tabular}
\end{center}
$\chi^2$ test rejects $H_0$. But which race is different?\\
Tempted to test all $2\times 2$ subtables, but they're not independent
\br
\begin{center}
\begin{tabular}{cc}
         \begin{tabular}{c|c|c|}
          & Black & White \\   
         \cline{2-3}
         Pass & 70 & 70  \\
         \cline{2-3}
         Fail & 30 & 30 \\
         \cline{2-3}
    \end{tabular}
    &
    \begin{tabular}{c|c|c|}
          & Black + White & Asian \\   
         \cline{2-3}
         Pass & 140 & 30  \\
         \cline{2-3}
         Fail & 60 & 70 \\
         \cline{2-3}
    \end{tabular}
\end{tabular}
\end{center}
As many subtables as there are d.o.f.
\end{frame}

\begin{frame}{Fisher vs. Chi-Squared}
    \begin{itemize}
        \item Use Fisher if your computer can do it
        \item Chi-Squared test: invalid of $E\leq 5$ in any cell
        \begin{itemize}
            \item Combine cells?
            \item Continuity correction (maybe automatic)
        \end{itemize}
    \end{itemize}
\end{frame}

\begin{frame}{Binomial/Proportion Test}
What fraction of people passed this test? $H_0$: $p=p_0$\\
$x_i \in \{0,1\}$, $i=1,2,\ldots,n$\\
$y=\sum_i x_i$, $\hat{p}=y/n$
\br
Recall binomial distribution: $Pr[Y=k] = {n \choose k} p_0^k (1-p_0)^{n-k}$\\
One-sided test: $Pr[Y\geq y] = \sum_{k=y}^n {n \choose k} p_0^k (1-p_0)^{n-k}$\\
Two-sided test (if $y>p_0n$): $Pr[Y\geq y] + Pr[Y\leq p_0n-(y-p_0n)]$\\
\br
Large samples: use Normal approximation w/ continuity correction
\br
\textbf{NOTE: You cannot test $H_0:p=0$ or $H_0:p=1$!}\\
Not with classical hypothesis testing, anyway.\\
Confidence interval is \textbf{not} a solution: In $(0,1)$ by construction
\end{frame}

\begin{frame}{Tests of Association}
Requires paired data! $n=m$
    \begin{itemize}
        \item Pearson (classic parametric test)
        $$
        r_{X,Y} = \frac{\sum_i (x_i-\bar{x})(y_i-\bar{y})}{\sqrt{\sum_i (x_i-\bar{x})^2} \sqrt{\sum_i(y_i-\bar{y})^2}} = \frac{\hat{Cov}(X,Y)}{\hat{\sigma}(X)\hat{\sigma}(Y)} \in [-1,1]
        $$
        \begin{itemize}
            \item Tests for \textit{linear} relationship between $X$, $Y$.
            \item Normal case:
            \begin{itemize}
                \item S.E. is $\sigma_r=\sqrt{\frac{1-r^2}{n-2}}$. Conf. interval.
                \item Also $r/\sigma_r \sim t_{n-1}$, so can get a $p$-value
            \end{itemize}
            \item If non-normal:
            \begin{itemize}
                \item Conf interval? Bootstrap. Redraw pairs $(x_i,y_i)$ w/ replacement.
                \item $p$-value for $H_o:r=0$? Perm test: Redraw $(x_i,y_{\pi(i)})\ \forall \pi$
            \end{itemize}
        \end{itemize}
    \end{itemize}
\end{frame}

\begin{frame}{Tests of Association}
    \begin{itemize}
        \item Spearman rank-order coefficient (non-parametric)
        \begin{itemize}
            \item Simply Pearson, but data are converted to ranks
            \begin{itemize}
                \item Just convert each $x_i$ to $R(x_i)\in\{1,\ldots,n\}$
                \item Ties? Use average of ranks among ties
            \end{itemize}
            \item Now testing a \sout{linear} \textit{monotonic} relationship
            \begin{itemize}
                \item Can have Spearman=1 but Pearson$<$ 1
            \end{itemize}
            \item Still in $[-1,1]$
            \item Conf interval? Bootstrap or Jackknife. $\exists$ packages
            \item $p$-value? Permutation test, redrawing $(x_i,y_{\pi(i)})$.
        \end{itemize}
    \end{itemize}
\end{frame}

\begin{frame}{Tests of Association}
    \begin{itemize}
        \item Kendall rank correlation
        \begin{itemize}
            \item Interval or ordinal
            \item Form pairs $(x_i,y_i)$ vs. $(x_j,y_j)$
            \item Define:
            \begin{itemize}
                \item $n_c$ = \# of pairs that ``move together'' ($x_j>x_i$ \& $y_j>y_i$)
                \item $n_d$ = \# of pairs that ``move oppositely''
                \item $n_1$ = \# of pairs where $x_i=x_j$
                \item $n_2$ = \# of pairs whre $y_i=y_j$
                \item $\bar{n}=\frac{n(n-1)}{2}=\sum_{i=1}^{n-1} i$
            \end{itemize}
            \item Test statistic: $\tau = \frac{n_c-n_d}{\sqrt{(\bar{n}-n_1)(\bar{n}-n_2)}} \in [-1,1]$
            \item Distribution of $\tau$ under $H_0$ known for small $n$
            \begin{itemize}
                \item Approximately normal for large $n$
                \item Analytical sol'ns for conf interval or $p$-value
            \end{itemize}
            \item Preferred to Spearman for small $n$ or outliers
        \end{itemize}
        \item Cramer
        \begin{itemize}
            \item Contingency tables
            \item Simply a rescaling of the $\chi^2$ statistic to $[0,1]$
        \end{itemize}
    \end{itemize}
    
\end{frame}


%----------------------------------------------------------------------------------------
%\section{Distributional Tests}
%----------------------------------------------------------------------------------------

\begin{frame}
\frametitle{Distributional Tests}

Distributional tests determine how likely a sample is to have come from a pre-specified distribution or how likely two samples are to have been drawn from the same distribution.

\vspace{.25in}

The most well-known (and general) of these tests is the Kolmogorov-Smirnov (KS) test.
\end{frame}

%----------------------------------------------------------------------------------------

\subsection{The One-Sample Kolmogorov-Smirnov Test}

\begin{frame}
\frametitle{One-Sample Kolmogorov-Smirnov Test}
\begin{itemize}
    \item The one-sample KS test compares the cumulative distribution of a sample of size n ($S_n(x)$) to a pre-specified cumulative distribution function ($F_0(x)$).
\end{itemize}
$$S_n(x) = k/n$$
\begin{itemize}
    \item where $k$ = the number of observations $\leq x$.
\end{itemize}
\end{frame}

%----------------------------------------------------------------------------------------

\begin{frame}
\frametitle{One-Sample Kolmogorov-Smirnov Test (Continued)}
\begin{itemize}
    \item The test is based entirely on the \textit{largest} deviation between $F_0(x)$ and $S_n(x)$, denoted as $D_n$.
\end{itemize}
\vspace{.25in}
$$ D_n = \sup_{x} |F_0(x) - S_n(x)| $$
\begin{itemize}
    \item Under the null hypothesis that the sample is drawn from $F_0(x)$, $\lim_{n \to \infty}D_n = 0.$
    \item the null hypothesis is rejected if $\sqrt{n}D_n > K_{\alpha}$, where $K_{\alpha}$ is found such that $Pr(K \leq K_{\alpha}) = 1 - \alpha$ and $K$ is the Kolmogorov distribution.
\end{itemize}
\end{frame}

%----------------------------------------------------------------------------------------

\begin{frame}
\frametitle{One-Sample Kolmogorov-Smirnov Test Figure}
    \begin{figure}
        \centering
         \includegraphics[scale=.70]{LectureSlides/graphics/mht/One_Sample_KS_Example.png}
         \caption*{The red line is $F_0(x)$, the blue line is $S_n(x)$, and the black line is $D_n$ (Bscan, CC0, via Wikimedia Commons)}
    \end{figure}
\end{frame}

%----------------------------------------------------------------------------------------

\begin{frame}
\frametitle{Example: One-Sample Kolmogorov-Smirnov Test}
\begin{itemize}
    \item Suppose that there are 5 different salsas where each subsequent salsa is spicier (i.e., the salsa denoted by $x_{n+1}$ is spicier than the salsa denoted by $x_n$).
    \item Further, suppose that the null hypothesis is that preferences over salsa spiciness is uniformly distributed in the population (i.e., $F_0(x) = \frac{x}{5}).$
    \item In a sample of 10 subjects ($n$ = 10), one subject prefers the least spicy salsa, 5 subjects prefer the second most spicy salsa, and 4 subjects prefer the spiciest salsa.
\end{itemize}
\end{frame}

%----------------------------------------------------------------------------------------

\begin{frame}
\frametitle{Example: One-Sample Kolmogorov-Smirnov Test (Continued)}
\begin{itemize}
    \item The difference between $F_0(x)$ and $S_{10}(x)$ is maximized at $x = 3$.
    \begin{itemize}
        \item $F_0(3) = \frac{3}{5}$ and $S_{10}(3) = \frac{1}{10}$.
    \end{itemize}
    \item Therefore, $D_n = \frac{3}{5} - \frac{1}{10} = .5$.
    \item $\sqrt{n}D_n = \sqrt{10} \cdot .5 = 1.581$
    \item $K_{.01} = .48895$
    \item Because $\sqrt{n}D_n = 1.581 > .48895 = K_{.01}$, we can reject the null hypothesis that the sample was drawn from a population whose preferences for salsa spiciness was uniformly distributed with 99\% confidence.
\end{itemize}
\end{frame}

%----------------------------------------------------------------------------------------

\subsection{Alternatives to the One-Sample KS Test}

\begin{frame}
\frametitle{Alternatives to the One-Sample KS Test}
\begin{itemize}
    \item Another test, which is based on the quadratic difference between the pre-specified distribution instead of the maximum difference is the Anderson-Darling (AD) test (Anderson \& Darling, 1952).
    \begin{itemize}
        \item The AD test is a modification of the Cramer-von Mises (CVM) test (1928).
    \end{itemize}
    \item The test statistic is:
\end{itemize}
\vspace{.25in}
$$
W_n^2=n \int_{-\infty}^{\infty}[S_n(x)-F_0(x)]^2 \psi(F_0(X)) d F_0(x)
$$
\begin{itemize}
    \item Where $\Psi = [F_0(X)(1-F_0(X))]^{-1}$
\end{itemize}
\end{frame}

%----------------------------------------------------------------------------------------

\begin{frame}
\frametitle{Power Comparison for KS and AD tests}
\begin{columns}
    \column{0.5\textwidth}
    \centering
        \copyrightbox[b]{\includegraphics[scale=0.4]{LectureSlides/graphics/mht/AD_KS_Figure1.JPG}}%
                  {Boyerinas (2016)}
    \column{0.5\textwidth}
    \centering
    \copyrightbox[b]{\includegraphics[scale=0.4]{LectureSlides/graphics/mht/AD_KS_Figure2.JPG}}%
                  {Boyerinas (2016)}
\end{columns}
\end{frame}

%----------------------------------------------------------------------------------------

\begin{frame}
\frametitle{Other Alternatives to the One-Sample KS Test}
\begin{itemize}
    \item Suppose you want to test ``my data came from a normal distribution'' but you don't know $\mu$ or $\sigma^2$.
    \item You have a few options: The Lillefors (LF) test (Lillefors, 1967) and the Shapiro-Wilk (SW) test (Shapiro \& Wilk, 1965).
    \item The SW test is the most powerful, and the LF test is the least powerful for a broad range of normal distributions (Razali \& Wah, 2011).
\end{itemize}
\end{frame}

%----------------------------------------------------------------------------------------

\begin{frame}
\frametitle{The Shapiro-Wilk Test}
\begin{itemize}
    \item The SW test uses the test statistic:
\end{itemize}
\vspace{.25in}
$$
W=\frac{(\sum_{i=1}^n a_i x_{(i)})^2}{\sum_{i=1}^n(x_i-\bar{x})^2}
$$
\begin{itemize}
    \item $\bar{x}$ is the sample mean
    \item and $\boldsymbol{a}_i=(a_1, \ldots, a_n)=\frac{m^T V^{-1}}{(m^T V^{-1} V^{-1} m)^{1 / 2}}$
\end{itemize}
\end{frame}

%----------------------------------------------------------------------------------------


\begin{frame}
\frametitle{The Shapiro-Wilk Test (Continued)}
\begin{itemize}
    \item where $\boldsymbol{m}_i=(m_1, \ldots, m_n)^T$ are the expected values of order statistics of independent and identically distributed random variables sampled from the standard normal distribution
    \item and $\boldsymbol{V}$ is the covariance matrix of those order statistics.
    \item m is computed using GLS, assuming that x is normally distributed.
\end{itemize}
$$
\begin{aligned}
& \hat{\mu}=\frac{m^{\prime} V^{-1}(m 1^{\prime}-1 m^{\prime}) V^{-1} x}{1^{\prime} V^{-1} 1 m^{\prime} V^{-1} m-(1^{\prime} V^{-1} m)^2} \\
& \hat{\sigma}=\frac{1^{\prime} V^{-1}(1 m^{\prime}-m 1^{\prime}) V^{-1} x}{1^{\prime} V^{-1} 1 m^{\prime} V^{-1} m-(1^{\prime} V^{-1} m)^2}
\end{aligned}
$$
\begin{itemize}
    \item In practice, $\boldsymbol{a}_i$ is algorithmically approximated using Royston's (1994) AS R94 for $3 \leq n \leq 5000$.
\end{itemize}
\end{frame}

%----------------------------------------------------------------------------------------

\begin{frame}
\frametitle{Power Comparison for KS, LF, AD, and SW Tests}
\centering
\begin{tabular}{cc}
     \includegraphics[scale=0.4]{LectureSlides/graphics/mht/AD_KS_LF_SW_Figure2.JPG} & \includegraphics[scale=0.4]{LectureSlides/graphics/mht/AD_KS_LF_SW_Figure1.JPG} \\
     Razali \& Wah, 2011
     & Razali \& Wah, 2011\\
\end{tabular}
\end{frame}

%----------------------------------------------------------------------------------------
\subsection{The Two-Sample Kolmogorov-Smirnov Test}

\begin{frame}
\frametitle{The Two-Sample Kolmogorov-Smirnov Test}
\begin{itemize}
    \item Calculating the two-sample Kolmogorov-Smirnov test is similar to the one-sample counterpart except we replace $F_0(x)$ with $S_{m}(x)$, where the second sample has $m$ members.
    \item Here, $D_{n, m}=\sup_x|S_{n}(x)-S_{m}(x)|$ for the two-sided test and $D_{n, m}=\sup_x [S_{n}(x)-S_{m}(x)]$ for the one-sided test.
    \item Siegel (1988) uses a heuristic that if $n$ or $m$ are less than 40, then $n$ must equal $m$, but I have not found this in other papers.
\end{itemize}
\end{frame}

%----------------------------------------------------------------------------------------


\begin{frame}
\frametitle{Two-Sample Kolmogorov-Smirnov Test Figure}
    \begin{figure}
        \centering
         \includegraphics[scale=.70]{LectureSlides/graphics/mht/Two_Sample_KS_Example.png}
         \caption*{The red line is $S_m(x)$, the blue line is $S_n(x)$, and the black line is $D_n$ (Bscan, CC0, via Wikimedia Commons)}
    \end{figure}
\end{frame}

%----------------------------------------------------------------------------------------

\begin{frame}
\frametitle{The Two-Sample Kolmogorov-Smirnov Test (Continued)}
\begin{itemize}
    \item The two-sample test has different critical values from the one-sample test, but I don't think there is an analytical solution for small samples (I couldn't find one if there is!)
    \item There are tables for small samples, and for larger samples, the equation for the critical value is $c(\alpha)\sqrt{\frac{n+m}{n \cdot m}}$
    \item where $c(\alpha) = \sqrt{-ln(\frac{\alpha}{2})\cdot \frac{1}{2}}$ (Knuth, 1998)
\end{itemize}
\end{frame}

%----------------------------------------------------------------------------------------

\subsection{Alternatives to the Two-Sample KS Test}

\begin{frame}
\frametitle{Alternatives to the Two-Sample KS Test}
\begin{itemize}
    \item Nearly all of the alternatives to the two-sample KS test are location-scale tests which incorporate both the sample means and standard deviations. The two most popular of this class are the Cucconi (C) test (1968) and the Lepage (L) test (1971).
    \item Both the C and the L tests are \textbf{FAR} more powerful than the Two-Sample KS Test in a simulation using several canonical distributions (Marozzi, 2009).
    \begin{itemize}
        \item That same paper indicates that the C test is slightly more powerful than the L test.
    \end{itemize}
\end{itemize}
\end{frame}

%----------------------------------------------------------------------------------------

\begin{frame}
\frametitle{Alternatives to the Two-Sample KS Test (Continued)}
\begin{itemize}
    \item This increase in power for location-scale tests is  partially due to their assumptions on the alternative hypothesis.
    \item In location-scale tests, $H_0$ is that $F \equiv G$ and $H_a$ is that $G(y) = F(ay + b)$ such that $a \neq 1$ and/or $b \neq 0$.
    \item In the two-sample KS test, $H_0$ is unchanged, but $H_a$ does not specify an alternative distribution.
\end{itemize}
\end{frame}

%----------------------------------------------------------------------------------------

\begin{frame}
\frametitle{Power Comparison for KS, C, and L Tests}
    \centering
    \includegraphics[scale=.45]{LectureSlides/graphics/mht/KS_C_L_Figure1.JPG}\\
    Marozzi, 2009
\end{frame}


\end{document}