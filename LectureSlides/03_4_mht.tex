\documentclass[10pt,t,english]{beamer}
\usepackage{fontawesome}
\usepackage{graphicx}
\usepackage{array}
\usepackage[normalem]{ulem}
\usepackage{amsfonts,amsmath,amssymb,bm,bbm}
\usepackage{mathrsfs}
\usepackage{sgame}
\usepackage{graphicx,pstricks}
\usepackage{xcolor}
\usepackage{colortbl}
\usepackage{makecell}
\usepackage{tikz,tikzsymbols,gnuplottex}
\usetikzlibrary{decorations.pathreplacing,shapes}
\usepackage[english]{babel}
\usepackage[utf8]{inputenc}
\usepackage{appendixnumberbeamer}
\usepackage{datetime2}
\usepackage{booktabs}
\usepackage{setspace}
\usepackage{rotating}
\usepackage{natbib}
\usepackage{listings}

%\usepackage{matlab-prettifier} % For enhanced MATLAB highlighting
\lstset{
    language=matlab,
    % Other options for styling (optional)
    basicstyle=\ttfamily\footnotesize, % Font style
    keywordstyle=\color{blue}, % Keyword color
    commentstyle=\color{green!50!black}, % Comment color
    stringstyle=\color{red!70!black}, % String color
    numbers=left, % Line numbers on the left
    numberstyle=\tiny\color{gray}, % Style for line numbers
    frame=single, % Frame around the listing
    breaklines=true, % Allow line breaking
    captionpos=b, % Caption at the bottom
    tabsize=4 % Tab size
}

% ------------------------------------------------------------------------------
% Use the beautiful metropolis beamer template
% ------------------------------------------------------------------------------
\usepackage[T1]{fontenc}
\usepackage[utf8]{inputenc}
\usepackage{fontawesome}
\usepackage{FiraSans} 
\mode<presentation>
{
  \usetheme[progressbar=foot,background=light]{metropolis} 
  \usecolortheme{default} % or try albatross, beaver, crane, ...
  \usefonttheme{default}  % or try default, serif, structurebold, ...
  \setbeamertemplate{navigation symbols}{}
  \setbeamertemplate{caption}[numbered]
  %\setbeamertemplate{frame footer}{My custom footer}
} 

\newenvironment{stepenumerate}{\begin{enumerate}[<+->]}{\end{enumerate}}
\newenvironment{stepitemize}{\begin{itemize}[<+->]}{\end{itemize} }
\newenvironment{stepenumeratewithalert}{\begin{enumerate}[<+-| alert@+>]}{\end{enumerate}}
\newenvironment{stepitemizewithalert}{\begin{itemize}[<+-| alert@+>]}{\end{itemize} }

\newtheorem{question}{Question}
\newtheorem{claim}{Claim}
\newtheorem{proposition}{Proposition}
\newtheorem{remark}{Remark}
\newtheorem{conjecture}{Conjecture}

\definecolor{metrop}{RGB}{29, 44, 44}
\colorlet{GrayLight}{black!15}
\colorlet{GrayMedium}{black!30}
\colorlet{ForestGreen}{green!60!black}

\newenvironment{transitionframe}{
  \setbeamercolor{background canvas}{bg=black!80}
  \begin{frame}}{
    \end{frame}
}

\newcommand{\br}{

\bigskip

}

\newcommand{\pd}{\partial}
\newcommand{\RR}{\mathbb{R}}

\newcommand*\hugme[1]{\tikz[baseline=(char.base)]{\node[shape=ellipse,draw,inner sep=0pt] (char) {#1};}}

\newcounter{saveenumi}
\newcommand{\seti}{\setcounter{saveenumi}{\value{enumi}}}
\newcommand{\conti}{\setcounter{enumi}{\value{saveenumi}}}
\resetcounteronoverlays{saveenumi}



\newcommand\dotprod[2]{\langle #1 , #2 \rangle}
\newcommand{\ft}[1]{\widehat #1}
\newcommand{\qabove}[1]{\overset{\text{\large \textbf ?}}{#1}}
\newcommand{\eqae}{\overset{\text{a.e.}}{=}}
\newcommand{\calp}{\mathcal{P}}
\newcommand{\calg}{\mathcal{G}}
\newcommand{\calb}{\mathcal{B}}
\newcommand{\textd}{\text{d}}
\newcommand{\bbr}{\mathbb{R}}
\newcommand{\binm}{\mathbin{M}}
\newcommand{\binc}{\mathbin{C}}
\newcommand{\binb}{\mathbin{B}}
\newcommand{\calc}{\mathcal{C}}
\newcommand{\calh}{\mathcal{H}}
\newcommand{\bfone}{\mathbf{1}}
\newcommand{\bbe}{\mathbb{E}}
\newcommand{\bfle}{\mathbf{e}}
\newcommand{\calf}{\mathcal{F}}
\newcommand{\cala}{\mathcal{A}}
\newcommand{\cale}{\mathcal{E}}
\newcommand{\bbn}{\mathbb{N}}
\newcommand{\cantor}{\calc}
\newcommand{\calY}{\mathcal{Y}}
\newcommand{\textb}{\text{B}}
\newcommand{\calm}{\mathcal{M}}
\newcommand{\bint}{\mathbin{T}}
\newcommand{\ep}{\epsilon}
\newcommand{\bbq}{\mathbb{Q}}
\newcommand{\bbp}{\mathbb{P}}
\newcommand{\cals}{\mathcal{S}}
\newcommand{\emptysequence}{e}
\newcommand{\bbz}{\mathbb{Z}}
\newcommand{\fraka}{\frak{A}}
\newcommand{\frakb}{\frak{B}}
\newcommand{\length}{\text{length}}
\newcommand{\bfn}{\mathbf{N}}
\newcommand{\support}{\text{support}}
\DeclareMathOperator*{\argmax}{arg\,max}
\newcommand{\dom}{\mbox{dom}}
\def\ut{\underline t}
\def\um{\underline m}
\def\PP{\mathbb{P}}
\def\EE{\mathbb{E}}
\def\RR{\mathbb{R}}

% Title page info
\title{ExpEcon Methods:\\Multiple Hypotheses Corrections}
\author[ECON 8877]{ECON 8877\\P.J. Healy\\First version thanks to Floyd Carey} \color{metrop}
\institute[OSU]{}
\date[]{\vfill {\tiny Updated \today}}

\begin{document}


\frame{\maketitle}




%----------------------------------------------------------------------------------------
%	PRESENTATION SLIDES
%----------------------------------------------------------------------------------------

%----------------------------------------------------------------------------------------
\section{Multiple Hypothesis Corrections}
%----------------------------------------------------------------------------------------

\begin{frame}{Multiple Tests}
The setup:
\begin{itemize}
    \item You're going to run $m$ tests (say, $m=2$)
    \item Each test on its own has a 0.05 Type-I error
    \item The game is that you ``win'' if \textit{at least} one is significant
    \begin{itemize}
        \item Publish a paper, claim a result, etc.
    \end{itemize}
    \item Problem: $\uparrow m \Rightarrow \uparrow$ chance you win
\end{itemize}
\begin{definition}
    The \textbf{Family-Wise Error Rate (FWER)} is the probability that you ``win'' given that all $m$ null hypotheses are true.
\end{definition}
\begin{itemize}
    \item We want FWER to be 0.05, not each test
\end{itemize}
    
\end{frame}

\begin{frame}{Multiple Tests}
Suppose you run two tests and {\color{red}``\textbf{win}''} if one is significant\\
Each has 0.05 Type-I error. Baseline case: independent tests
\begin{center}
    \textbf{Independent} tests:\\
     \begin{tabular}{r|c|c|l}
          \multicolumn{1}{r}{ } & \multicolumn{1}{c}{Accept} & \multicolumn{1}{c}{Reject} & \\
          \cline{2-3}
           Accept & {\color{olive}0.9025} & {\color{red} 0.0475} & 0.95 \\
          \cline{2-3}
          Reject & {\color{red} 0.0475} & {\color{red} 0.0025} & 0.05 \\
          \cline{2-3}
          \multicolumn{1}{r}{ }  & \multicolumn{1}{c}{0.95}  & \multicolumn{1}{c}{0.05} &  {\color{red}$FWER=0.0975$}
     \end{tabular}
\end{center}
{\color{blue}\textbf{\u{S}id\'{a}k correction}}: use a lower $\alpha$:
\begin{center}
     \begin{tabular}{r|c|c|l}
          \multicolumn{1}{r}{ } & \multicolumn{1}{c}{Accept} & \multicolumn{1}{c}{Reject} & \\
          \cline{2-3}
           Accept & {\color{olive}$(1-\alpha)^2$} & {\color{red} $(1-\alpha)\alpha$} & $1-\alpha$ \\
          \cline{2-3}
          Reject & {\color{red} $(1-\alpha)\alpha$} & {\color{red} $\alpha^2$} & $\alpha$ \\
          \cline{2-3}
          \multicolumn{1}{r}{ }  & \multicolumn{1}{c}{$1-\alpha$}  & \multicolumn{1}{c}{$\alpha$} &  {\color{red}$FWER=2\alpha - \alpha^2$}
     \end{tabular}
\end{center}
For $FWER=0.05$ use $\alpha \approx 0.025321$. If you have $k$ tests:\\
$1-(1-\alpha)^k=0.05 \Rightarrow {\color{blue}\alpha^* = 1-(1-0.05)^{1/k}}$ which is $>0.05/k$
\end{frame}


\begin{frame}{Multiple Tests}
Worst-case situation: perfectly negative correlation\\
Need a bigger correction!
\begin{center}
    \textbf{Perfect Negative Correlation}:\\
     \begin{tabular}{r|c|c|l}
          \multicolumn{1}{r}{ } & \multicolumn{1}{c}{Accept} & \multicolumn{1}{c}{Reject} & \\
          \cline{2-3}
           Accept & {\color{olive}0.90} & {\color{red} 0.05} & 0.95 \\
          \cline{2-3}
          Reject & {\color{red} 0.05} & {\color{red} 0} & 0.05 \\
          \cline{2-3}
          \multicolumn{1}{r}{ }  & \multicolumn{1}{c}{0.95}  & \multicolumn{1}{c}{0.05} &  {\color{red}$Pr(R)=0.10$}
     \end{tabular}
\end{center}
{\color{blue}\textbf{Bonferroni correction}}: what's the right $\alpha$?
\begin{center}
     \begin{tabular}{r|c|c|l}
          \multicolumn{1}{r}{ } & \multicolumn{1}{c}{Accept} & \multicolumn{1}{c}{Reject} & \\
          \cline{2-3}
           Accept & {\color{olive}$1-2\alpha$} & {\color{red} $\alpha$} & $1-\alpha$ \\
          \cline{2-3}
          Reject & {\color{red} $\alpha$} & {\color{red} $0$} & $\alpha$ \\
          \cline{2-3}
          \multicolumn{1}{r}{ }  & \multicolumn{1}{c}{$1-\alpha$}  & \multicolumn{1}{c}{$\alpha$} &  {\color{red}$Pr(R)=2\alpha$}
     \end{tabular}
\end{center}
For $Pr(R)=0.05$ use $\alpha \approx 0.025$\\
$k$ tests: $1-(1-k\alpha)=0.05 \Rightarrow k\alpha=0.05\Rightarrow {\color{blue}\alpha^*=0.05/k}$
\end{frame}


\begin{frame}{Multiple Tests}
Best case: perfect positive correlation\\
Extra tests don't add to the FWER!
\begin{center}
    \textbf{Perfect Positive Correlation}:\\
     \begin{tabular}{r|c|c|l}
          \multicolumn{1}{r}{ } & \multicolumn{1}{c}{Accept} & \multicolumn{1}{c}{Reject} & \\
          \cline{2-3}
           Accept & {\color{olive}0.95} & {\color{red} 0} & 0.95 \\
          \cline{2-3}
          Reject & {\color{red} 0} & {\color{red} 0.05} & 0.05 \\
          \cline{2-3}
          \multicolumn{1}{r}{ }  & \multicolumn{1}{c}{0.95}  & \multicolumn{1}{c}{0.05} &  {\color{red}$Pr(R)=0.05$}
     \end{tabular}
\end{center}
No correction needed!
\begin{center}
     \begin{tabular}{r|c|c|l}
          \multicolumn{1}{r}{ } & \multicolumn{1}{c}{Accept} & \multicolumn{1}{c}{Reject} & \\
          \cline{2-3}
           Accept & {\color{olive}$1-\alpha$} & {\color{red} $0$} & $1-\alpha$ \\
          \cline{2-3}
          Reject & {\color{red} $0$} & {\color{red} $\alpha$} & $\alpha$ \\
          \cline{2-3}
          \multicolumn{1}{r}{ }  & \multicolumn{1}{c}{$1-\alpha$}  & \multicolumn{1}{c}{$\alpha$} &  {\color{red}$Pr(R)=\alpha$}
     \end{tabular}
\end{center}
\u{S}id\'{a}k or Bonferroni would be way too conservative!
\end{frame}


\subsection{Types of Corrections}

\begin{frame}
\frametitle{The Bonferroni Correction}
Setup:
\begin{itemize}
    \item $k$ tests. Nulls: $H_0^1,\ldots,H_0^k$
    \item $\alpha_f$ is your adjusted $p$-value on each
    \item FWER (Family-Wise Error Rate) is $Pr(R)$ on at least one test
\end{itemize}
Bonferonni Correction: $\alpha_f = \alpha/k$
\begin{itemize}
    \item The most popular (and conservative)
    \item Safe: appropriate regardless of correlation
    \item Too safe: likely have FWER $< 0.05$
    \item Tradeoff: high chance of Type-II error (failure to reject false $H_0$)
    %\item Bonferroni has the added advantage of controlling for the per-family error rate, which is the average number of false rejections per family (a more exacting standard) (Frane, 2015).
    %\item The issue with the Bonferroni correction (and all corrections) is that you cannot decrease the probability of a Type-1 error ($\alpha$) without increasing the probability of a Type-2 error ($1-\beta$).
\end{itemize}
\u{S}id\'{a}k Correction: $\alpha_f = 1-(1-\alpha)^{1/k}$
\begin{itemize}
    \item Exact correction for independent tests
    \item In practice, Bonferroni $\approx$ \u{S}id\'{a}k
\end{itemize}
\end{frame}

%----------------------------------------------------------------------------------------

\begin{frame}
\frametitle{The Holm-Bonferroni Correction (Holm, 1979)}
\begin{itemize}
    \item More powerful while keeping FWER $\leq 0.05$
    \item Order the p-values lowest to highest ($p_1 \leq p_2 \leq \ldots \leq p_{k}$). Will test nulls $(H_0^1,H_0^2,\ldots,H_0^k)$ \textit{sequentially}:
\end{itemize}
\pause
\begin{enumerate}
    \item Is $p_1<\frac{\alpha}{k}$?
    \begin{itemize}
        \item Yes: Reject $H_0^1$ and move on to test $H_0^2$.
        \item No: Do not reject any $H_0^i$ (as in Bonferroni). Stop.
        \begin{itemize}
            \item Note: $k-1$ tests remaining, so correction increases to $\frac{\alpha}{k-1}$
        \end{itemize}
    \end{itemize}
    \item Is $p_2 < \frac{\alpha}{k-1}$?
    \begin{itemize}
        \item Yes: Reject $H_0^2$ as well and continue. $k-2$ tests remain.
        \item No: Do not reject $H_0^2$ through $H_0^k$. Stop.
    \end{itemize}
    \item[j.] Is $p_j < \frac{\alpha}{k+1-j}$?
    \begin{itemize}
        \item Yes: Reject $H_0^j$ as well and continue.
        \item No. Do not reject $H_0^j$ through $H_0^k$. Stop.
    \end{itemize}
\end{enumerate}
Recall: Bonferroni also allows only some nulls to be rejected. Same.\\
``Win'' if you reject at least one? Then Holm-Bonferroni = Bonferroni.\\
Can use \u{S}id\'{a}k version assuming independence: $1-(1-\alpha)^{1/(k+1-j)}$
\end{frame}

%----------------------------------------------------------------------------------------

\begin{frame}
\frametitle{The Hotchberg Step-Down Procedure}
\begin{itemize}
    \item Holm-Bonferonni: Reject $H_0^1,\ldots,H_0^j$ where $j$ is the smallest index for which $p_{j+1}\geq \frac{\alpha}{k+1-(j+1)}$
    \begin{itemize}
        \item Reject up to the ``first crossing'' of the threshold
    \end{itemize}
    \item Hotchberg procedure: Reject $H_0^1,\ldots,H_0^j$ where $j$ is the largest index for which $p_j\leq \frac{\alpha}{k+1-j}$
    \begin{itemize}
        \item Reject up to the ``last crossing'' of the threshold
    \end{itemize}
    \item Alternatively, first crossing when working top-to-bottom.
    \item This method is more powerful than the Holm-Bonferroni correction, but it sometimes does not control the FWER (see Dmitrienko et al., 2010 for details).
    \begin{itemize}
        \item Not valid for negative correlation (worst case)
    \end{itemize}
    %\item An issue with the Holm and Hotchberg corrections is that neither are balanced.
\end{itemize}
\end{frame}

%----------------------------------------------------------------------------------------

% \begin{frame}
% \frametitle{Issues with the Holm and Hotchberg Corrections}
% \begin{itemize}
%     \item They are not balanced, so that there is the potential for a rejection of $H_0$ for one test which has a higher unadjusted p-value than another test whose null hypothesis is not rejected.
%     \item Romano and Wolf's (2005,2010) method deals with these issues and creates a correction that is more powerful than either the Holm or Hotchberg corrections.
% \end{itemize}
% \end{frame}

%----------------------------------------------------------------------------------------

\begin{frame}
\frametitle{Balanced Resampling Using Bootstrapping}
\begin{itemize}
    \item How do we know the correlation across tests??
    \item Can be estimated via resampling methods! Hooray!
    \item Romano and Wolf (2005,2005,2016)
    \item This, combined with a ``step-down" method like that used in Holm (1979), creates a more powerful correction.
    \item Furthermore, this method also creates balance, such that all tests contribute equally to error control.
    \item List et al. (2019) develop version of this correction for experiments where treatment is randomly assigned.
\end{itemize}
\br 
I would use these methods!\\
Stata: \texttt{rwolf} can be downloaded
\end{frame}

%----------------------------------------------------------------------------------------

% \begin{frame}
% \frametitle{Miscellaneous}
% \begin{itemize}
%     \item There are even more corrections such as the \v{S}id\'{a}k correction (1967), the Dunnet correction (1964), The Scheff\'{e} (1959) correction, the Tukey (1949) correction, the Harmonic Mean p-value (Good, 1958) and Duncan's new multiple range test (1955), but they are less powerful (I'm pretty sure).
%     \item Furthermore, there are other paradigms through which to consider multiple hypothesis corrections besides the FWER such as the more liberal False Discovery Rate (Benjamini \& Hochberg, 1995) and the more conservative experimentwise error rates (Tukey, 1956).
%     \item I'm not going to get into this stuff, but I figured I'd mention for completeness.
% \end{itemize}
% \end{frame}

%----------------------------------------------------------------------------------------


\begin{frame}
\frametitle{The Family-Wise Error Rate}
\begin{itemize}
    \item What is the ``family'' in the Family-Wise Error Rate? What tests should be ``combined''?
    \begin{itemize}
        \item A ``family'' is (frustratingly) loosely defined, but an intuitive way to think about it is a set of tests whose inference is getting at the same question.
        \item An easy experimental example: suppose you have two treatments and a control group, and you want to determine if either of the treatments increased the mean, so you perform two t-tests. Both of those t-tests constitute a family.
    \end{itemize}
\end{itemize}
\end{frame}

%----------------------------------------------------------------------------------------


%----------------------------------------------------------------------------------------


\subsection{When to Use Corrections?}


%----------------------------------------------------------------------------------------

\begin{frame}
\frametitle{When to Use Corrections?}
\begin{itemize}
    \item Some \sout{people} non-statisticians say we should \textit{never} use them (O'Keefe, 2003; Perneger, 1998; Rothman, 1990)
    \item Other \sout{people} non-statisticians say we should \textit{always} use them (Bennett  et  al., 2009; Goeman \& Solari, 2014; Moy\'{e}, 1998; Ottenbacher, 1998)
    \item Still others say we should use them only in exploratory research (Armstrong, 2014; Cramer et al., 2016; Streiner, 2015)
    \item Finally, some say we should use them only in confirmatory research (Bender \& Lange, 2001; Schochet, 2009; Stacey et al., 2012; Tutzauer, 2003; Wason et al., 2014)
\end{itemize}
\end{frame}

%----------------------------------------------------------------------------------------

\begin{frame}
\frametitle{When to Use Corrections? (Continued)}
\begin{itemize}
    \item MHT corrections in ExpEcon are fairly rare, but growing.
    \item List et al. (2019) argue we \textit{should} correct when:
\end{itemize}
\begin{enumerate}
    \item Multiple outcomes for a given treatment (eg, GPA, SAT, ACT)
    \item Multiple subgroups for given trt (eg, Black, Hispanic, Asian)
    \item Multiple treatments (eg, different incentive schemes on effort)
\end{enumerate}
\br
PJ's view: ``Can your claim be validated by \emph{any} test being significant?''\\
Thus, depends on what you're claiming!\\
Ex: ``Giving teachers incentives improves educational outcomes''\\
vs. ``Giving teachers incentives improves GPAs''\\
\br
If GPA, SAT, and ACT are 3 different hypotheses, then don't correct
\br
Discussion: I write a paper with 20 unrelated hypotheses. MHT??
\end{frame}

%----------------------------------------------------------------------------------------

\begin{frame}
\frametitle{When to Use Corrections? (Continued)}
Fancy language for this idea: Rubin (2021)
\br
\begin{itemize}
    \item Disjunction testing: ``win'' if you reject at least once
    \begin{itemize}
        \item $H_0$: both green and red jelly beans \textit{do not} cause acne
        \item $H_1$: either green or red jelly beans (or both) cause acne
        \item Rule: reject $H_0$ if either is significant (correction needed!)
    \end{itemize}
    \item Conjunction testing: ``win'' only if you reject all
    \begin{itemize}
        \item $H_0$: either green or red jelly beans (or both) \textit{do not} cause acne
        \item $H_1$: both green and red jelly beans cause acne
        \item Rule: reject $H_0$ if bother are significant (no correction needed)
    \end{itemize}
    \item Individual testing: each test has its own ``win''
    \begin{itemize}
        \item $H_0^1$: green jelly beans do not cause acne
        \item $H_1^1$: green jelly beans cause acne
        \item $H_0^2$: red jelly beans do not cause acne
        \item $H_1^2$: red jelly beans cause acne
        \item Rule: two separate tests. No ex-ante ``either/both'' claim
    \end{itemize}
\end{itemize}
\end{frame}

%----------------------------------------------------------------------------------------

\begin{frame}
\frametitle{Conclusion}
\begin{itemize}
    \item Mathematically, there are ways to correct for MHT
    \item Degree of correction should depend on correlation across tests
    \begin{itemize}
        \item Resampling methods (Wolf-Romano) are best here.
    \end{itemize}
    \item The hard question is: when to use them??
    \begin{itemize}
        \item Be precise in what exactly you're claiming!
        \item ``educational outcomes'' vs. ``GPAs''
    \end{itemize}
    \item Final thought: readers are decent Bayesians. They don't just blindly trust 0.05. Tie your hands with preregistration and report everything. They'll update appropriately.
\end{itemize}
\end{frame}

%----------------------------------------------------------------------------------------


\end{document}