\documentclass[10pt,t,english]{beamer}
\usepackage{fontawesome}
\usepackage{graphicx}
\usepackage{array}
\usepackage[normalem]{ulem}
\usepackage{amsfonts,amsmath,amssymb,bm,bbm}
\usepackage{mathrsfs}
\usepackage{sgame}
\usepackage{graphicx,pstricks}
\usepackage{xcolor}
\usepackage{colortbl}
\usepackage{makecell}
\usepackage{tikz,tikzsymbols,gnuplottex}
\usetikzlibrary{decorations.pathreplacing,shapes}
\usepackage[english]{babel}
\usepackage[utf8]{inputenc}
\usepackage{appendixnumberbeamer}
\usepackage{datetime2}
\usepackage{booktabs}
\usepackage{setspace}
\usepackage{rotating}
\usepackage{natbib}
\usepackage{listings}

%\usepackage{matlab-prettifier} % For enhanced MATLAB highlighting
\lstset{
    language=matlab,
    % Other options for styling (optional)
    basicstyle=\ttfamily\footnotesize, % Font style
    keywordstyle=\color{blue}, % Keyword color
    commentstyle=\color{green!50!black}, % Comment color
    stringstyle=\color{red!70!black}, % String color
    numbers=left, % Line numbers on the left
    numberstyle=\tiny\color{gray}, % Style for line numbers
    frame=single, % Frame around the listing
    breaklines=true, % Allow line breaking
    captionpos=b, % Caption at the bottom
    tabsize=4 % Tab size
}

% ------------------------------------------------------------------------------
% Use the beautiful metropolis beamer template
% ------------------------------------------------------------------------------
\usepackage[T1]{fontenc}
\usepackage[utf8]{inputenc}
\usepackage{fontawesome}
\usepackage{FiraSans} 
\mode<presentation>
{
  \usetheme[progressbar=foot,background=light]{metropolis} 
  \usecolortheme{default} % or try albatross, beaver, crane, ...
  \usefonttheme{default}  % or try default, serif, structurebold, ...
  \setbeamertemplate{navigation symbols}{}
  \setbeamertemplate{caption}[numbered]
  %\setbeamertemplate{frame footer}{My custom footer}
} 

\newenvironment{stepenumerate}{\begin{enumerate}[<+->]}{\end{enumerate}}
\newenvironment{stepitemize}{\begin{itemize}[<+->]}{\end{itemize} }
\newenvironment{stepenumeratewithalert}{\begin{enumerate}[<+-| alert@+>]}{\end{enumerate}}
\newenvironment{stepitemizewithalert}{\begin{itemize}[<+-| alert@+>]}{\end{itemize} }

\newtheorem{question}{Question}
\newtheorem{claim}{Claim}
\newtheorem{proposition}{Proposition}
\newtheorem{remark}{Remark}
\newtheorem{conjecture}{Conjecture}

\definecolor{metrop}{RGB}{29, 44, 44}
\colorlet{GrayLight}{black!15}
\colorlet{GrayMedium}{black!30}
\colorlet{ForestGreen}{green!60!black}

\newenvironment{transitionframe}{
  \setbeamercolor{background canvas}{bg=black!80}
  \begin{frame}}{
    \end{frame}
}

\newcommand{\br}{

\bigskip

}

\newcommand{\pd}{\partial}
\newcommand{\RR}{\mathbb{R}}

\newcommand*\hugme[1]{\tikz[baseline=(char.base)]{\node[shape=ellipse,draw,inner sep=0pt] (char) {#1};}}

\newcounter{saveenumi}
\newcommand{\seti}{\setcounter{saveenumi}{\value{enumi}}}
\newcommand{\conti}{\setcounter{enumi}{\value{saveenumi}}}
\resetcounteronoverlays{saveenumi}



\newcommand\dotprod[2]{\langle #1 , #2 \rangle}
\newcommand{\ft}[1]{\widehat #1}
\newcommand{\qabove}[1]{\overset{\text{\large \textbf ?}}{#1}}
\newcommand{\eqae}{\overset{\text{a.e.}}{=}}
\newcommand{\calp}{\mathcal{P}}
\newcommand{\calg}{\mathcal{G}}
\newcommand{\calb}{\mathcal{B}}
\newcommand{\textd}{\text{d}}
\newcommand{\bbr}{\mathbb{R}}
\newcommand{\binm}{\mathbin{M}}
\newcommand{\binc}{\mathbin{C}}
\newcommand{\binb}{\mathbin{B}}
\newcommand{\calc}{\mathcal{C}}
\newcommand{\calh}{\mathcal{H}}
\newcommand{\bfone}{\mathbf{1}}
\newcommand{\bbe}{\mathbb{E}}
\newcommand{\bfle}{\mathbf{e}}
\newcommand{\calf}{\mathcal{F}}
\newcommand{\cala}{\mathcal{A}}
\newcommand{\cale}{\mathcal{E}}
\newcommand{\bbn}{\mathbb{N}}
\newcommand{\cantor}{\calc}
\newcommand{\calY}{\mathcal{Y}}
\newcommand{\textb}{\text{B}}
\newcommand{\calm}{\mathcal{M}}
\newcommand{\bint}{\mathbin{T}}
\newcommand{\ep}{\epsilon}
\newcommand{\bbq}{\mathbb{Q}}
\newcommand{\bbp}{\mathbb{P}}
\newcommand{\cals}{\mathcal{S}}
\newcommand{\emptysequence}{e}
\newcommand{\bbz}{\mathbb{Z}}
\newcommand{\fraka}{\frak{A}}
\newcommand{\frakb}{\frak{B}}
\newcommand{\length}{\text{length}}
\newcommand{\bfn}{\mathbf{N}}
\newcommand{\support}{\text{support}}
\DeclareMathOperator*{\argmax}{arg\,max}
\newcommand{\dom}{\mbox{dom}}
\def\ut{\underline t}
\def\um{\underline m}
\def\PP{\mathbb{P}}
\def\EE{\mathbb{E}}
\def\RR{\mathbb{R}}

% Title page info
\title[Incentives in Experiments: Theory]{ExpEcon Methods:\\A Decision Theory Primer}
\author[ECON 8877]{ECON 8877\\P.J. Healy} \color{metrop}
\institute[OSU]{}
\date[]{\vfill {\tiny Updated \today\ at\ \DTMcurrenttime}}

\begin{document}

\frame{\maketitle}

\begin{frame}{Differing Frameworks}
\begin{itemize}
    \item The theory of incentives is built on classic decision theory frameworks
    \item Each models uncertainty in a different way
    \item Difference: which probabilities are ``common knowledge''?
    \item Each has different axioms
    \begin{itemize}
        \item So, sufficient conditions for an IC experiment will differ
    \end{itemize}
    \item Historical confusion among experimentalists due to using different/unclear frameworks
\end{itemize}
\end{frame}

\begin{frame}{An Example}
\begin{center}
    \includegraphics[width=2.5in]{LectureSlides/graphics/ach/trees_marblefirst.png}\\

    \only<1>{How to model this gamble?}
    \only<2>{von Neumann-Morgenstern (vNM): All probabilities are known\\
    Objective Lottery: $L=(\$1,0.5;\$2,0.25;\$3,0.25)$}
    \only<3>{Savage: No probabilities are known\\
    State space: $\Omega=\{RH,RT,BH,BT\}$. Outcomes: $X=\{\$1,\$2,\$3\}$\\
    Act: $f(RH)=\$1$, $f(RT)=\$2$, $f(BH)=\$1$, $f(BT)=\$3$}
    \only<4>{Anscombe-Aumann (AA): Probabilities known \textit{only} in the 2nd stage\\
    State space: $\Omega=\{R,B\}$. Outcomes: $X=\{\$1,\$2,\$3\}$\\
    AA-act: $f(R)=(\$1,0.5;\$2,0.5)$, $f(B)=(\$1,0.5;\$3,0.5)$}
\end{center}

    
\end{frame}


\begin{transitionframe}[c]
  \begin{center}{ \Huge \textcolor{white}{The vNM Framework:\\Objective Lotteries}}\end{center}
\end{transitionframe}

\begin{frame}{vNM (1944)}
\begin{itemize}
    \item Outcomes: $x\in X = \{x_1,\ldots,x_n\}$
    \item Simple lotteries: $p\in L_1=\Delta(X)$, $p=(x_1,p_1;\ldots;x_n,p_n)$
    \item Mixture operation: 
    $$\alpha p + (1-\alpha) q = (x_1,\alpha p_1 + (1-\alpha) q_1;\ldots;x_n,\alpha p_1 + (1-\alpha)q_1) \in L_1 $$
    \item $\succeq$ over $L_1$ (complete, transitive, \& continuous, so $\exists \ U(p)$)
    \item vNM's Mixture Independence Axiom (IND):
    $$
     p\succeq q \iff \alpha p + (1-\alpha) r \succeq \alpha q + (1-\alpha) r
    $$
\end{itemize}
\br
vNM EU Theorem: $\succeq$ satisfies Mixture Independnece if and only if
$$
   \exists u(\cdot) : U(p) = \sum_x p(x)u(x)
$$
We just need to learn your risk aversion ($u(x)$ ``utility index'')
\end{frame}

\begin{frame}{Segal (1990): Compound Lotteries}
\begin{itemize}
    \item Two-stage lottery: $P=(q^1,P_1;\ldots;q^m;P_m)\in L_2$
    \item $\succeq$ now over $L_2$
    \item $\delta^1_q = (q,1)\in L_2$ and $\delta^2_q = ((x_1,1),q_1;\ldots;(x_n,1),q_n)\in L_2$
    \item Time Neutrality Axiom: $\delta^1_q \sim \delta^2_q$
    \item ``$p\succeq q$'' means $\delta^1_p \succeq \delta^1_q$
    \item ``$x\succeq y$'' means $\delta^1_{(x,1)}\succeq \delta^1_{(y,1)}$
    \item Reduced Lottery: given $P=(q^1,P_1;\ldots;q^m;P_m)$, let
    $$
        r(P) = \left( x_1,\sum_{j=1}^m P_jq^j_1;\ldots;x_n,\sum_{j=1}^m P_jq^j_n \right) \in L_1
    $$
    \item ROCL Axiom: $P \sim r(P)$\ \ \  (technically, $P\sim \delta^1_{r(P)}$)
\end{itemize}
\end{frame}

\begin{frame}{Segal (1990): Compound Lotteries}
\begin{itemize}
    \item Recall Mixture operation for simple lotteries:
    $$
        \alpha p + (1-\alpha) q = (x_1,\alpha p_1 + (1-\alpha) q_1;\ldots;x_n,\alpha p_1 + (1-\alpha)q_1) \in L_1 
    $$
    Mixture is ``in between'' $p$ and $q$, in the same space ($L_1$)
    \item 1st-Stage Mixture operation: 
    $$
        \alpha P + (1-\alpha) Q = (q^1,\alpha P_1 + (1-\alpha) Q_1;\ldots;q^n,\alpha P_1 + (1-\alpha)Q_1) \in L_2
    $$
    \item Compound operation:
    $$
        \alpha p \oplus (1-\alpha) q = (p,\alpha;q,1-\alpha)\in L_2
    $$
    \item Compound Independence Axiom (1st attempt):
    $$
        p \succeq q \iff \alpha p\oplus (1-\alpha) r \succeq \alpha q\oplus (1-\alpha) r
    $$
    \item Problem: only applies to 2-element compound lotteries
\end{itemize}
\end{frame}

\begin{frame}{Segal (1990): Compound Loteries}
\begin{itemize}
    \item Replacement operation: Fix $P=(q^1,P_1;\ldots;q_n,P_n)$. Define 
    $$
    [P|p,i] = (q^1,P_1;\ldots;p,P_i;\ldots;q_n,P_n)
    $$
    ``Replace the $i$th branch with $p$''
    \item Compound Independence Axiom (fully general):
    $$
        p\succeq q \iff [P|p,i] \succeq [P|q,i]
    $$ 
\end{itemize}
\br
How does this apply to experiments?
\end{frame}

\begin{frame}{Experiments as Compound Lotteries}
Imagine two decisions, each a choice between two lotteries
\begin{enumerate}
    \item $D_1=\{p^1,q^1\}$
    \item $D_2=\{p^2,q^2\}$
\end{enumerate}
Coin flip determines whether $D_1$ or $D_2$ is paid.\\
True preference: $p^1\succ q^1$ and $p^2\succ q^2$
\begin{itemize}
    \item Would the subject want to lie in $D_1$?
    \item Fix choice of $p^2$ in $D_2$. Compare $p^1$ vs. $q^1$
    \item Announce $p^1$: get $\frac{1}{2}p^1 \oplus \frac{1}{2}p^2$
    \item Announce $q^1$: get $\frac{1}{2}q^1 \oplus \frac{1}{2}p^2$
\end{itemize}
Incentive compatibility: If $p^1\succ q^1$ then
$$
    \frac{1}{2}p^1 \oplus \frac{1}{2}p^2 \succ \frac{1}{2}q^1 \oplus \frac{1}{2}p^2.
$$
\br
That's exactly Compound Independence!!    
\end{frame}

\begin{frame}{Segal (1990): Compound Lotteries}
What would EU for two-stage lotteries look like?
$$
    U(P) = \sum_i P_i \left(\sum_x q^i(x) u(x)\right)
$$
 \br
 EU with ROCL?
 \br
 What axioms give EU for two-stage lotteries? EU with reduction
\end{frame}

\begin{frame}{Segal (1990): Compound Lotteries}
When do we get EU with reduction?
\begin{enumerate}
    \item Mixture independence $\Rightarrow$ EU on simple lotteries
    \begin{itemize}
        \item But depends on which timing you use: $\delta^1$ vs $\delta^2$
    \end{itemize}
    \item MixIND + Time Neutrality $\Rightarrow$ 2nd stage EU regardless of $\delta^1$ or $\delta^2$
    \begin{itemize}
        \item But might not have EU in 1st stage
    \end{itemize}
    \item MixIND + Time Neutrality + CompIND $\Rightarrow$ EU w/ Reduction
    \begin{itemize}
        \item Compound Independence ``connects'' the two stages
    \end{itemize}
\end{enumerate}
\br
What's the role of ROCL?
\begin{enumerate}
    \item MixIND + TimeNeut + CompIND $\Rightarrow$ ROCL (see above)
    \item ROCL connects the two IND axioms
    \begin{enumerate}
        \item ROCL + MixIND $\Rightarrow$ CompIND
        \item ROCL + CompIND $\Rightarrow$ MixIND\\
        So you can replace either with ROCL and still get EU w/ Reduction
    \end{enumerate}
\end{enumerate}
\end{frame}

\begin{frame}{Allais Paradox}
    Why we might not want to assume MixIND (thus, EU):
    \begin{itemize}
        \item \textbf{Option 1A}: 100\% chance of \$1M
        \item Option 1B: 10\% \$5M, 89\% \$1, 1\% \$0
    \end{itemize}
    and
    \begin{itemize}
        \item Option 2A: 11\% \$1M, 89\% \$0
        \item \textbf{Option 2B}: 10\% \$5M, 90\% \$0
    \end{itemize}
    \begin{center}
        \includegraphics[width=3in]{LectureSlides/graphics/ach/AllaisCurves.png}\\
        Need ``fanning out'' indifference curves (could be nonlinear)
    \end{center}
\end{frame}

\begin{frame}{Stochastic Dominance}
What if we don't want to assume EU? Minimal: $\succeq$ respects dominance
\begin{itemize}
    \item Second-Stage (``subjective'') Stochastic Dominance: 
    $$
        p\sqsupset_2 q \iff \forall x\in X\ \ \sum_{y:y\succeq x} p(y) \geq \sum_{y:y\succeq x} q(y)\ \  \mathrm{(one\ strict)}
    $$
    \item First-Stage Stochastic Dominance:
    $$
        P\sqsupset_1 Q \iff \forall p\in supp(P,Q)\ \  \sum_{q:q\succeq p} P(q) \geq \sum_{q:q\succeq p} Q(q)\ \  \mathrm{(one\ strict)}
    $$
    \item 2nd-Stage Monotonicity: $p \sqsupseteq_2 q \Rightarrow p \succeq q$ and $p \sqsupset_2 q \Rightarrow p \succ q$
    \item 1st-Stage Monotonicity: $P \sqsupseteq_1 Q \Rightarrow P\succeq Q$ and $P \sqsupset_1 Q \Rightarrow P\succ Q$
    \item CompIND+TimeNeut connects the two monotonicity axioms
    \begin{itemize}
        \item (CompIND+TimeNeut) + 2nd-Stage MONO $\Rightarrow$ 1st-Stage MONO
        \item (CompIND+TimeNeut) + 1st-Stage MONO $\Rightarrow$ 2nd-Stage MONO
        \item Can replace (CompIND+TimeNeut) with ROCL
    \end{itemize}
\end{itemize}
\end{frame}

\begin{frame}{Back to Experiments}
If we only have dominance axioms, can we say anything about experiments?
\br
Recall: CompoundIND $\Rightarrow$ tell the truth
\br
\textbf{Lemma:} 1st-Stage Monotonicity $\Rightarrow$ CompoundIND ($\Rightarrow$ tell the truth)
\br
Proof
\begin{itemize}
    \item Suppose $p\succeq q$
    \item Then $[P|p,i] \sqsupseteq_1 [P|q,i]$ (they only differ on the $i$th branch)
    \item By 1st-Stage MONO, $[P|p,i]\succeq [P|q,i]$
\end{itemize}
\end{frame}

\begin{frame}{Back to Experiments}
So any non-EU theory satisfying CompIND (or 1st-Stage MONO) is fine...
\br
But recall:\\
\begin{center}
    CompIND + ROCL $\Rightarrow$ MixIND $\Rightarrow$ EU (for either $\delta^1$ or $\delta^2$)
\end{center}
or, the contrapositive:
\begin{center}
    non-EU theory $\Rightarrow$ $\neg$ MixIND $\Rightarrow$ ($\neg$ CompIND) or ($\neg$ ROCL)
\end{center}
\br
So, if you have a non-EU theory, you either have:
\begin{enumerate}
    \item $\neg$ CompIND, so IC will fail (for some experiments), or
    \item $\neg$ ROCL, in which case you can still have CompIND \& IC
\end{enumerate}
So we better hope ROCL fails!! (Hint: it does. More later...)
\end{frame}


\begin{frame}{Non-EU Theories}
What non-EU theories are there in this framework?
\begin{enumerate}
    \item Probability Weighting (from original Prospect Theory 1977)
    \begin{itemize}
        \item $p=(x_1,p_1;\ldots;x_n,p_n)$
        $$
            U(p) = \sum_{i=1}^n u(x_i) w(p_i)
        $$
        \item Weighting function $w(\cdot)$ is concave then convex.
        \item Problem! This can violate (2nd-Stage) MONO
        \item Solution: Editing phase. People never pick dominated lotteries
    \end{itemize}
    \seti
\end{enumerate}
\end{frame}

\begin{frame}{Non-EU Theories}
\begin{enumerate}
    \conti
    \item Rank-Dependent Utility
    \begin{itemize}
        \item $p=(x_1,p_1;\ldots;x_n,p_n)$ where $x_1<x_2<\cdot <x_n$
        $$
            U(p) = \sum_{i=1}^n u(x_i) \underbrace{\left[ w(\sum_{j=1}^i p_j) - w(\sum_{j=1}^{i-1} p_j) \right]}_{\textrm{weighted ``probability'' of $x_i$}}
        $$
        \item Weighting function $w(\cdot)$ is concave then convex. $p^*\approx 0.4$?
        \item Weights on \textit{cumulative} probability avoids MONO violations
        \item Quiggen (1982), Prelec (1993) weighting function, many others
    \end{itemize}
    \seti
\end{enumerate}
\begin{center}
    \includegraphics[width=1in]{LectureSlides/graphics/ach/RDUCurves.png}
\end{center}
\end{frame}

\begin{frame}{Non-EU Theories}
\begin{enumerate}
        \conti
    \item Cautious Expected Utility (Cerreia-Vioglio et al. 2015)
    \begin{itemize}
        \item Certainty equivalent of $p$ solves $u(c_p^u)=\sum_x p(x)u(x)$, or $c_p^u = u^{-1}(\sum_x p(x)u(x))$
        \item Agent has a set of utility indices $\mathcal{V}$ and is ``pessimistic''
        $$
            U(p) = \inf_{u\in\mathcal{V}} u^{-1}(\sum_x p(x)u(x)) = \inf_{u\in\mathcal{V}} c_p^u
        $$
        \item NCI (weakening of MixIND): $p\succeq \delta_x \Rightarrow \alpha p + (1-\alpha) r \succeq \alpha \delta_x + (1-\alpha) r$
    \end{itemize}
    \seti
\end{enumerate}
\begin{center}
    \includegraphics[width=1.5in]{LectureSlides/graphics/ach/CEU.png}
\end{center}
\end{frame}

\begin{frame}{Other Non-EU Theories}
\begin{enumerate}
    \conti
    \item Cumulative PT (K\&T 1992): RDU with loss aversion
    \item Weighted EU (Chew \& McCrimmon 1979)
    \item Decision Weighted Utility (Handa 1977), but violates MONO
\end{enumerate}
\br
Today: RDU and CPT are the most popular
\end{frame}


\begin{transitionframe}[c]
  \begin{center}{ \Huge \textcolor{white}{The Savage Framework:\\Entirely Subjective Beliefs}}\end{center}
\end{transitionframe}


\begin{frame}{Jimmy Savage}
\includegraphics[height=0.4in]{LectureSlides/graphics/ach/SavageMug.png} Leonard ``Jimmy'' Savage
\begin{enumerate}
    \item Genius from Detroit (like Eminiem??)
    \item Wayne State $\rightarrow$ Michigan BS \& PhD in math (1941)
    \item IAS Princeton, then Chicago. 
    \begin{itemize}
        \item Milton Friedman \& W. Allen Wallis were mentors
    \end{itemize}
    \item WWII: assistant to John von Neumann
    \item \textit{The Foundation of Statistics} (1954)
    \begin{itemize}
        \item Subjective expected utility without objective lotteries
    \end{itemize}
\end{enumerate}
\end{frame}

\begin{frame}{Savage (1954)}
\begin{itemize}
    \item States: $\omega\in \Omega$ (need $\Omega$ to be infinite)
    \item Events: $E\subseteq \Omega$
    \item Outcomes: $x\in X$ (prizes, consequences...)
    \item Acts: $f:\Omega \rightarrow X$.\ \ \ $f\in \mathcal{F} = X^\Omega$
    \item $\succeq$ over $\mathcal{F}$
    \item Notation: $xEy$ is binary act where $f(E)=x$ and $f(E^c)=y$
    \item More general: $fEg$ = $\{f(\omega)$ if $\omega\in E$, $g(\omega)$ if $\omega\not\in E\}$
\end{itemize}
Savage's omelette example: crack next egg into separate bowl?
\begin{itemize}
    \item $\Omega = \{$good egg,rotten egg$\}$
    \item $f$ = crack into same bowl. $g$ = crack into separate bowl
    \item $f(good)$ = omelette, wash 1 bowl. $f(bad)$ = no omelette, wash 1
    \item $g(good)$ = omelette, wash 2 bowls. $g(bad)$ = omelette, wash 2
    \item $f\succeq g$ or $g\succeq f$?
\end{itemize}
\end{frame}

\begin{frame}{Savage (1954)}
What would EU look like??
\begin{itemize}
    \item vNM EU: assume linearity, just need to learn $u(x)$
    \item Savage: assume more, but need to learn $u(x)$ \textbf{and} $p(\omega)$
\end{itemize}
\br
The goal:
\begin{align*}
    U(f) &= \sum_{x\in X} p(\underbrace{\{\omega: f(\omega)=x\}}_{\text{Event ``}x\text{ is paid''}})\ u(x)\\
         &= \sum_{x\in X} p(\, f^{-1}(x)\, )\ u(x)
\end{align*}
The building blocks: Savage's 6 Postulates (P1--P6) (ignore P7 here)\\
\textbf{P1:} $\succeq$ is complete, reflexive, and transitive (``ordering'')\\
\textbf{P5:} There are $x,y\in X$ s.t. $x\succ y$ (``non-degeneracy'')
\br
The hard part: How to learn $p(\omega)$ from $\succeq$??
\end{frame}

\begin{frame}{Ramsey (1926)}
\includegraphics[height=0.6in]{LectureSlides/graphics/ach/RamseyMug.png} Frank Plumpton Ramsey
\begin{itemize}
    \item Born into academic privilege, Cambridge, England
    \item Easy-going, simple, modest, loved swimming
    \item Translated Wittgenstein, went to Austria, became his friend
    \item Math undergrad at Cambridge, advisor was Keynes. No PhD
    \item Keynes (1921) \textit{A Treatise on Probability}: prob. must be objective
    \item Ramsey (1926) ``Truth and Probability'': probability is subjective
    \item Your beliefs are \textit{defined by} the bets you'd make. The odds.
    \item Problem: confounded with risk aversion! Assume risk neutrality?
    \item De Finetti (1937) independently developed same ideas
    \item Liver problems, surgery, died at age 26. Infection from river?
\end{itemize}
\end{frame}

\begin{frame}{Savage (1954)}
How can we learn your beliefs?
\begin{itemize}
    \item Would you rather bet on event $E$ or $F$? Pick $1E0$ or $1F0$?
    \item Leads to a ranking of all possible events: $E \trianglerighteq F$
    \item Beliefs work like utility! $E \trianglerighteq F \Rightarrow p(E)\geq p(F)$.
    \item $p$ is ``qualitative probability''. Ordinal, unless we add structure
    \item Can $p(E)=0$? $E$ is ``null'' if $fEh\sim gEh$ regardless of $f,g,h$
\end{itemize}
\begin{enumerate}
    \item The stakes of the bet shouldn't matter:\\
    \textbf{P4:} If $x'\succ x$ and $y'\succ y$ then $(x'Ex \succeq x'Fx)\iff (y'Ey \succeq y'Fy)$
    \item ``Small'' events exist:\\
    \textbf{P6:} For any $f\succ g$ and $x$ I can find ``small enough'' events $A_1,A_2$ such that $f\succ xA_1 g$ and $xA_2 f\succ g$\\
    (Substituting in $x$ doesn't change the ordering of $f$ and $g$)
    \item Eventwise Monotonicity (similar to Compound Independence):\\
    \textbf{P3:} $x\succeq y \iff xEg \succeq yEg$ (if $E$ is not null)
\end{enumerate}
\end{frame}

\begin{frame}{Savage (1954)}
We're almost there!!
\begin{enumerate}
    \item \textbf{P1: Ordering}
    \item \textbf{P2: ???}
    \item \textbf{P3: Eventwise Monotonicity} (compound independence)
    \item \textbf{P4: Weak Comparative Probability} (stakes don't matter)
    \item \textbf{P5: Nondegeneracy}
    \item \textbf{P6: Small Event Continuity}
\end{enumerate}
\br
The missing piece (P2) helps gives us linearity for EU
\br
{\footnotesize Hartmann (2020): P3 is implied by the others}
\end{frame}

\begin{frame}{Savage (1954)}
\textbf{P2: The Sure-Thing Principle}
$$
    f'Eg \succeq fEg \Rightarrow f'Eh \succeq fEh
$$
``I can rank $f'$ vs $f$ conditional on $E$,\\
and what's paid off of $E$ won't matter.''
\br
Has a flavor of Mixture Independence from vNM
\br
\textbf{Savage's Subjective Expected Utility (SEU) Theorem:}
$$
    \succeq \text{satisfies P1--P6} \Rightarrow \exists u,p :\ U(f) = \sum_x p(\{\omega:f(\omega)=x\})u(x)
$$
\end{frame}


\begin{frame}{Which Axiom Do We Need for IC Experiments?}
Two decisions, choice objects are abstract (acts, lotteries, \$, ...)
\begin{enumerate}
    \item $D_1=\{a^1,b^1\}$
    \item $D_2=\{a^2,b^2\}$
\end{enumerate}
Payment act: $\omega_1 \mapsto D_1$ is paid, and $\omega_2\mapsto D_2$ is paid\\
True preference: $a^1\succ b^1$ and $a^2\succ b^2$
\begin{itemize}
    \item Would the subject want to lie in $D_1$?
    \item Fix choice of $a^2$ in $D_2$. Compare $a^1$ vs. $b^1$. Let $E=\{\omega_1\}$
    \item Announce $a^1$: get $a^1 E a^2$
    \item Announce $b^1$: get $b^1 E a^2$
\end{itemize}
Incentive compatibility: If $a^1\succ b^1$ then
$$
    a^1 E a^2 \succ b^1 E a^2
$$
\br
That's \textbf{P3} (eventwise monotonicity)!!\ \ $x\succeq y\Rightarrow xEg\succeq yEg$
\end{frame}

\begin{frame}{Finite States?}
Could we have a finite number of states?
\begin{itemize}
    \item Representation would not be unique
    \begin{itemize}
        \item Small changes to $p$ and $u$ wouldn't alter $\succeq$
    \end{itemize}
    \item Kraft, Pratt, \& Seidenberg (1959)
    \begin{itemize}
        \item Can construct $\trianglerighteq$ that is not represented by any probability measure $p$!
        \item Requires $n=5$, not very intuitive (to me)
    \end{itemize}
    \item Gul (1992) gets SEU for finite $\Omega$
    \begin{itemize}
        \item Requires $\exists E$ where $E\sim_\trianglerighteq E^C$
        \item Other, stronger axioms
    \end{itemize}
    \item Others have, too (Wakker, Davidson \& Suppes, Nakamura...)
\end{itemize}
    
\end{frame}

\begin{frame}{The Paradoxes}
\begin{itemize}
    \item OK, we care about P3 (eventwise montonicity)
    \item P3 is part of EU
    \item But we know EU is violated!
    \item Allais paradox: indifference curves aren't parallel
    \item Even worse: Ellsberg paradox
    \begin{itemize}
        \item People don't even \textit{have} well-defined probabilities!
    \end{itemize}
\end{itemize}
\br
Which axioms do these paradoxes violate??
\br
Is P3 (eventwise monotonicity) okay???
    
\end{frame}

\begin{frame}{Ellsberg (1961)}
\includegraphics[width=0.4in]{LectureSlides/graphics/ach/EllsbergMug.png} Daniel Ellsberg
\begin{itemize}
    \item Detroit kid, Marine, RAND Corp
    \item Harvard PhD student, wrote the Ellsberg paradox
    \item Worked at the Pentagon under McNamara, went to Vietnam
    \item Left Pentagon for RAND. Contributed to the ``Pentagon Papers'', a complete analysis of the conduct of the US military in Vietnam
    \item Became sympathetic to war resisters
    \item Leaked the Pentagon Papers to the NY Times
    \item Politically embarrassing to Kennedy, Johnson, Nixon
    \item Nixon's ``White House Plumbers'' and Watergate ensued
    \item Tried for espionage, acquitted
    \item Died June 16, 2023
\end{itemize}
\end{frame}

\begin{frame}{Ellsberg (1961)}
Urn: 30 red + (60 black or yellow)
\begin{center}
    \begin{tabular}{r|ccc}
         & $\overbrace{\phantom{red}}^{\text{30 balls}}$ & \multicolumn{2}{c}{$\overbrace{\phantom{black\ yellow}}^{\text{60 balls}}$} \\
         & red & black & yellow \\
         \hline
        $f_1$ (Bet R) & \$100 & \$0 & \$0 \\
        $f_2$ (Bet B) & \$0 & \$100  & \$0 \\
        \hline
        $g_1$ (Bet RY) & \$100 & \$0 & \$100 \\
        $g_2$ (Bet BY) & \$0 & \$100  & \$100 \\
    \end{tabular}
\end{center}
\br
\begin{itemize}
    \item People avoid ambiguity: $f_1\succ f_2$ but $g_2\succ g_1$
    \item Clearly violates P2 Sure Thing Principle
    \item Also means you can't have a probability!
    \begin{itemize}
        \item $f_1\succ f_2 \Rightarrow R\triangleright B$
        \item $g_2\succ g_1 \Rightarrow BY\triangleright RY \Rightarrow B\triangleright R$
    \end{itemize}
    \item Not a test of P3. Phew!
    \item What theories can accommodate this ``ambiguity aversion''?
\end{itemize}
\end{frame}

\begin{frame}{Frank Knight}
\includegraphics[height=0.4in]{LectureSlides/graphics/ach/KnightMug.png} Frank Knight
\begin{itemize}
    \item Another Chicago school patriarch
    \item Born 1885. Tennessee undergrad, Cornell PhD 1916
    \item Advisor of Milton Friedman, George Stigler
    \item Praised by Coase, Hayek, Samuelson.
    \item Known for \textit{Risk, Uncertainty, and Profit} (1921)
    \begin{itemize}
        \item Based on his PhD work at Cornell
    \end{itemize}
    \item Distinguished ``risk'' and ``uncertainty'' (ambiguity)
    \item Agrued that they differ in a ``deep'' way
    \item Claimed that uncertainty can lead to positive profits in competitive industries
    \item Today: Ambiguity = ``Knightian uncertainty''
\end{itemize}

\end{frame}

\begin{frame}{David Schmeidler}
\includegraphics[height=0.4in]{LectureSlides/graphics/ach/SchmeidlerMug.png} David Schmeidler
\begin{itemize}
    \item Born in Poland, 1939. Family evaded WWII in Russia
    \item Studied math at Hebrew U., PhD under Robert Aumann
    \item Full-time at Tel Aviv U. since 1971, part-time at OSU since 1987
    \item Father of ``post-Savage'' decision theory, incorporating ambiguity/Knightian uncertainty
    \item Died March 17, 2022
\end{itemize}
\br
Machina \& Schmeidler (1992):\\
What does it mean to ``have a probability'' if you're not EU??\\
(Note: we're forgetting Ellsberg's paradox for now...)
\end{frame}

\begin{frame}{Machina \& Schmeidler (1992)}
What we want:
\begin{enumerate}
    \item You have $\trianglerighteq$ over events that leads to $p(\omega)$ (P1,P4,P5,P6)
    \item Acts ``become'' lotteries via $p(\omega)$
    \item You respect FOSD over those lotteries (P3)
\end{enumerate}
\br
Can we just drop P2?? Not quite!\\
P4 is what really gives $\trianglerighteq$, but only for two outcomes ($E$ and $E^c$)\\
Need to strengthen P4 to deal with more outcomes 
\br
Let $xEyFg$ be ``$x$ on $E$, $y$ on $F$, and $g$ otherwise''
\br
\textbf{P4*:} ($\forall E,F$ disjoint)($\forall x'\succ x, y'\succ y$) ($\forall g,h$)
$$
    x'ExFg \succeq xEx'Fg \Rightarrow y'EyFh\succeq yEy'Fh
$$
``$E\trianglerighteq F$ regardless of stakes \textit{or} what's paid outside of $E\cup F$''
\end{frame}

\begin{frame}{Machina \& Schmeidler (1992)}
Let $r(f,p)$ be the lottery generated by $f$ using beliefs $p$
\br
\textbf{Theorem:}
If $\succeq$ satisfies P1, \phantom{P2}, P3, P4*, P5, P6 then there exists
\begin{enumerate}
    \item a subjective probability measure $p(\omega)$ on $\Omega$
    \item a function $V(p)$ over lotteries that respects FOSD
\end{enumerate}
such that
$$
    U(f) = V(r(f,p))
$$
\br
They say $\succeq$ is \textit{probabilistically sophisticated}
\br
Ellsberg is a test of probabilistic sophistication?? Which axiom?
\end{frame}

\begin{frame}{Allais vs. Ellsberg}
\begin{tabular}{c|c}
    \textbf{ALLAIS (CC)} & \textbf{ELLSBERG} \\
    \begin{tabular}{r|cc|c}
         &  1\% & 10\%  & 89\% \\
         & \#1 & \#2--11 & \#12--100 \\
         \hline
        \checkmark $f_1$ & \$100 & \$100 & \$100 \\
        \phantom{\checkmark} $f_2$ & \$0 & \$500  & \$100 \\
        \hline
        \phantom{\checkmark} $g_1$ & \$100 & \$100 & \$0 \\
        \checkmark $g_2$ & \$0 & \$500  & \$0 \\
    \end{tabular} &
    \begin{tabular}{r|cc|c}
         &  30 & \multicolumn{2}{c}{-----60-----}\\
         & red & black & yellow \\
         \hline
        \checkmark $f_1$ & \$100 & \$0 & \$0 \\
        \phantom{\checkmark} $f_2$ & \$0 & \$100  & \$0 \\
        \hline
        \phantom{\checkmark} $g_1$ & \$100 & \$0 & \$100 \\
        \checkmark $g_2$ & \$0 & \$100  & \$100 \\
    \end{tabular} \\
     & \\
     \multicolumn{2}{c}{EU: \textbf{P2}: $f'Eg\succeq fEg \Rightarrow f'Eh \succeq fEh$} \\
    Violated! $E=\{1,2$--$11\}$
    & 
    Violated! $E=\{$red,black$\}$\\
     & \\
    
     \multicolumn{2}{c}{\textbf{P4}: $(x'Ex\succeq x'Fx)\iff (y'Ey\succeq y'Fy)$} \\
    Not tested. No $E$ vs. $F$
    & 
    Not tested!\\
     & \\
    
     \multicolumn{2}{c}{P.S.: \textbf{P4*}: $(x'ExFg\succeq xEx'Fg) \iff (y'EyFh\succeq yEy'Fh)$} \\
    Also not tested.
    & 
     Violated! $E=\{$red$\}$, $F=\{$black$\}$
\end{tabular} 
\end{frame}


\begin{frame}{Two Faces of Ellsberg}
\begin{tabular}{c|c}
    \textbf{ELLSBERG 2-COLOR, 2-URN} & \textbf{ELLSBERG 3-COLOR, 1-URN} \\
    \begin{tabular}{r|cc|cc}
         &  \multicolumn{4}{c}{Urn 1: 50R, 50B. Urn 2: ???} \\
         & $R_1B_2$ & $B_1R_2$ & $B_1B_2$ & $R_1R_2$  \\
         \hline
        \checkmark $R_1$ & \$100 & \$0 & \$0 & \$100 \\
        \phantom{\checkmark} $R_2$ & \$0  & \$100 & \$0 & \$100 \\
        \hline
        \phantom{\checkmark} $B_2$ & \$100 & \$0 & \$100 & \$0  \\
        \checkmark $B_1$ & \$0  & \$100 & \$100 & \$0 \\
    \end{tabular} &
    \begin{tabular}{r|cc|c}
         &  30 & \multicolumn{2}{c}{-----60-----}\\
         & red & black & yellow \\
         \hline
        \checkmark $f_1$ & \$100 & \$0 & \$0 \\
        \phantom{\checkmark} $f_2$ & \$0 & \$100  & \$0 \\
        \hline
        \phantom{\checkmark} $g_1$ & \$100 & \$0 & \$100 \\
        \checkmark $g_2$ & \$0 & \$100  & \$100 \\
    \end{tabular} \\
     & \\
     \multicolumn{2}{c}{EU: \textbf{P2}: $f'Eg\succeq fEg \Rightarrow f'Eh \succeq fEh$} \\
    Violated! $E=\{R_1B_2,B_1R_2\}$
    & 
    Violated! $E=\{$red,black$\}$\\
     & \\
    
     \multicolumn{2}{c}{\textbf{P4}: $(x'Ex\succeq x'Fx)\iff (y'Ey\succeq y'Fy)$} \\
    Not tested. No $E$ vs. $F$
    & 
    Not tested!\\
     & \\
    
     \multicolumn{2}{c}{P.S.: \textbf{P4*}: $(x'ExFg\succeq xEx'Fg) \iff (y'EyFh\succeq yEy'Fh)$} \\
    Violated! $E=\{R_1B_2\}$, $F=\{B_1R_2\}$
    & 
     Violated! $E=\{$red$\}$, $F=\{$black$\}$
\end{tabular} 
\end{frame}

\begin{frame}{Separating P4* From P2 via Changing the Stakes}
\begin{tabular}{c|c}
    \textbf{ELLSBERG 2-COLOR, 2-URN} & \textbf{ELLSBERG 3-COLOR, 1-URN} \\
    \begin{tabular}{r|cc|cc}
         &  \multicolumn{4}{c}{Urn 1: 50R, 50B. Urn 2: ???} \\
         & $R_1B_2$ & $B_1R_2$ & $B_1B_2$ & $R_1R_2$  \\
         \hline
        \checkmark $R_1$ & \$100 & \$0 & \$0 & \$100 \\
        \phantom{\checkmark} $R_2$ & \$0  & \$100 & \$0 & \$100 \\
        \hline
        \phantom{\checkmark} $B_2$ & \$500 & \$0 & \$500 & \$0  \\
        \checkmark $B_1$ & \$0  & \$500 & \$500 & \$0 \\
    \end{tabular} &
    \begin{tabular}{r|cc|c}
         &  30 & \multicolumn{2}{c}{-----60-----}\\
         & red & black & yellow \\
         \hline
        \checkmark $f_1$ & \$100 & \$0 & \$0 \\
        \phantom{\checkmark} $f_2$ & \$0 & \$100  & \$0 \\
        \hline
        \phantom{\checkmark} $g_1$ & \$500 & \$0 & \$500 \\
        \checkmark $g_2$ & \$0 & \$500  & \$500 \\
    \end{tabular} \\
     & \\
     \multicolumn{2}{c}{EU: \textbf{P2}: $f'Eg\succeq fEg \Rightarrow f'Eh \succeq fEh$} \\
    Not tested! $f'$ and $f$ change
    & 
    Not tested!\\
     & \\
    
     \multicolumn{2}{c}{\textbf{P4}: $(x'Ex\succeq x'Fx)\iff (y'Ey\succeq y'Fy)$} \\
    Not tested. No $E$ vs. $F$
    & 
    Not tested!\\
     & \\
    
     \multicolumn{2}{c}{P.S.: \textbf{P4*}: $(x'ExFg\succeq xEx'Fg) \iff (y'EyFh\succeq yEy'Fh)$} \\
    Violated! $E=\{R_1B_2\}$, $F=\{B_1R_2\}$
    & 
     Violated! $E=\{$red$\}$, $F=\{$black$\}$
\end{tabular} 
\end{frame}


\begin{frame}{Ambiguity Aversion Literature}
\begin{itemize}
    \item Want to explain ambiguity aversion! (``Knightian uncertainty'')
    \item So, models that violate probabilistic sophiciation.
    \item Specifically, violate P4*
    \item But, for experiments, we hope P3 is maintained!
\end{itemize}
\br
I'll review these... but they mostly use a simpler framework!
\br
A framework called...
    
\end{frame}


\begin{transitionframe}[c]
  \begin{center}{ \Huge \textcolor{white}{The\\Anscombe-Aumann\\Framework}}\end{center}
\end{transitionframe}

\begin{frame}{Aumann \& Anscombe}
\includegraphics[height=0.4in]{LectureSlides/graphics/ach/AumannMug.png} Robert Aumann
\begin{itemize}
    \item Born in Germany 1938, family fled to NYC before Kristallnacht
    \item City College of NY, math PhD at MIT 1955
    \item Knot theory. Loved puzzles. ``Absolutely useless.'' But, DNA...
    \item Learned game theory at Princeton, postdoc \& sabbatical
    \item Hebrew U. Jerusalem since 1956, Stony Brook visitor
    \item Hugely important papers in many areas
    \begin{itemize}
        \item Correlated equilibrium, common knowledge, division problems, continuum economies, epistemics, repeated games, cooperative game theory, integrals of correspondences...
    \end{itemize}
    \item Aumann $\rightarrow$ Ehud Lehrer $\rightarrow$ Yaron Azrieli
\end{itemize}
\includegraphics[height=0.3in]{LectureSlides/graphics/ach/AnscombeMug.png} Frank Anscombe
\begin{itemize}
    \item English statistician, 20yrs older than Aumann. Princeton
    \item Gave a lecture on Savage, which Aumann attended...
\end{itemize}
    
\end{frame}

\begin{frame}{Anscombe-Aumann (1963)}
    \begin{itemize}
        \item The Savage framework is..
        \begin{enumerate}
            \item great because it gives us subjective $p(\omega)$ from $\succeq$
            \item too intractible to work with!
        \end{enumerate}
        \item Insight: vNM theorem works for \textit{any} convex space
    \end{itemize}
Theorem: Let $K$ be a convex space, and $p,q\in K$ (not necessarily lotteries). Under vNM axioms (ordering, continuity, MixIND) $\exists U$:
$$
    U(\alpha p + (1-\alpha)r) = \alpha U(p) + (1-\alpha) U(r)\ \ \ \text{(}U\text{ is affine)}
$$
\br
Let $K$ be a space of ``acts that pay lotteries''\\
``Horse races (subjective) and roulette wheels (objective)''
\br
$\Omega = \{\omega_1,\ldots,\omega_n\}$, $f=(p^1,\ldots,p^n)$\\
$f\in \mathcal{F} = \Delta(X)^\Omega$ is a convex space!
\end{frame}

\begin{frame}{AA (1963)}
$$
U(\alpha f + (1-\alpha)g) = \alpha U(f) + (1-\alpha) U(g)
$$
How does this mixture operation work?
\begin{itemize}
    \item At each $\omega$, $f(\omega)$ and $g(\omega)$ are lotteries.
    \item $(\alpha f+ (1-\alpha)g)(\omega) = \alpha f(\omega) + (1-\alpha) g(\omega)$
    \item Mixture lottery state-by-state
\end{itemize}
What does affine $U$ imply? There exists \textit{state-dependent} $u(x|\omega)$\\
vNM's \textbf{A1--A3} applied to $\Delta(X)^\Omega$ $\Rightarrow$
$$
    U(f) = \sum_\omega \sum_x \underbrace{f(\omega)(x)}_{Pr(x)\text{ at }\omega} u(x|\omega)
$$
The ``weight'' on state $\omega$ is embedded in $u(x|\omega)$. No $p(\omega)$.
\br
Want state-independent utility index for EU, thus a separate $p(\omega)$
\end{frame}

\begin{frame}{AA (1963)}
\textbf{A1--A3} give:
$$
    U(f) = \sum_\omega \sum_x \underbrace{f(\omega)(x)}_{Pr(x)\text{ at }\omega} u(x|\omega)
$$
\textbf{A4:} Monotonicity: $(\forall\ \omega)\ f(\omega)\succeq g(\omega) \Rightarrow f\succeq g$\\
\textbf{A4:} CompIND: $p \succeq q \Rightarrow [f|p,i] \succeq [f|q,i]$ (HW: CompInd=MONO)\\
\textbf{A5:} Non-degeneracy: $(\exists\, f,g)\ :\ f\succ g$
\br
MONO (aka ``State independence'') gives $u(x|\omega)=\alpha(\omega) u(x)$\\
$u(x)$ captures the curvature of $u(x|\omega)$, $\alpha(\omega)$ captures its ``height''
\br
Normalize these $\alpha(\omega)$'s to sum to one. Now it's a belief over $\Omega$!!\\
\textbf{AA EU Theorem:} If $\succeq$ satisfies Ordering, Continuity, MixIND in Objective Lotteries, Monotonicity, and Non-Degeneracy then $\exists p,u$:
$$
U(f) = \sum_\omega \frac{\alpha(\omega)}{\sum_{\omega'}\alpha(\omega')} \sum_x f(\omega)(x)u(x) = \sum_\omega \underbrace{p(\omega)}_{\text{Subj. belief}} \sum_x \underbrace{f(\omega)(x)}_{Pr(x)\text{ at }\omega}\ u(x)
$$
\end{frame}

\begin{frame}{Back to Machina-Schmeidler}
What if we want to relax EU? But maintain dominance and $\exists p$
\begin{itemize}
    \item Machina Schmeidler (1995): Prob. Sophistication in AA world
    \begin{enumerate}
        \item Replace A3 (MixIND) with FOSD Monotonicity
        \item Replace A4 with an axiom that only applies to two-outcome bets
        \begin{itemize}
            \item Thus, affects $p(\omega)$ but doesn't restrict risk prefs.
        \end{itemize}
    \end{enumerate}
    \textbf{A4*:} for disjoint $E$ and $F$,
    $$
        1E0F0 \sim (1,\alpha)E(1,\alpha)F0 \Rightarrow pEqFr \sim (\alpha p+(1-\alpha)q)E(\alpha p+(1-\alpha)q)Fr
    $$
    \br
    Indifferent between bet on $E$ and lower-stakes bet on $E\cup F$\\
    $\Rightarrow$ same indiff when payoffs are replaced by lotteries
    \br
    Allows for $p(E)=\alpha p(E\cup F)$, regardless of ``stakes''
    \br
    A1,A2,FOSD-MONO,A4* $\Rightarrow$ $\exists p,V: f\succeq g \iff V(r(f,p))\geq V(r(g,p))$
\end{itemize}
\end{frame}

\begin{frame}{Non-EU Theories}
What are the famous non-EU theories to explain Ellsberg, etc?
\begin{enumerate}
    \item Schmeidler: Choquet Expected Utility 
    \item Gilboa-Schmeidler: Maxmin Expected Utility
    \item KMM: Smooth Ambiguity Preference
    \item Seo: Two-Stage EU
    \item $\vdots$
\end{enumerate}
\br
But first, what exactly \textit{is} ambiguity aversion??
\br
More than just violating P4*. There's a direction to it...
\end{frame}

\begin{frame}{Schmeidler's Definition of Ambiguity Aversion}
An ambiguity averse person should recognize ``hedging'' opportunities
\begin{itemize}
    \item AA setting. Set $\Omega = \{\omega_1,\omega_2\}$
    \item $f=((\$100,1),(\$0,1))$ (a bet on $\omega_1$)
    \item $g=((\$0,1),(\$100,1))$ (a bet on $\omega_2$)
    \item Let $h=g$ (for use later)
    \item Suppose $f\sim g$, meaning $\{\omega_1\} \sim_\trianglerighteq \{\omega_2\}$
    \item $\frac{1}{2}f+\frac{1}{2}h = ((\$100,\frac{1}{2};\$0,\frac{1}{2}),(\$100,\frac{1}{2};\$0,\frac{1}{2}))$ pays same lottery in both states
    \item $\frac{1}{2}g+\frac{1}{2}h = g$, payoff still depends on the state
    \item First mixture ``hedges away'' ambiguity!
    \item $g\succeq f$ but $\frac{1}{2}f+\frac{1}{2}h \succ \frac{1}{2}g+\frac{1}{2}h$
    \item Violate AA's MixIND (thus, EU). Specifically, prefs are convex
    \item Convex preferences = ``hedging incentive''
\end{itemize}
\end{frame}

\begin{frame}{Schmeidler's Definition of Ambiguity Aversion}
\begin{center}
    \includegraphics[width=3.75in]{LectureSlides/graphics/ach/ConvexAmbiguityPrefs.png}
\end{center}
\phantom{Kinked-linear indifference curves best capture hedging incentives}
\end{frame}

\begin{frame}{Schmeidler's Definition of Ambiguity Aversion}
\begin{center}
    \includegraphics[width=3.75in]{LectureSlides/graphics/ach/ConvexAmbiguityPrefs2.png}
\end{center}
Kinked-linear indifference curves best capture hedging incentives
\end{frame}



\begin{frame}{Choquet EU (Schmeidler 1989)}
    \begin{itemize}
        \item Sort of like probability weighting...
        \item AA setting. $f=(p^1,p^2,\ldots,p^n)$, each $p^i\in \Delta(X)$
        \item Goal:
        $$
        U(f) = \sum_\omega v(\omega) \sum_x f(\omega)(x)\ u(x)
        $$
        \item But here $v(\cdot)$ can be non-additive: $v(A)+v(B)\neq v(A\cup B)$
        \begin{enumerate}
            \item $v(\emptyset)=0$, 
            \item $v(\Omega)=1$
            \item $A\subset B \Rightarrow v(A)\leq v(B)$
        \end{enumerate}
        \item Ambiguity aversion: ``subadditive'' $v(\cdot)$\\
        Technically, \textit{convex}: $v(A)+v(B)-v(A\cap B) \leq v(A\cup B)$
        \item Example: Bent coin. $v(H)=0.4$ and $v(T)=0.4$
        \begin{itemize}
            \item Normalize $u(1)=1$, $u(0)=0$
            \item Bet on heads: $U(f)=0.4\cdot 1\cdot u(1) + 0.4\cdot 1\cdot u(0) = 0.4$
            \item Bet on tails: $U(g)=0.4\cdot 1\cdot u(1) + 0.4\cdot 1\cdot u(0) = 0.4$
            \item Bet on a fair coin: $U(h) = 1\cdot \left(0.5\ u(1) + 0.5\ u(0)\right) = 0.5$\\
            (Recall $v(H\cup T)=1$)
        \end{itemize}
    \end{itemize}
\end{frame}

\begin{frame}{Choquet EU (Schmeidler 1989)}
The problem with standard (Reimann) integration:
\begin{itemize}
    \item Suppose $v$ is non-additive (eg, convex)
    \item $(1,E_1;1,E_2;0,E_3) = (1,E_1\cup E_2; 0,E_3)$ are identical
    \item Weight on winning event: $v(E_1)+v(E_2) \neq v(E_1\cup E_2)$. Different!
\end{itemize}
\br
Schmeidler's solution: use the Choquet (1955) integral!
\begin{itemize}
    \item Suppose $f=(x_1,E_1;\ldots,x_n,E_n)$, where $x_1 \geq x_2 \geq \cdots \geq x_n \geq 0$
    \item Reimann: $\sum_i x_i\ v(E_i)$
    \item Choquet: $\sum_i x_i \left[v(\cup_{j=1}^i E_j) - v(\cup_{j=1}^{i-1} E_j) \right]$
    \item Similar to RDU! Also avoids dominance violations
\end{itemize}
\br
Above example: 
$1\,[{\color{red}v(E_1)}] + 1\, [{\color{blue}v(E_1\cup E_2)}{\color{red}- v(E_1)}] + {\color{red}0}\,[ v(E_1\cup E_2 \cup E_3)-v(E_1\cup E_2) ]$\\
$=1\,[{\color{blue}v(E_1\cup E_2)}] + {\color{red}0}\,[v(E_1\cup E_2 \cup E_3) - v(E_1\cup E_2)]$

\end{frame}


\begin{frame}{Choquet EU (Schmeidler 1989)}
\begin{center}
    \includegraphics[width=2in]{LectureSlides/graphics/ach/ConvexAmbiguityPrefs2.png}
\end{center}
Choquet calculation implies linear between two blue dots:
\begin{align*}
    U(a) &= 0.95\, [v(\omega_2)] + 0.55\, [v(\Omega)-v(\omega_2)]\\
    U(b) &= 0.75\, [v(\omega_2)] + 0.60\, [v(\Omega)-v(\omega_2)]\\
    U(\frac{1}{2}a+\frac{1}{2}b) &= 0.85\, [v(\omega_2)] + 0.575\, [v(\Omega)-v(\omega_2)] = \frac{1}{2}U(a)+\frac{1}{2}U(b)
\end{align*}
Linear on either side of the 45-degree line!    
\end{frame}

\begin{frame}{Choquet EU (Schmeidler 1989)}
    Points on the same side of the 45-degree line are ``comonotonic''
    \begin{center}
        $f(\omega)\geq f(\omega') \Rightarrow g(\omega)\geq g(\omega')$
    \end{center}
    Change MixIND to \textbf{A3*:} Comonotonic Independence.\\
    MixIND but only for comonotonic acts $f$ and $g$\\
    (Constant acts are comonotonic, so we do get EU over lotteries)
    \br
    \textbf{Theorem:} \textbf{A1}, \textbf{A2}, \textbf{ComonotonicIND}, \textbf{A4}, \textbf{A5} $\Rightarrow$\\
    $\exists$ non-additive $v(\cdot)$ and $\exists\ u(\cdot)$ unique up to p.a.t. s.t.
    $$
        U(f) = \sum_i \underbrace{\left( v(\cup_{j=1}^i \omega_j) - v(\cup_{j=1}^{i-1} \omega_j) \right)}_{\text{Non-additive ``belief'' of }\omega_i} \underbrace{\sum_x f(\omega_i)(x)\ u(x)}_{\text{EU over lottery }f(\omega_i)} 
    $$
    Can have ambiguity-loving if $v(\cdot)$ is ``superadditive''\\
    Still satisfies Mono/CompIND (\textbf{A4})! Good news for experiments\\
    His student Itzhak Gilboa extended this to the Savage framework
\end{frame}

\begin{frame}{Choquet EU becomes Maxmin EU}
An alternative representation
\begin{itemize}
    \item Suppose you have Choquet EU with ambiguity aversion
    \item Ambiguity aversion $\Rightarrow$ $v(\cdot)$ is convex
    \begin{itemize}
        \item $v(A)+v(B) - v(A\cap B) \leq v(A\cup B)$
    \end{itemize}
    \item Convex $v(\cdot)$ has a ``core'' of additive distributions...
    \begin{itemize}
        \item $C_v = \{p\in \Delta(\Omega)\, :\, (\forall E\subseteq \Omega)\ p(E)\geq v(E)\}$
        \item Ex: if $v(H)=v(T)=0.4$ then $C_v=\{p\in \Delta(\{H,T\}): p(H)\in [0.4,0.6]\}$
    \end{itemize}
    \item ... the Choquet expectation can equivalently be written as
    \begin{align*}
        \sum_i \left[v(\cup_{j=1}^i \omega_j) - v(\cup_{j=1}^{i-1} \omega_j) \right] \left(\sum_x f(\omega_i)(x)\ u(x)\right)\\
        = \min_{p\in C_v} \sum_\omega p(\omega) \left( \sum_x f(\omega)(x)\ u(x) \right)
    \end{align*}
    \item Convex $v(\cdot)$ $\Rightarrow$ ``Maxmin expected utility (MEU)''
\end{itemize}
\end{frame}

\begin{frame}{Gilboa \& Schmeidler (1989) Maxmin Expected Utility (MEU)}
    \begin{itemize}
        \item \textbf{A1}: Ordering (complete \& transitive)
        \item \textbf{A2}: Continuity
        \item \textbf{A3.1}: Certainty Independence: Let $h_c=(q,q,\ldots,q)$ be any constant act
        $$
            f\succ g \iff \alpha f + (1-\alpha) h_c \succ \alpha g + (1-\alpha) h_c
        $$
        (This gives kinked-linear indifference curves)
        \item \textbf{A3.2}: Uncertainty Aversion: $f\sim g \Rightarrow \alpha f + (1-\alpha) g \succeq f$\\
        (Convex indifference curves)
        \item \textbf{A4}: Monotonicity / Compound Independence
        \item \textbf{A5}: Non-degeneracy
    \end{itemize}
    $\Rightarrow \exists$ closed, convex set of beliefs $C$ and $u$ s.t.
    $$
        U(f) = \min_{p\in C} \sum_\omega p(\omega) \left( \sum_x f(\omega)(x)\ u(x) \right)
    $$
\end{frame}

\begin{frame}{Choquet EU vs. MEU?}
Choquet EU ($v(\cdot)$) to MEU ($C$)?
\begin{itemize}
    \item If $v(\cdot)$ is convex (ambiguity averse) then $C$ is core of $v(\cdot)$ \checkmark
    \item If $v(\cdot)$ is additive then core is just $v(\cdot)$, so CEU=MEU=EU
    \item If $v(\cdot)$ is amb. loving then $\not\exists$ MEU representation
    \begin{itemize}
        \item Max-EU representation??
    \end{itemize}
    \item Aribtrary $v(\cdot)$??
\end{itemize}
\br
MEU ($C$) to Choquet EU ($v(\cdot)$)?
\begin{itemize}
    \item Given $C$, you can define $v(E)=\min_{p\in C}p(E)$, but...
    \begin{enumerate}
        \item $v(\cdot)$ may not be convex, and even if it is...
        \item its core $C_v$ may be different from the $C$ you started with
    \end{enumerate}
\end{itemize}
\br
I don't fully understand the differences :)
\end{frame}

\begin{frame}{$\alpha$-Maxmin}
$\alpha$-Maxmin EU is a generalization:
\begin{align*}
U(f) &= \alpha \min_{p\in C} \sum_\omega p(\omega) \left( \sum_x f(\omega)(x)\ u(x) \right)\\
    &+ (1-\alpha) \max_{p\in C} \sum_\omega p(\omega) \left( \sum_x f(\omega)(x)\ u(x) \right)
\end{align*}
\br
Things I haven't figured out yet:
\begin{itemize}
    \item Who invented this model?
    \item What do indifference curves look like?
    \item Does an axiomatization exist?
\end{itemize}
\end{frame}

\begin{frame}{Seo (2009) and Order Reversal}
\begin{center}
    \includegraphics[width=1.5in]{LectureSlides/graphics/ach/trees_marblefirst.png} \hspace{0.in} \includegraphics[width=1.5in]{LectureSlides/graphics/ach/trees_coinfirst.png}
\end{center}
What if we flipped the order? Would it matter?
\br
Modeling problem: How to put these in the same framework?
\br
Solution: Lotteries over acts over lotteries!\\
Actually used in the original AA paper
\end{frame}

\begin{frame}{Seo (2009) and Order Reversal}
    Let $P=(f^1,P_1;\ldots,f^n,P_n)$ be a (first-stage) lottery over acts.
    \br
    First-stage mixture operation (``in between'' $P$ and $Q$):
    \begin{align*}
    \alpha P \oplus (1-\alpha) Q &= (f^1,\alpha P_1 + (1-\alpha)Q_1;\ldots,f^n,\alpha P_n + (1-\alpha)Q_n)
    \end{align*}
    If first stage is degenerate, acts like a compounding mixture:
    \begin{align*}
    \alpha \delta_f \oplus (1-\alpha) \delta_g &= (f,\alpha;g,1-\alpha)
    \end{align*}
    First-Stage IND:
    $$
        P\succeq Q \Rightarrow \alpha P \oplus (1-\alpha) R \succeq \alpha Q \oplus (1-\alpha) R
    $$
    
\end{frame}

\begin{frame}{Seo (2009) and Order Reversal}
    Let $f=(p^1,\omega_1;\ldots;p^n,\omega_n)$ and $g=(q^1,\omega_1;\ldots;q^n,\omega_n)$
    \br
    Second-stage mixture operation (MixIND at each $\omega$):
    \begin{align*}
        \alpha \delta_f + (1-\alpha) \delta_g = (\alpha p^1 + (1-\alpha)q^1,\omega_1;\ldots;\alpha p^n + (1-\alpha) q^n)
    \end{align*}
    \br
    Apply to degenerate $f$,$g$ to get Third-Stage IND (vNM):
    $$
        p\succeq q \Rightarrow \alpha p + (1-\alpha) r \succeq \alpha q + (1-\alpha) r
    $$
    (technically, should write $\delta_{\delta_p}$, $\delta_{\delta_q}$, and $\delta_{\delta_r}$ for $p$, $q$, and $r$)
\end{frame}

\begin{frame}{Seo (2009) and Order Reversal}
    Order Reversal Example:\\
    Bet on red:\ \ \ $f(\text{red})=(\$1,1)$, $f(\text{blue})=(\$0,1)$\\
    Bet on blue: $g(\text{red})=(\$0,1)$, $g(\text{blue})=(\$1,1)$
\begin{center}
    \begin{tabular}{ccc}
    \includegraphics[height=2in]{LectureSlides/graphics/ach/threestage_coinfirst.png}     &  & \includegraphics[height=2in]{LectureSlides/graphics/ach/threestage_coinsecond.png} \\
        $\alpha \delta_f \oplus (1-\alpha) \delta_g$ & $\sim$ & $\alpha \delta_f + (1-\alpha) \delta_g$
    \end{tabular}
     \hspace{0.25in} 
\end{center}
\end{frame}

\begin{frame}{Seo (2009) and Order Reversal}
    Order Reversal: mixing $f$ and $g$ ``up'' or ``down'' doesn't matter.
    \br
    Recall first-stage mixing for degenerate first stage (coin before):
    \begin{align*}
    \alpha \delta_f \oplus (1-\alpha) \delta_g &= (f,\alpha;g,1-\alpha)
    \end{align*}
    And second-stage mixing for degenerate first stage (coin after):
    \begin{align*}
        \alpha \delta_f + (1-\alpha) \delta_g = (\alpha p^1 + (1-\alpha)q^1,\omega_1;\ldots;\alpha p^n + (1-\alpha) q^n)
    \end{align*}
    \br
    Order Reversal:
    $$
        \alpha \delta_f \oplus (1-\alpha) \delta_g \sim \alpha \delta_f + (1-\alpha) \delta_g
    $$
    (AA 1963 actually had Order Reversal, but it's since been simplified)
\end{frame}

\begin{frame}{Seo (2009) and Order Reversal}
The AA Axioms in this 3-stage framework:
\begin{itemize}
    \item \textbf{A1:} Ordering
    \item \textbf{A2:} Continuity
    \item \textbf{A3.1:} First-Stage Independence
    \item \textbf{A3.2:} Order Reversal
    \item \textbf{A3.3:} Third-Stage Independence
    \item \textbf{A4:} Second-Stage Monotonicity
    $$
        p\succeq q \Rightarrow [f|p,i] \succeq [f|q,i]
    $$
\end{itemize}
\textbf{Theorem:} A1--A4 imply $\exists$ additive belief $p$ and utility index $u$ s.t
$$
    U(P) = \sum_i P_i \sum_\omega p(\omega) \sum_x f^i(\omega)(x)\ u(x)
$$
\end{frame}

\begin{frame}{Seo (2009) and Order Reversal}
Which axiom is important for experiments that pay one randomly?\\
\begin{itemize}
    \item Monotonicity is in second stage (acts), so no.
    \item We need 1st-Stage IND for degenerate acts!
    \begin{itemize}
        \item $\alpha \delta_f \oplus (1-\alpha) \delta_g = (f,\alpha;g,1-\alpha)$
    \end{itemize}
\end{itemize}
\br
Fact: Order Reversal ``connects'' the two lottery stages:
\begin{itemize}
    \item O.R. + 1st-Stage IND $\Rightarrow$ 3rd-Stage IND
    \item O.R. + 3rd-Stage IND $\Rightarrow$ 1st-Stage IND
\end{itemize}
    
\end{frame}

\begin{frame}{Order Reversal and Incentive Compatibility}
But that can be a problem!\\
Remember for 2-stage lotteries we had:
\begin{center}
    CompIND + ROCL $\Rightarrow$ MixIND $\Rightarrow$ EU\\
    or\\
    non-EU theory $\Rightarrow$ $\neg$ MixIND $\Rightarrow$ ($\neg$ CompIND) or ($\neg$ ROCL)\\
    non-EU theory $\Rightarrow$ $\neg$ MixIND $\Rightarrow$ ($\neg$ I.C.) or ($\neg$ ROCL)\\
\end{center}
\br
In this AA framework we have:
\begin{center}
    1stStageIND + OR $\Rightarrow$ 3rdStageIND $\Rightarrow$ EU\\
    or\\
    non-EU theory $\Rightarrow$ $\neg$ 3rdStageIND $\Rightarrow$ ($\neg$ 1stStageIND) or ($\neg$ OR)\\
    non-EU theory $\Rightarrow$ $\neg$ 3rdStageIND $\Rightarrow$ ($\neg$ I.C.) or ($\neg$ OR)\\
\end{center}
\br
If you want to allow for non-EU preferences, we better hope that ROCL or OR are not satisfied!
\end{frame}

\begin{frame}{Seo (2009) and Second-Order SEU}
Seo (2009): What if we get rid of OR but keep both IND axioms?
\br
Needs to modify Monotonicity to apply to first stage instead.\\
(AA didn't need this because they had O.R. to do it for them)
\br
Let $r(P,q)$ be the reduced (two-stage) ``objective'' lottery generated by applying belief $q(\omega)$ over $\Omega$. So, combine stages 2 and 3.\\
Note: $\succeq$ ranks 2-stage lotteries via degenerate 2nd stage (acts)
\br
1st-Stage Dominance: $P\sqsupseteq Q$ iff $r(P,q)\succeq r(Q,q)$ for \textit{every} $q\in\Delta(\Omega)$
\br
\textbf{A4*:} If $P\sqsupseteq Q$ then $P\succeq Q$ 
\br
Question: How does this compare to 1st-Stage IND?
\end{frame}

\begin{frame}{Seo (2009) and SOSEU}
Seo's axioms:
\begin{itemize}
    \item \textbf{A1:} Ordering
    \item \textbf{A2:} Continuity
    \item \textbf{A3.1:} First-Stage Independence
    \item \sout{\textbf{A3.2:} Order Reversal}
    \item \textbf{A3.3:} Third-Stage Independence
    \item \textbf{A4*:} First-Stage Monotonicity
\end{itemize}
\textbf{Theorem:} A1--A4* imply $\exists$ a \textit{belief over beliefs} $\pi\in\Delta(\Delta(\Omega))$, utility $u$, and a bounded, increasing function $\phi$:
$$
    U(P) = \sum_i P_i \sum_{q\in \Delta(S)} \underbrace{\pi(q)}_{\text{Pr(belief}=q\text{)}}\ \phi ( \underbrace{\sum_\omega q(\omega) \sum_x f^i(\omega)(x)\ u(x)}_{\text{SEU of }f\text{ w/ belief }q} )
$$
Ambiguity averse $\iff$ $\phi$ is concave
\end{frame}

\begin{frame}{Seo (2009) SOSEU \& Hedging}
\hspace{1in}\begin{tabular}{cc|cc}
      & 30 marbles & \multicolumn{2}{c}{60 marbles}  \\
      & R & B & Y \\
      \hline
      $f$ & 0 & 100 & 0 \\
      $g$ & 0 & 0 & 100\\
      \hline
\end{tabular}\\
New state space: \# black marbles: $\Omega=\{0,1,2,\ldots,60\}$\\
$f(\omega)= (\frac{\omega}{90},100;\frac{90-\omega}{90},0)$\ \ \ \ 
$g(\omega)= (\frac{60-\omega}{90},100;\frac{90-(60-\omega)}{90},0)$\\
\hspace{0.75in} $(\frac{1}{2}f+\frac{1}{2}g)(\omega) = (\frac{1}{3},100;\frac{2}{3},0)\ \ \forall \omega$\\
Suppose $q^1(20)=1$ and $q^2(40)=1$, with $\pi(q^1)=\pi(q^2)=1/2$\\
Normalize $u(100)=1$, $u(0)=0$\\
\begin{tabular}{ll}
EU of $f$ at $q^1$: 20/90 & EU of $g$ at $q^1$: 40/90\\
EU of $f$ at $q^2$: 40/90 & EU of $g$ at $q^2$: 20/90\\
$U(f)=\frac{1}{2}\phi(2/9)+\frac{1}{2}\phi(4/9)$ & $U(g)=\frac{1}{2}\phi(4/9)+\frac{1}{2}\phi(2/9)$
\end{tabular}
\begin{center}
EU of $0.5f+0.5g$ at $q^1$ or $q^2$: $\frac{1}{2}\frac{20}{90}+\frac{1}{2}\frac{40}{90}=\frac{1}{3}$\\
$U(0.5f+0.5g) = \frac{1}{2}\phi(1/3)+\frac{1}{2}\phi(1/3)=\phi(1/3)$\\
Concave $\phi$ $\Rightarrow$ $\phi(1/3)>\frac{1}{2}\phi(4/9)+\frac{1}{2}\phi(2/9)$\\
Convex preferences / preference for hedging
\end{center}
\end{frame}

\begin{frame}{Smooth Ambiguity}
Klibanoff, Marinacci \& Mukerji (KMM, 2005)\\
``Smooth ambiguity'' has basically the same form, but in a framework without ``time''
\br
Frameworks with multiple sources but without time?
\br
Savage: \hspace{0.25in}
\begin{tabular}{c|c|c|}
     & $\psi_1$ & $\psi_2$  \\
     \hline
     $\xi_1$ & $\omega_1$ & $\omega_2$ \\
     \hline
     $\xi_2$ & $\omega_3$ & $\omega_4$ \\
     \hline
\end{tabular} \hspace{0.25in} 
\br
AA: Can have $\succeq$ over $\mathcal{F}\cup \Delta(X)$\\
...but then you'll need different axioms for $f\succeq g$, $p\succeq q$, and $f\succeq p$
\br
Literature on ``source dependence''
\end{frame}

\begin{frame}{Time and Risk}
Discounted expected utility for a stream of lotteries:
$$
    U(p^0,p^1,p^2,\ldots) = \sum_{t=0}^\infty \delta^t (\sum_x p^t(x)\ u(x))
$$
Problem: $u$ represents both risk preferences \textit{and} time preferences!
\br
Models that separate them:
\begin{itemize}
    \item Kreps-Porteus (1978)
    \item Chew-Epstein (1989)
    \item Epstein-Zin (1989)
\end{itemize}
\br
DeJarnette et al. (2020): If payment date is uncertain (``time lottery'') then DEU predicts risk-seeking preferences! But experiment shows risk aversion. They provide new generalizations of DEU. 
\end{frame}

\begin{frame}{That's It!}
\begin{itemize}
    \item That's the end of this primer
    \item The DT literature is huge, and worth exploring.
    \item But sometimes axioms feel like framing effects
    \begin{itemize}
        \item Example: Order Reversal
        \item And, can we really control which order subjects perceive?
    \end{itemize}
    \item ...so things get incredibly nuanced
    \item ...which unfortunately leads to a lot of fights
\end{itemize}
\br
Regardless, this lays the foundation for formulating a theory of incentives in experiments
    
\end{frame}

\end{document}