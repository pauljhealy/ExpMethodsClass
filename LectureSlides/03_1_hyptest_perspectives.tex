\documentclass[10pt,t,english]{beamer}
\usepackage{fontawesome}
\usepackage{graphicx}
\usepackage{array}
\usepackage[normalem]{ulem}
\usepackage{amsfonts,amsmath,amssymb,bm,bbm}
\usepackage{mathrsfs}
\usepackage{sgame}
\usepackage{graphicx,pstricks}
\usepackage{xcolor}
\usepackage{colortbl}
\usepackage{makecell}
\usepackage{tikz,tikzsymbols,gnuplottex}
\usetikzlibrary{decorations.pathreplacing,shapes}
\usepackage[english]{babel}
\usepackage[utf8]{inputenc}
\usepackage{appendixnumberbeamer}
\usepackage{datetime2}
\usepackage{booktabs}
\usepackage{setspace}
\usepackage{rotating}
\usepackage{natbib}
\usepackage{listings}

%\usepackage{matlab-prettifier} % For enhanced MATLAB highlighting
\lstset{
    language=matlab,
    % Other options for styling (optional)
    basicstyle=\ttfamily\footnotesize, % Font style
    keywordstyle=\color{blue}, % Keyword color
    commentstyle=\color{green!50!black}, % Comment color
    stringstyle=\color{red!70!black}, % String color
    numbers=left, % Line numbers on the left
    numberstyle=\tiny\color{gray}, % Style for line numbers
    frame=single, % Frame around the listing
    breaklines=true, % Allow line breaking
    captionpos=b, % Caption at the bottom
    tabsize=4 % Tab size
}

% ------------------------------------------------------------------------------
% Use the beautiful metropolis beamer template
% ------------------------------------------------------------------------------
\usepackage[T1]{fontenc}
\usepackage[utf8]{inputenc}
\usepackage{fontawesome}
\usepackage{FiraSans} 
\mode<presentation>
{
  \usetheme[progressbar=foot,background=light]{metropolis} 
  \usecolortheme{default} % or try albatross, beaver, crane, ...
  \usefonttheme{default}  % or try default, serif, structurebold, ...
  \setbeamertemplate{navigation symbols}{}
  \setbeamertemplate{caption}[numbered]
  %\setbeamertemplate{frame footer}{My custom footer}
} 

\newenvironment{stepenumerate}{\begin{enumerate}[<+->]}{\end{enumerate}}
\newenvironment{stepitemize}{\begin{itemize}[<+->]}{\end{itemize} }
\newenvironment{stepenumeratewithalert}{\begin{enumerate}[<+-| alert@+>]}{\end{enumerate}}
\newenvironment{stepitemizewithalert}{\begin{itemize}[<+-| alert@+>]}{\end{itemize} }

\newtheorem{question}{Question}
\newtheorem{claim}{Claim}
\newtheorem{proposition}{Proposition}
\newtheorem{remark}{Remark}
\newtheorem{conjecture}{Conjecture}

\definecolor{metrop}{RGB}{29, 44, 44}
\colorlet{GrayLight}{black!15}
\colorlet{GrayMedium}{black!30}
\colorlet{ForestGreen}{green!60!black}

\newenvironment{transitionframe}{
  \setbeamercolor{background canvas}{bg=black!80}
  \begin{frame}}{
    \end{frame}
}

\newcommand{\br}{

\bigskip

}

\newcommand{\pd}{\partial}
\newcommand{\RR}{\mathbb{R}}

\newcommand*\hugme[1]{\tikz[baseline=(char.base)]{\node[shape=ellipse,draw,inner sep=0pt] (char) {#1};}}

\newcounter{saveenumi}
\newcommand{\seti}{\setcounter{saveenumi}{\value{enumi}}}
\newcommand{\conti}{\setcounter{enumi}{\value{saveenumi}}}
\resetcounteronoverlays{saveenumi}



\newcommand\dotprod[2]{\langle #1 , #2 \rangle}
\newcommand{\ft}[1]{\widehat #1}
\newcommand{\qabove}[1]{\overset{\text{\large \textbf ?}}{#1}}
\newcommand{\eqae}{\overset{\text{a.e.}}{=}}
\newcommand{\calp}{\mathcal{P}}
\newcommand{\calg}{\mathcal{G}}
\newcommand{\calb}{\mathcal{B}}
\newcommand{\textd}{\text{d}}
\newcommand{\bbr}{\mathbb{R}}
\newcommand{\binm}{\mathbin{M}}
\newcommand{\binc}{\mathbin{C}}
\newcommand{\binb}{\mathbin{B}}
\newcommand{\calc}{\mathcal{C}}
\newcommand{\calh}{\mathcal{H}}
\newcommand{\bfone}{\mathbf{1}}
\newcommand{\bbe}{\mathbb{E}}
\newcommand{\bfle}{\mathbf{e}}
\newcommand{\calf}{\mathcal{F}}
\newcommand{\cala}{\mathcal{A}}
\newcommand{\cale}{\mathcal{E}}
\newcommand{\bbn}{\mathbb{N}}
\newcommand{\cantor}{\calc}
\newcommand{\calY}{\mathcal{Y}}
\newcommand{\textb}{\text{B}}
\newcommand{\calm}{\mathcal{M}}
\newcommand{\bint}{\mathbin{T}}
\newcommand{\ep}{\epsilon}
\newcommand{\bbq}{\mathbb{Q}}
\newcommand{\bbp}{\mathbb{P}}
\newcommand{\cals}{\mathcal{S}}
\newcommand{\emptysequence}{e}
\newcommand{\bbz}{\mathbb{Z}}
\newcommand{\fraka}{\frak{A}}
\newcommand{\frakb}{\frak{B}}
\newcommand{\length}{\text{length}}
\newcommand{\bfn}{\mathbf{N}}
\newcommand{\support}{\text{support}}
\DeclareMathOperator*{\argmax}{arg\,max}
\newcommand{\dom}{\mbox{dom}}
\def\ut{\underline t}
\def\um{\underline m}
\def\PP{\mathbb{P}}
\def\EE{\mathbb{E}}
\def\RR{\mathbb{R}}

% \documentclass[11pt,aspectratio=169]{beamer}

% % Packages
% \usepackage[utf8]{inputenc}
% \usepackage{amsmath, amsfonts, amssymb, amsthm}
% \usepackage[T1]{fontenc}
% \usepackage{xcolor}
% \usepackage{setspace}
% \usepackage{graphicx}
% \usepackage{natbib}
% \usepackage{appendixnumberbeamer}
% \usepackage{tikz}
% \usepackage{bm}
% \usepackage{booktabs}

% % Theme
% \usetheme{Boadilla}
% \usecolortheme{default}
% \usefonttheme[onlymath]{serif}

% % Customize template
% \setbeamertemplate{itemize item}[circle]
% \setbeamertemplate{itemize subitem}{$\circ$}
% \setbeamertemplate{itemize subsubitem}{--}
% \setbeamercolor{itemize item}{fg=black}
% \setbeamercolor{itemize subitem}{fg=black}
% \setbeamercolor{itemize subsubitem}{fg=black}
% \setbeamercolor{item projected}{bg=darkgray,fg=white}
% \definecolor{blue}{rgb}{0.2, 0.2, 0.7}
% \definecolor{red}{rgb}{0.9, 0.1, 0.1}
% \setbeamercolor{alerted text}{fg=blue}
% \setbeamertemplate{enumerate items}[circle]
% \setbeamertemplate{headline}{}
% \let\olditemize=\itemize
% \let\endolditemize=\enditemize
% \renewenvironment{itemize}{\olditemize \itemsep0em}{\endolditemize}
% \let\oldenumerate=\enumerate
% \let\endoldenumerate=\endenumerate
% \renewenvironment{enumerate}{\oldenumerate \itemsep0em}{ \endoldenumerate}


% % Math formatting
% \theoremstyle{definition}
% \newtheorem{defn}{Definition}
% \theoremstyle{plain}
% \newtheorem{prop}{Proposition}
% \newtheorem{thm}{Theorem}
% \newtheorem{lem}{Lemma}
% \newtheorem{corol}{Corollary}
 \newtheorem{per}{Perspective}
% \DeclareMathOperator*{\argmax}{\arg\!\max}
% \DeclareMathOperator*{\E}{\mathbb{E}}
% \DeclareMathOperator*{\R}{\mathbb{R}}
% \DeclareMathOperator*{\var}{\rm Var}
% \DeclareMathOperator*{\cov}{\rm Cov}
% \DeclareMathOperator*{\supp}{\rm Supp}

% % Custom commands
 \newcommand{\leftcol}{}
 \newcommand{\maincol}{}
 \newcommand{\margin}{}
 \newcommand{\myspacing}{}	
 \newcommand{\thus}{$\Rightarrow$ }

% % Other hacks
% \pdfstringdefDisableCommands{\def\translate#1{#1}}
% \vfuzz=100pt
% \hfuzz=100pt
% \setbeamercovered{transparent}

% Title info
\title[Hypothesis Test Perspectives]{ExpEcon Methods:\\
Intro to Hypothesis Testing and\\Fay \& Proschan's (2010) ``Perspectives''}
  \author[ECON 8877]{ECON 8877\\P.J. Healy\\First version thanks to Sungmin Park}
  \institute[OSU]{}
\date[]{\vfill {\tiny Updated \today}}
%\setbeamertemplate{navigation symbols}{}

\begin{document}
%\setstretch{1.25}

\begin{frame}
    \maketitle
\end{frame}

\begin{frame}{A Primer/Reminder on Hypothesis Tests}
\begin{itemize}
    \item Assumed model/DGP (``population''): $F(X;\theta)$, $\theta\in \Theta$
    \item Sample (r.v.): $X$. Often $X=(X_1,\ldots,X_n)$. Realization: $x$
    \item Sample statistic: $W(X)$
    \begin{itemize}
        \item Ex: 2 groups. $X_i=(Y_i^0,Y_i^1)$, $W(X)=\sum_i Y_i^0/n-\sum_iY_i^1/n$
    \end{itemize}
    \item Hypotheses: $H_0:\theta\in\Theta_0$, $H_1:\theta\in\Theta_1$
    \begin{itemize}
        \item Here: Assume $\Theta_1=\Theta\setminus\Theta_0$
    \end{itemize}
    \item Do not reject $H_0$ if $W(x)\in W_0\subseteq \Bbb{R}$
    \item Reject if $W(x)\in W_0^C$. Let $R=\{x:W(x)\in W_0^C\}$
    \begin{itemize}
        \item Decision rule: $\delta(x;W_0^C)=1$ if $W(x)\in W_0^C$
        \item Decision rule: $\delta(x;W_0^C)=0$ if $W(x)\not\in W_0^C$
    \end{itemize}
    \item Rejection region (in sample space): reject $H_0$ if $x\in R$
    \begin{itemize}
        \item Pr(Type I Error at $\theta\in\Theta_0$) = $P_\theta(X\in R)$
        \item Pr(Type II Error at $\theta\in\Theta_0^C$) = $P_\theta(X\not\in R)=1-P_\theta(X\in R)$
    \end{itemize}
    \item Power function: $\beta(\theta)=P_\theta(X\in R)$
    \begin{itemize}
        \item Ideal: $\beta(\theta)=0\ \forall \theta\in\Theta_0$, $\beta(\theta)=1\ \forall\theta\in\Theta_0^C$
    \end{itemize}
\end{itemize}
\end{frame}

\begin{frame}{Primer: One-Sided Test}
    Example: $X=(X_1,\ldots,X_n)\in\Bbb{R}^n$, iid $N(\theta,\sigma^2)$, $\sigma^2$ known.
    \begin{itemize}
        \item Question: $H_0:\theta\leq\theta_0$ for some $\theta_0$ (given by research question)
        \item Test statistic: $W(X)=\frac{\bar{X}-\theta_0}{\sigma/\sqrt{n}}$ (average ``$z$-score'')
        \item Reject if $W(X)>c$ for some $c$ (that you choose)
        \begin{itemize}
            \item Why? Equivalent to LRT $sup_{\theta\in\Theta_0}L(\theta|x)/sup_\theta L(\theta|x)<\lambda$
            \item Parametric distribution $\Rightarrow$ $L(\theta|x)$ known
        \end{itemize} 
        \item How to pick $c$? Want $\beta(\theta)\leq 0.05$ for all $\theta\leq\theta_0$. So
        \begin{align*}
            \beta(\theta) &= P_\theta\left(\frac{\bar{X}-\theta_0}{\sigma/\sqrt{n}}> c\right)\\
            &= P_\theta\left(\frac{\bar{X}-\theta}{\sigma/\sqrt{n}}> c+\frac{\theta_0-\theta}{\sigma/\sqrt{n}}\right)\\
            &= P_\theta\left(Z>c+\frac{\theta_0-\theta}{\sigma/\sqrt{n}}\right)=1-\Phi\left(c+\frac{\theta_0-\theta}{\sigma/\sqrt{n}}\right)
        \end{align*}
        \item Increasing in $\theta$ $\Rightarrow$ Want $\beta(\theta_0)=1-\Phi(c)=0.05$ $\Rightarrow$ Set $c=1.645$
    \end{itemize}
\end{frame}

\begin{frame}{Primer: One-Sided Test}
    Example: $X=(X_1,\ldots,X_n)\in\Bbb{R}^n$, iid $N(\theta,\sigma^2)$, $\sigma^2$ known.\\
    Test: Reject if $W(X)=\frac{\bar{X}-\theta_0}{\sigma/\sqrt{n}} > c=1.645$\\
    Graph: $\beta(\theta)$ for $\theta_0=0$, $n=10$\\
    \begin{center}
        \includegraphics[height=1.5in]{LectureSlides/graphics/PowerFunctionNormalEx.png}
    \end{center}
    Maximum type-I error (among $\theta\leq \theta_0=0$) is 0.05 (by design)\\
    Size of test: $\alpha:=\sup_{\theta\in\Theta_0}\beta(\theta)$\\
    ``I want 80\% power.'' OK but... at which $\theta$???
\end{frame}

\begin{frame}{Primer: $p$-Values}
So, what's a $p$-value?
\begin{itemize}
    \item In general, just another statistic $p(X)$
    \item But it's an alternative (equivalent) way to run the same test
    \item But most commonly, rejection rule is $R=\{x:p(x)< \alpha\}$ where
    \begin{align*}
        p(x)=\sup_{\theta\in\Theta_0} P_\theta(W(X)\geq W(x))
    \end{align*}
    \item ``Under $H_0$, what the probability of a more-extreme $W(X)$?''
    \item Reject iff $p(x)<\alpha$
    \begin{itemize}
        \item Decision rule: $\delta(x;\alpha)=1$ if $p(x)<\alpha$, $\delta(x;\alpha)=0$ if $p(x)\geq \alpha$
        \item Previously you chose $c$, reject if $W(X)> c$
        \item Now you choose $\alpha$, reject if $p(X)<\alpha$
    \end{itemize}
    \item This will generate a valid test for any $\alpha$
    \item One-sided test: reject if $W(X)>w^{**}$
    \item Two-sided test: reject if $W(x)\not\in (w^*,w^{**})$
\end{itemize}
\end{frame}

\begin{frame}{The Example}
Again, $X_i\sim N(\theta,\sigma^2)$, $H_0:\theta\leq \theta_0(=0)$.
\begin{align*}
    p(x) &= sup_{\theta\in\Theta_0} P_\theta(W(X)\geq W(x)) \\
    &= sup_{\theta\in\Theta_0} P_\theta\left( \frac{\bar{X}-\theta_0}{\sigma/\sqrt{n}} \geq \frac{\bar{x}-\theta_0}{\sigma/\sqrt{n}}\right)\\
    &= P_{\theta_0}\left( \frac{\bar{X}-\theta_0}{\sigma/\sqrt{n}} \geq \frac{\bar{x}-\theta_0}{\sigma/\sqrt{n}}\right) \\
    &= P_{\theta_0}\left( Z \geq W(x)\right)
\end{align*}
\begin{itemize}
    \item Usual rule: reject $H_0$ iff $p(x)<0.05$
    \item Note: $p(x)< 0.05$ iff $W(x)> 1.645$
    \item So, reject if $W(x)>1.645$. Same rule as before!!! $c=1.645$
    \item Has same meaning regardless of $W(\cdot)$ and $W_0$
    \item $p$-value gives a useful measure of ``how close you were'' to rejecting/not rejecting
\end{itemize}
\end{frame}

\begin{frame}{How To Simulate It}
    \begin{itemize}
        \item In the normal example, $\beta(\theta)$ has analytic solution
        \item In general might not exist/too hard to solve
        \item We can simulate it! Steps:
        \begin{enumerate}
            \item Set a grid of $\theta$ values
            \item Choose a sample size $n$ and number of ``runs'' $R$
            \item At each ``true'' $\theta$ generate $R$ iid samples of size $n$
            \begin{itemize}
                \item $r^\text{th}$ sample is $x_r^\theta=(x_{1,r}^\theta,\ldots,x_{n,r}^\theta)$ where $x_{i,r}^\theta\overset{iid}{\sim}F(\cdot,\theta)$
            \end{itemize}
            \item For each $x_r^\theta$ determine $\delta(x_r^\theta;\alpha)$
            \item Estimated power function: $\hat{\beta}(\theta)=\sum_{r=1}^R\frac{1}{R}\delta(x_r^\theta;\alpha)$
            \item Redo this for various $n$, $\alpha$, $W(x)$, whatever...\\
            Plot them all and check:
            \begin{enumerate}
                \item Is $\hat{\beta}(\theta)\leq 0.05$ when $\theta\in\Theta_0$?
                \item Which has the greatest $\hat{\beta}(\theta)$ when $\theta\not\in\Theta_0$?            
            \end{enumerate}
        \end{enumerate}
        \item Very useful for picking your actual sample size!
    \end{itemize}
\end{frame}

\begin{frame}[fragile]
    \frametitle{Example MATLAB Code}
    \begin{lstlisting}[title={Part 1: Setting Parameters}]
%% Set parameters
tnot = 0; % the cutoff theta_0
tgrid = [-2.5 -1 -0.5 -0.1 0 0.1 0.5 1 1.5 2 2.5 5 10]; %true means (theta)
n = 100;
c = 1.645;
sig = 10; %true std deviation
runs = 5000; %how many samples to generate for each t and n
    \end{lstlisting}
\end{frame}


\begin{frame}[fragile]
    \frametitle{Example MATLAB Code}
    \begin{lstlisting}[title={Part 2: Running the Simulation}]
%% Now run the simulation
for ti=1:length(tgrid)
    t = tgrid(ti); % current "true" theta
    fdist = makedist('Normal','mu',t,'sigma',sig); %true dist'n
    for r = 1:runs
        xr = random(fdist,n,1); %rth sample, an nx1 vector
        W(ti,r) = (mean(xr)-tnot)/(sig/sqrt(n)); %sample statistic
        rej(ti,r) = W(ti,r) > c; %reject or not
        pval(ti,r) = 1-normcdf(W(ti,r),tnot,1); %our p-value
    end
    rejavg(ti) = mean(rej(ti,:));
    pvalavg(ti) = mean(pval(ti,:));
end
    \end{lstlisting}
\end{frame}


\begin{frame}[fragile]
    \frametitle{Example MATLAB Code}
    \begin{lstlisting}[title={Part 3: Plotting Output}]
%% Now plot the output
figure; %open new figure
plot(tgrid,rejavg,"LineWidth",1.5,"Color","blue"); %plot rejavg (power)
title("n="+n+" c="+c+" \sigma="+sig);
xticks(tgrid);
ylim([0 1]);
hold on; %allows multiple plots
plot(tgrid,pvalavg,"LineWidth",1.5,"Color","red"); %plot pvalavg
plot(tgrid,0.05*ones(length(tgrid)),"LineStyle","--","Color","k"); %plot y=0.05 as black dashed line
plot([tnot tnot],[0 1],"LineStyle",":","Color","k"); %plot x=0
hold off;
    \end{lstlisting}
\end{frame}

\begin{frame}{Example MATLAB Simulation}
    \centering
    \includegraphics[width=4in]{LectureSlides/graphics/simplehypsim.png}
\end{frame}

\begin{frame}{Example MATLAB Simulation}
    \centering
    \includegraphics[width=4in]{LectureSlides/graphics/simplehypsim2.png}
\end{frame}


\begin{frame}{Primer: Two-Sided $t$-Test}
    Example: $X=(X_1,\ldots,X_n)\in\Bbb{R}^n$, iid $N(\theta,\sigma^2)$, $\sigma^2$ \textbf{NOT} known.
    \begin{itemize}
        \item Question: $H_0:\theta=\theta_0$, $H_1:\theta\neq \theta_0$
        \item Test statistic: $W(X)=\frac{\bar{X}-\theta_0}{\hat{\sigma}/\sqrt{n}}$ where $\hat{\sigma}=\sqrt{\sum_i \frac{1}{n-1}(X_i-\bar{X})^2}$
        \item Reject if $W(X)\not\in[-c,c]$ for some $c$. How to pick $c$?
        \item Want $\beta(\theta)\leq 0.05$ for all $\theta\in\Theta_0=\{\theta_0\}$. So
        \begin{align*}
            \beta(\theta_0) &= 2\cdot P_{\theta_0}\left(\frac{\bar{X}-\theta_0}{\hat{\sigma}/\sqrt{n}}> c\right)\\
             &= 2\cdot P_{\theta_0}\left(\frac{\frac{\bar{X}-\theta_0}{\sigma/\sqrt{n}}}{\sqrt{\frac{\sum_i (X_i-\bar{X})^2}{\sigma^2}\frac{1}{n-1}}}> c\right)\\
            &= 2\cdot P_{\theta_0}\left(\frac{Z}{\sqrt{\chi^2/(n-1)}}>c\right)=2\cdot(1-T_{n-1}(c))
        \end{align*}
        \item Want $\beta(\theta_0)=0.05$ $\Rightarrow$ Set $c=T^{-1}_{n-1}(0.975)$. Note: depends on $n$
        \item Does a closed-form solution exist for $\beta(\theta)$? Let's just simulate!!
    \end{itemize}
\end{frame}

\begin{frame}{Primer: Two-Sided Test}
\centering
\includegraphics[width=4in]{LectureSlides/graphics/PowerFunctionTwoSidedTTest.png}\\
$x$-axis: True mean. Blue: Power function. Red: Average $p$-value.
\end{frame}

\begin{frame}{Primer: Two-Sided Test}
\centering
\includegraphics[width=4in]{LectureSlides/graphics/PowerFunctionTwoSidedTTestHighVar.png}\\
Double the variance
\end{frame}

\begin{frame}{Primer}
Comparing tests
\begin{itemize}
    \item Suppose you have a class of tests $\mathcal{C}$ for a fixed $H_0$
    \item Fix $n$. Test w/ power function $\beta\in\mathcal{C}$ is \alert{uniformly most powerful (UMP)} if $\forall \beta'\in\mathcal{C},\ \forall \theta\in\Theta_0^C$, $\beta(\theta)\geq\beta(\theta')$
    \begin{itemize}
        \item May not exist
    \end{itemize}
    \item Test with $\beta$ is asymptotically (uniformly) most powerful if it becomes UMP as $n\rightarrow \infty$
    \item A test is \alert{valid} for size $\alpha$ if $\sup_{\theta\in\Theta_0}\beta(\theta)\leq \alpha$
    \item It is \alert{asymptotically valid} if it's valid as $n\rightarrow\infty$
\end{itemize}
\end{frame}


\begin{frame}{Now We're Ready}
    OK, now onto Fay \& Proschan (2010)...
\end{frame}

\begin{frame}{Two Popular Statistical Tests}
Two samples: $Y^0=(Y_1^0,\ldots,Y_n^0)$ and $Y^1=(Y_1^1,\ldots,Y_m^1)$\\
Assume $Y_i^0 \overset{iid}{\sim} F$ and $Y_i^1\overset{iid}{\sim}G$. Is $G$ ``bigger'' than $F$?\\
In what sense? Mean? Median? FOSD? What does a given test measure?
\begin{itemize}
    \item \textbf{Student's $t$-test}: reject if
        	\vspace{1em}
        	\begin{columns}[]
        		\column{0.2\linewidth}
        		\column{0.4\linewidth}
        		\vspace{-1em}
        		\begin{equation*}
        			\left| \frac{\hat\mu_1-\hat\mu_0}{\hat\sigma\sqrt{\frac{1}{n}+\frac{1}{m}}}
        			\right| > t^{-1}_{n+m-2}(1-\alpha/2),
        		\end{equation*}
        		\column{0.4\linewidth}\scriptsize
        	 \setstretch{1}where $\hat\sigma^2$ is the pooled sample variance and $t_d(\cdot)$ is the CDF of Student $t$ distribution with degree of freedom $d$.
        	\end{columns}
         \begin{itemize}
             \item Parametric. Test of means? Assumes normality?
         \end{itemize}
    \item \textbf{Wilcoxon/Mann-Whitney rank-sum test}: reject if \vspace{-1em}
\begin{equation*}\vspace{-0.5em}\setstretch{0.8}
		\sum_{i=1}^n \sum_{j=1}^m S(Y^0_i, Y^1_j) < U_{n_0,n_1}^{-1}(1-\alpha/2)
		\text{\quad where }
		S(x, y)=
		\begin{cases}
			1, & \text{ if } $x > y$,\\
			\frac{1}{2}, & \text{ if } $x = y$,\\
			0, & \text{ if } $x < y$.
		\end{cases}
\end{equation*}
    \begin{itemize}
        \item Non-parametric. Test of medians??? Assumes what???
    \end{itemize}
\end{itemize}
\end{frame}

\begin{frame}
  \frametitle{\citet{FayProschan2010}}
  \vspace{1em}
\begin{columns}[t]
	\column{0.1\linewidth}
	\column{0.2\linewidth}
	  \textbf{Question}
	 \column{0.6\linewidth}
     When is it appropriate to use \alert{Wilcoxon-Mann-Whitney (WMW) test}
     or \alert{$t$-test} to compare two samples?
     %\pause
     \begin{itemize}\footnotesize
         \item When is it \alert{valid} \& \alert{consistent}? When is it \alert{optimal}?
     \end{itemize}
	 \column{0.1\linewidth}
\end{columns}
%\pause
\vspace{1.5em}
\begin{columns}[t]
    \column{0.1\linewidth}
	\column{0.2\linewidth}
	\textbf{Answer}
	\column{0.6\linewidth}
    They are appropriate for different pairs of \alert{null} and \alert{alternative}
    hypotheses (\alert{``perspectives''})
\column{0.1\linewidth}
\end{columns}
\vspace{2em}
\begin{columns}
    \column{0.3\linewidth}
    \centering
    \includegraphics[width=1\linewidth]{LectureSlides/graphics/dist.pdf}
\end{columns}
\end{frame}


% -------------------------------------------
% \setbeamertemplate{headline} {
% 	\setbeamercolor{section in head/foot}{fg=black, bg=white} \vskip0.5em \scriptsize
% 	\insertsectionnavigationhorizontal{1\paperwidth}{\hspace{0.5\paperwidth}}{}
% }
\section{Illustration}
%------------------------------------------
\begin{frame}[t]\frametitle{Illustration: 9th Grade Math Ability of Boys \& Girls}
\vspace{-2em}
\begin{columns}[T]
		\column{0.55\linewidth}
        \begin{figure}
            \caption{Histograms of math ability}
        \includegraphics[width=1\linewidth]{LectureSlides/graphics/score.pdf}
        \end{figure}
		\column{0.45\linewidth}
%\pause
\begin{table}[]
    \footnotesize
    \caption{Summary statistics of math ability}
    \begin{tabular}{llccc}
    \toprule
                 &                & \multicolumn{2}{c}{Sample ($j$)}  \\ \cmidrule(ll){3-4}
    Statistic    &                & Male ($0$)& Female ($1$)   \\ \midrule
    Obs. & $n_j$            & 10,887    & 10,557         \\
    Mean         & $\hat\mu_j$      & 40.17     & 40.20          \\
    Median       &                  & 40.44     & 40.36          \\
    Variance     & $\hat\sigma^2_j$ & 152.00    & 134.74         \\ \bottomrule
    \end{tabular}
\end{table}
\end{columns}
\parbox{1\linewidth}{\vspace{-0.5em}
\scriptsize Source: High School Longitudinal Study (HSLS) of 2009}
%\pause

\vspace{0em}
\begin{itemize}
	\item Assuming each obs is \alert{independent}, should we use $t$-test? WMW test? To test what?
        %\pause
	\item Fay and Proschan (2010) say that the answer depends on your perspective(s).
        %\pause
	\item A \alert{perspective} is a pair of null ($H$) and alternative ($K$) hypotheses.
\end{itemize}

\end{frame}


%------------------------------------------
\begin{frame}\frametitle{One perspective you know from Stats 101}
	\begin{columns}
		\column{0.95\linewidth}
        \\
	\begin{per}[Shift in normal distribution]
	Let $Y$ denote a random variable. The \textbf{shift-in-normal perspective} is
	\vspace{-1em}
	\begin{equation*}\vspace{-1em}
		H:\mathbb{E}_F(Y)=\mathbb{E}_G(Y) \text{\quad versus \quad} K:\mathbb{E}_F(Y) \neq \mathbb{E}_G(Y),
	\end{equation*}
	where $F$ and $G$ are two \alert{normal} distributions with the \alert{same variance}. (Difference must be in means.)
	\end{per}
    %\pause
        \begin{itemize}

        	\item \textbf{Student's $t$-test} {\scriptsize (decision rule)}: Given data $X$ and significance level $\alpha$, reject $H$ if
        	\vspace{1em}
        	\begin{columns}[]
        		\column{0.2\linewidth}
        		\column{0.4\linewidth}
        		\vspace{-1em}
        		\begin{equation*}
        			\left| \frac{\hat\mu_1-\hat\mu_0}{\hat\sigma\sqrt{\frac{1}{n_1}+\frac{1}{n_0}}}
        			\right| > t^{-1}_{n-2}(1-\alpha/2),
        		\end{equation*}
        		\column{0.4\linewidth}\scriptsize
        	 \setstretch{1}where $\hat\sigma^2$ is the pooled sample variance,
	         $n=n_1+n_0$,
	         and $t_d(\cdot)$ is the CDF of Student $t$ distribution with degree of freedom $d$.
        	\end{columns}
        	\vspace{1em}
            %\pause
            \item Under the above, Student's $t$-test is not only \alert{valid} {\scriptsize ($\alpha$ works as intended)} but also \alert{uniformly most powerful (UMP) unbiased}. It's also \alert{asymptotically most powerful (AMP)}.
        \end{itemize}
        \vspace{1em}

	\end{columns}
\end{frame}



%------------------------------------------
\begin{frame}\frametitle{A relaxed perspective, also from Stats 101}
	\begin{columns}
		\column{0.95\linewidth}
		\vspace{-0.5em}
		\begin{per}[Behrens-Fisher]
The \textbf{Behrens-Fisher perspective} is
	\vspace{-1em}
	\begin{equation*}\vspace{-1em}
		H:\mathbb{E}_F(Y)=\mathbb{E}_G(Y) \text{\quad versus \quad} K:\mathbb{E}_F(Y) \neq \mathbb{E}_G(Y),
	\end{equation*}
	where $F$ and $G$ are two \alert{normal} distributions with possibly \alert{different variances}. 
	\end{per}
    %\pause
	\begin{itemize}\vspace{-0.5em}
		\item Under this relaxed perspective, Student's $t$-test is no longer valid because it pools the variances.
		\item \textbf{Welch's $t$-test} uses separate variance estimates, thus is \alert{asymptotically valid} and \alert{asymptotically most powerful}:
		\vspace{-1.5em}
		\begin{equation*}
			\left| \frac{\hat\mu_1 - \hat\mu_0}{\sqrt{\frac{\hat\sigma^2_1}{n_1} + \frac{\hat\sigma^2_0}{n_0}}} \right|
			> t_{d_W}^{-1}(1-\alpha/2),
			\text{\quad\quad where }
			{\scriptsize
			d_W = \frac{\left(
				\frac{\hat\sigma^2_1}{n_1} + \frac{\hat\sigma^2_0}{n_0}
			\right)^2}{\frac{(\hat\sigma^2_1/n_1)^2}{n_1-1} + \frac{(\hat\sigma^2_0/n_0)^2}{n_0-1}}
		}
		\end{equation*}
\end{itemize}
\vspace{-1em}
\thus Each statistical test can have multiple valid perspectives. The authors call this idea the \alert{Multiple perspective decision rules (MPDR) framework}
	\end{columns}
\end{frame}


%------------------------------------------
\begin{frame}\frametitle{Even more relaxed perspective}
	\begin{columns}
		\column{0.95\linewidth}
		\begin{per}[Distributions equal or not]
	\begin{equation*}\vspace{-0.5em}
	H:F=G \text{\quad versus \quad} K:F \neq G,
\end{equation*}
where $F$ and $G$ are any two distributions.
		\end{per}
\begin{itemize}
	\item Under this perspective, the $t$-tests are \alert{asymptotically valid} and the WMW test is \alert{valid}. But neither are \alert{consistent}! {\scriptsize (power approaches 1 as $n\rightarrow \infty$)}
	\item The WMW test {\scriptsize (or Mann-Whitney U test or Wilcoxon rank-sum test)} is to reject if \vspace{-1em}
\begin{equation*}\vspace{-0.5em}\setstretch{0.8}
		\sum_{i=1}^n \sum_{j=1}^m S(Y^0_i, Y^1_j) < U_{n_0,n_1}^{-1}(1-\alpha/2)
		\text{\quad where }
		S(x, y)=
		\begin{cases}
			1, & \text{ if } $x > y$,\\
			\frac{1}{2}, & \text{ if } $x = y$,\\
			0, & \text{ if } $x < y$.
		\end{cases}
\end{equation*}
\item Neither t-tests nor WMW test reject the null hypothesis for the 9th-graders' data
%\item \alert{Kolmogorov–Smirnov test} seems to work, although it's not mentioned in the paper
\end{itemize}
	\end{columns}
\end{frame}


%------------------------------------------
\begin{frame}\frametitle{Philosophy behind the MPDR framework}
	\framesubtitle{Assumptions in scientific research}
	\begin{columns}
		\column{0.95\linewidth}
\begin{itemize}\itemsep1em
\item The \alert{Multiple perspective decision rules (MPDR) framework} has practical value because it suits the nature of scientific theories.\\

\item A \alert{scientific theory} is often a \alert{vague idea} or a \alert{qualitative result} that can be described by more than one statistical model.
\begin{itemize}
	\item In biological sciences, for example, the Physicians' Health Study (PHS) aims to test a theory that says \alert{prolonged low-dose aspirin} decreases \alert{cardiovascular mortality}.
	\item Researchers testing this theory assume a particular statistical model to formulate the null hypothesis, but that model is \alert{just one way} of representing the data's randomness.
\end{itemize}

\item So we should consider the \alert{set of possible statistical assumptions} behind a scientific theory to assess which statistical tests (decision rules) are the most useful.
\end{itemize}
\end{columns}
\end{frame}



%------------------------------------------
% \begin{frame}\frametitle{}
% 	\begin{columns}
% 		\column{0.95\linewidth}
% 		\begin{center}
% 			\Large
% 			\alert{Framework for assessing decision rules}
% 		\end{center}
% 	\end{columns}
% \end{frame}


\newcommand{\lw}{0.25}
\newcommand{\rw}{0.65}
\newcommand{\mar}{0.05}
\newcommand{\vs}{\vspace{1.5em}}

\renewcommand{\leftcol}{0.25\linewidth}
\renewcommand{\maincol}{0.65\linewidth}
\renewcommand{\margin}{0.05\linewidth}
\renewcommand{\myspacing}{\vspace{1.5em}}

\section{Framework}
%------------------------------------------
\begin{frame}\frametitle{Terminology}
	\setstretch{1}
	\begin{columns}[t]
		\column{\margin}
		\column{\leftcol}
\textbf{Data}
		\column{\maincol}
		$X \in \mathcal{X}$, where $\mathcal{X}$ is the sample space. Write $X_n$ to denote number of observations $n$
		\column{\margin}
	\end{columns}
\myspacing
\begin{columns}[t]
	\column{\margin}
	\column{\leftcol}
	\textbf{``Probability model''}
	\column{\maincol}
	A distribution
	$P \in \mathcal{P}$ on $\mathcal{X}$, where $\mathcal{P} = \{P_\theta | \theta \in \Theta\}$ with a given parameter space $\Theta$
	\column{\margin}
\end{columns}
\myspacing\begin{columns}[t]
	\column{\margin}
	\column{\leftcol}
	\textbf{Null hypothesis}
	\column{\maincol}
	$H = \{P_\theta | \theta \in \Theta_H\}$
	\column{\margin}
\end{columns}
\myspacing\begin{columns}[t]
	\column{\margin}
	\column{\leftcol}
	\textbf{\small Alternative hypothesis}
	\column{0.3\linewidth}
	$K = \{P_\theta | \theta \in \Theta_K\}$
	\column{0.4\linewidth}
	\small
	($\Theta_H$ and $\Theta_K$ are \alert{disjoint subsets} of $\Theta$)
\end{columns}
\myspacing\begin{columns}[t]
	\column{\margin}
	\column{\leftcol}
	\textbf{``Assumption''}
	\column{\maincol}
	$A = (\mathcal{X}, H, K)$
	\column{\margin}
\end{columns}
\myspacing\begin{columns}[t]
	\column{\margin}
	\column{\leftcol}
	\textbf{Decision rule {\footnotesize (test)}}
	\column{\maincol}
	$\delta(X, \alpha) \in \{0 \text{\scriptsize\,(not reject)},1 \text{\scriptsize\,(reject)}\}$, for all data $X \in \mathcal{X}$ and \alert{significance level} $\alpha \in (0,0.5)$
	\column{\margin}
\end{columns}
\end{frame}


\renewcommand{\leftcol}{0.15\linewidth}
\renewcommand{\maincol}{0.75\linewidth}
\renewcommand{\margin}{0.05\linewidth}
\renewcommand{\myspacing}{\vspace{1.25em}}
%------------------------------------------
\begin{frame}\frametitle{Terminology about decision rule {\normalsize (test)} $\delta$}
\framesubtitle{for a significance level $\alpha \in (0,0.5)$}
\begin{columns}[t]
	\column{\margin}
	\column{\leftcol}
	\textbf{``Power''}
	\column{0.55\linewidth}
	$ Pow[\delta(X_n,\alpha);\theta] = \Pr[ \delta(X_n,\alpha) = 1; \theta] $
	\column{0.25\linewidth}\raggedleft
{\footnotesize (Probability of rejecting)}
\end{columns}
\myspacing\begin{columns}[t]
	\column{\margin}
	\column{\leftcol}
	\textbf{``Size''}
	\column{0.45\linewidth}
	$\displaystyle \alpha_n^* = \sup_{\theta \in \Theta_H} Pow[\delta(X_n,\alpha);\theta].$
	\column{0.35\linewidth}\raggedleft
{\footnotesize \vspace{1em}(Max. prob. of rejecting given null)}
\end{columns}
\myspacing\begin{columns}[t]
	\column{\margin}
	\column{\leftcol}
	\textbf{Validity}
	\column{\maincol}
	A test $\delta$ is \alert{valid} if $\alpha^*_n \leq \alpha$ for all $n$.\\
		A test $\delta$ is \alert{uniformly asymptotically valid (UAV)} if
		$\displaystyle \limsup_{n \rightarrow \infty} \alpha_n^* \leq \alpha.$\\
	A test $\delta$ is \alert{pointwise asymptotically valid (PAV)} if, for all $\theta \in \Theta_H$,
	\vspace{-1em}
\begin{equation*}
	\limsup_{n \rightarrow \infty} Pow[\delta(X_n,\alpha);\theta] \leq \alpha.
\end{equation*}
	\column{\margin}
\end{columns}
\myspacing\begin{columns}[t]
	\column{\margin}
	\column{\leftcol}
	\textbf{p-value}
	\column{0.45\linewidth}
	$p(X) = \inf\{ \alpha' : \delta(X,\alpha')=1\}$
	\column{0.35\linewidth}\raggedleft
	{\footnotesize (the strictest $\alpha'$ that rejects)}
\end{columns}
\end{frame}

\renewcommand{\leftcol}{0.15\linewidth}
\renewcommand{\maincol}{0.75\linewidth}
\renewcommand{\margin}{0.05\linewidth}
\renewcommand{\myspacing}{\vspace{1em}}
%------------------------------------------
\begin{frame}\frametitle{Terminology about optimal decision rules}
\framesubtitle{for a significance level $\alpha \in (0,0.5)$}


\begin{columns}[t]
	\column{\margin}
	\column{\leftcol}
	\textbf{Bias}
	\column{\maincol}
	A test $\delta$ is \alert{unbiased} if, for all $\theta \in \Theta_K$, power $\geq$ size.
	\column{\margin}
\end{columns}
\myspacing
\begin{columns}[t]
    \column{0.05\linewidth}
	\column{0.15\linewidth}
	\textbf{Consistency}
	\column{0.8\linewidth}
    A test $\delta$ is \alert{consistent} if, for all $\theta \in
    \Theta_K$, the power approaches 1 as $n \rightarrow \infty$.
\end{columns}
\myspacing
\begin{columns}[t]
	\column{\margin}
	\column{\leftcol}
	\textbf{Optimality}
	\column{\maincol}
	\setstretch{1.15}A test $\delta$ is \alert{uniformly most powerful (UMP)} if, $\forall \delta'$ and $\forall \theta \in \Theta_K$,
	\vspace{-0.5em}
	\begin{equation*}
		Pow[\delta(X,\alpha);\theta] \geq Pow[\delta'(X,\alpha);\theta].
	\end{equation*}
	A test is \alert{UMP unbiased} if it is UMP among all unbiased tests.\\
	\vspace{0.5em}
	
	A test is \alert{asymptotically most powerful (AMP)} if, as $\theta_n$ approaches $\theta_0$,
		\vspace{-1em}
	\begin{equation*}\vspace{-1em}
		\limsup_{n\rightarrow \infty} \;	Pow[\delta(X_n,\alpha);\theta_n] - Pow[\delta'(X_n,\alpha);\theta_n] \geq 0
	\end{equation*}
	as $\theta_n \in \Theta_K$ approaches $\theta_0 \in \Theta_H$.
	\column{\margin}
\end{columns}
\end{frame}



\section{Perspectives}
%------------------------------------------
% \begin{frame}\frametitle{}
% 	\begin{columns}
% 		\column{0.95\linewidth}
% 		\begin{center}
% 			\Large
% \alert{Perspectives in the MPDR Framework}
% 		\end{center}
% 	\end{columns}
% \end{frame}


\renewcommand{\myspacing}{\vspace{1em}}
%------------------------------------------
\begin{frame}\frametitle{Perspective 1}
\framesubtitle{Following the original notation}
	\begin{columns}
		\column{0.95\linewidth}
\begin{per}[Difference in means; same null distribution]
\vspace{-1.5em}
\begin{align*}
    H &= \{F, G: F=G\}\\
    K &= \{F, G: \mathbb{E}_F(Y) \neq \mathbb{E}_G(Y)\}
\end{align*}
\end{per}
\begin{itemize}
\item Weird (``\alert{focusing}'') perspective because it leaves out many pairs of distributions
\item Still, the alternative hypotheses $K$ is a pretty large set
\item The WMW test is \alert{valid but inconsistent}
\item The paper doesn't mention how the t-tests fare, but they are likely inconsistent, too.
\item So, don't take this perspective.
\end{itemize}
\end{columns}
\end{frame}



\renewcommand{\myspacing}{\vspace{1em}}
%------------------------------------------
\begin{frame}\frametitle{Perspective 2}
\framesubtitle{A more restrictive one}
	\begin{columns}
		\column{0.95\linewidth}
		\begin{per}[Stochastic ordering]
			Let $\Psi_C$ denote the set of continuous distributions. Write $F <_{st} G$ if $G$ has \alert{first-order stochastic dominance} over $F$ {\scriptsize (i.e. $F(y)\geq G(y)$ for all $y$ and $F(y)> G(y)$ for some $y$)}.
			\vspace{-1em}
			\begin{align*}
				H &= \{F, G: F=G; F \in \Psi_C\}\\
				K &= \{F, G: F<_{st}G \text{ or } G<_{st}F; F,G \in \Psi_C \}
			\end{align*}
		\end{per}
		\begin{itemize}
			\item Under this perspective, the WMW test is \alert{valid} and \alert{consistent} {\scriptsize (Mann and Whitney, 1947)}. It's also \alert{unbiased} {\scriptsize (Lehmann, 1951)}
			\item The t-tests {\scriptsize (both Student's and Welch's)} are \alert{asymptotically valid} and \alert{consistent}
			\item So, both the WMW test and t-tests work under this perspective!
		\end{itemize}
	\end{columns}
\end{frame}

%------------------------------------------
\begin{frame}\frametitle{Perspective 3}
	\begin{columns}
		\column{0.95\linewidth}		\begin{per}[Mann-Whitney Functional]
			Let $Y_F \sim F$ and $Y_G \sim G$.
			Define the \alert{Mann-Whitney functional} $\phi$ as
			\vspace{-1em}
			\begin{equation*}\vspace{-1em}
				\phi(F,G) = \Pr[Y_F > Y_G] + \frac{1}{2}\Pr[Y_F=Y_G]
			\end{equation*}
		The \alert{Mann-Whitney functional perspective} is
		\vspace{-1em}
			\begin{align*}\vspace{-1em}
				H &= \{F, G: F=G; F \in \Psi_C\},\\
				K &= \{F, G: \phi(F,G)\neq\frac{1}{2}; F,G \in \Psi_C \}.
			\end{align*}
		\end{per}
		\begin{itemize}
			\item A natural perspective by construction. Especially appropriate for ordinal data
			\item The WMW test is valid and consistent, whereas the t-tests are inconsistent
			\item So don't use t-tests under this perspective. Use the WMW test
		\end{itemize}
	\end{columns}
\end{frame}




%------------------------------------------
\begin{frame}\frametitle{Perspective 4}
	\framesubtitle{You've seen this before}
	\begin{columns}
		\column{0.95\linewidth}
		\begin{per}[Distribution equal or not]
			\vspace{-1.5em}
		\begin{align*}
			H=\{F,G:F=G\}\\
			K=\{F,G:F\neq G\}
		\end{align*}
		\end{per}
		\begin{itemize}
			\item The WMW test is valid but inconsistent. The t-tests are asymptotically valid but iconsistent.
			\item If you take this perspective, find a different test like Kolmogorov-Smirnov
		\end{itemize}
	\end{columns}
\end{frame}

%------------------------------------------
\begin{frame}\frametitle{Perspectives 5--8: Shifts \& scale in distributions}
	\small
	Let $\Psi_L$, $\Psi_C$, and $\Psi_{LG}$ denote the sets of \alert{logistic}, \alert{continuous}, and \alert{log-gamma} distributions. \\
	Let $\Psi_{D_k}$ denote the set of discrete distributions with sample space $\{1,2,\dots,k\}$
	\begin{columns}
		\column{0.475\linewidth}
		\begin{per}[\small Shift in logistic distribution]
	\vspace{-1.5em}
	\small
	\begin{align*}
		H&=\{F,G: F=G; F \in \Psi_L \}\\
		K&=\{F,G: G(y)=F(y+\Delta); \Delta\neq0; F \in \Psi_L\}
	\end{align*}
\end{per}
		\column{0.475\linewidth}
\begin{per}[\small Shift in continuous distribution]
	\vspace{-1.5em}
	\small
	\begin{align*}
		H&=\{F,G: F=G; F \in \Psi_C \}\\
		K&=\{F,G: G(y)=F(y+\Delta); \Delta\neq0; F \in \Psi_C\}
	\end{align*}
\end{per}
	\end{columns}
		\begin{columns}[t]
		\column{0.475\linewidth}
		\begin{per}[\small Shift in log-gamma distribution]
			\vspace{-1.5em}
			\small
			\begin{align*}
				H&=\{F,G: F=G; F \in \Psi_{LG} \}\\
				K&=\{F,G: G(y)=F(y+\Delta); \Delta\neq0; F \in \Psi_{LG}\}
			\end{align*}
		\end{per}
		\column{0.475\linewidth}
		\begin{per}[\small Proportional odds]
			\vspace{0.5em}
			\small
			$H=\{F,G: F=G; F \in \Psi_{D_k} \}$\\
			\footnotesize
			$K=\{F,G: \frac{F(y)}{1-F(y)}=\frac{G(y)}{1-G(y)}\Delta;\Delta\neq1; F \in \Psi_{D_k}\}$
		\end{per}
	\end{columns}
\begin{itemize}
\item WMW: valid and consistent. t-tests: asymp. valid and consistent.
\end{itemize}
\end{frame}


%------------------------------------------
\begin{frame}\frametitle{Perspective 11: Differences in means assuming normality with same variance}
\framesubtitle{I skip Perspectives 9--10 because they are weird and neither WMW test nor t-tests work}
	\begin{columns}
		\column{0.95\linewidth}
		\begin{per}[Shift in normal distribution]
	\vspace{-1.5em}
	\begin{align*}\vspace{-1.5em}
		H &=\{F,G:F=G; F \in \Psi_N\}\\
		K &=\{F,G:G(y) = F(y+\Delta); \Delta \neq 0; F \in \Psi_N \}
	\end{align*}
	where $\Psi_N$ is the set of normal distributions.
\end{per}
\begin{itemize}
	\item The first perspective you've seen at the beginning.
	\item The WMW test and the Student's t-test are \alert{valid and consistent}. The Student's t-test is \alert{optimal}, because it is \alert{UMP unbiased} and \alert{asymptotically most powerful}. The Welch's t-test is \alert{asymptotically valid and consistent}.
\end{itemize}
	\end{columns}
\end{frame}


%------------------------------------------
\begin{frame}\frametitle{Perspective 14: \large  Differences in means assuming normality with different variance}
	\framesubtitle{Another one you've seen at the beginning}
	\begin{columns}
		\column{0.95\linewidth}
		\begin{per}[Behrens-Fisher: Diference in normal means, different variances]
			\vspace{-1.5em}
			\begin{align*}\vspace{-1.5em}
				H &=\{F,G:\mathbb{E}_F(Y)=\mathbb{E}_G(Y); F,G \in \Psi_N\}\\
				K &=\{F,G:\mathbb{E}_F(Y)\neq\mathbb{E}_G(Y); F,G \in \Psi_N\}
			\end{align*}
			where $\Psi_N$ is the set of normal distributions.
		\end{per}
		\begin{itemize}
			\item Both the WMW test and the Student's t-test are \alert{invalid and inconsistent}
			\item Welch's t-test is \alert{uniformly asymptotically valid} and \alert{consistent}
			\item So, use Welch's t-test if you take this perspective... but better ones exist:
		\end{itemize}
	\end{columns}
\end{frame}



%------------------------------------------
\begin{frame}\frametitle{Perspectives 12--13: Differences in means without assuming normality}
	\begin{columns}
		\small
	\column{0.475\linewidth}
	\begin{per}[Finite variances]
		\vspace{-1.5em}
		\begin{align*}\vspace{-1.5em}
H &=\{F,G:F=G; F \in \Psi_{fv}\}\\
K &=\{F,G:\mathbb{E}_F(Y)\neq\mathbb{E}_G(Y); F,G \in \Psi_{fv}\}
		\end{align*}
		where $\Psi_{fv}$ is the set of distributions with finite variances.
	\end{per}
	\begin{itemize}
		\item The WMW test is \alert{valid but inconsistent}
		\item The t-tests are \alert{pointwise asymptotically valid} and \alert{consistent}
	\end{itemize}
	\column{0.475\linewidth}
\begin{per}[Finite 4th moments]
	\vspace{-1.5em}
	\begin{align*}\vspace{-1.5em}
H &=\{F,G:F=G; F \in \Psi_{B_\epsilon}\}\\
K &=\{F,G:\mathbb{E}_F(Y)\neq\mathbb{E}_G(Y); F,G \in \Psi_{B_\epsilon}\}
	\end{align*}
	where $\Psi_{B_\epsilon}$ is the set of distributions with $Var(Y)\geq \epsilon >0$ and $\mathbb{E}(Y^4)\leq B<\infty$.
\end{per}
	\begin{itemize}
	\item The WMW test is \alert{valid but inconsistent}
	\item The t-tests are \alert{uniformly asymptotically valid} and \alert{consistent}
\end{itemize}
\end{columns}
\vspace{0.5em}
\thus t-tests are clearly preferable in large samples
\end{frame}



%------------------------------------------
\begin{frame}\frametitle{Perspective 15: Seemingly natural but invalid perspective}
	\begin{columns}
		\column{0.95\linewidth}
	\begin{per}[Difference in means; any distributions]
	\vspace{-1.5em}
	\begin{align*}\vspace{-1.5em}
		H &=\{F,G:\mathbb{E}_F(Y) =  \mathbb{E}_G(Y)\}\\
		K &=\{F,G:\mathbb{E}_F(Y)\neq\mathbb{E}_G(Y)\}
	\end{align*}
\end{per}
\begin{itemize}
\item There \alert{exists no valid decision rule} with some power greater than its significant level
\item If you take this loose perspective, nothing works!
\item Your perspective needs more structure
\end{itemize}
	\end{columns}
\end{frame}


%------------------------------------------
\begin{frame}\frametitle{If you want to see the full picture...}
\framesubtitle{The paper has the logical relationship between all perspectives 1--15 along with their validity \& consistency}
	\begin{columns}
		\column{0.95\linewidth}
		\begin{figure}
			\centering
			\includegraphics[width=1\linewidth]{LectureSlides/graphics/perspectives}
			\label{fig:perspectives}
		\end{figure}
	\end{columns}
\end{frame}


\section{Discussion}
%------------------------------------------
\begin{frame}\frametitle{Takeaways}
	\begin{columns}
		\column{1\linewidth}
		\textbf{So... WMW test or t-test?}
		\begin{itemize}
                \item It's important to identify your perspective first! Be \alert{precise}!
                \item $t$-test is usually only asymptotically valid...
			\item In the math ability example, maybe use \alert{Welch's t-test} since $n,m\geq 10,000$
			\item But depending on the application, the \alert{WMW test} may be more appropriate
			\begin{itemize}
			\item For example, if the variable is \alert{ordinal}. Also, the authors argue that the WMW test is often more powerful than the t-tests in \alert{small samples}
			\end{itemize}
			\item In any case, the decision should not depend on whether the data look normally distributed or not, because there are valid perspectives without the normality assumption
            \item But, stay tuned for the permutation test!
		\end{itemize}
	\end{columns}
\end{frame}

\begin{frame}[allowframebreaks]
    \frametitle{References:}
    \small
    \bibliographystyle{plainnat}
    %\bibliographystyle{elsarticle-harv}
    \bibliography{pjbib}
\end{frame}


  % -------------------------------------------
%   \setbeamertemplate{headline}
%   {
% 	  \setbeamercolor{section in head/foot}{fg=black, bg=white}
% 	  \vskip1em \tiny \insertsectionnavigationhorizontal{1\paperwidth}{\hspace{1\paperwidth}}{}
%   }
%   \setbeamertemplate{footline}{}
% \appendix
%   \section{}
%------------------------------------------
\begin{frame}\frametitle{}
	\vspace{5em}
	\begin{columns}
		\column{1\linewidth}
		\centering
		{\Large \alert{The End!}}
	\end{columns}
	\vspace{3em}
	    \begin{columns}
		\column{0.3\linewidth}
		\centering
		\includegraphics[width=1\linewidth]{LectureSlides/graphics/dist.pdf}
	\end{columns}
\end{frame}

\end{document}

