\documentclass[10pt,t,english]{beamer}
\usepackage{hyperref}
\usepackage{fontawesome}
\usepackage{graphicx}
\usepackage{array}
\usepackage[normalem]{ulem}
\usepackage{amsfonts,amsmath,amssymb,bm,bbm}
\usepackage{mathrsfs}
\usepackage{sgame}
\usepackage{graphicx,pstricks}
\usepackage{xcolor}
\usepackage{colortbl}
\usepackage{makecell}
\usepackage{tikz,tikzsymbols,gnuplottex}
\usetikzlibrary{decorations.pathreplacing,shapes}
\usepackage[english]{babel}
\usepackage[utf8]{inputenc}
\usepackage{appendixnumberbeamer}
\usepackage{datetime2}
\usepackage{booktabs}
\usepackage{setspace}
\usepackage{rotating}
\usepackage{natbib}
\usepackage{listings}
\usepackage{matlab-prettifier} % For enhanced MATLAB highlighting
\lstset{
    language=matlab,
    % Other options for styling (optional)
    basicstyle=\ttfamily\footnotesize, % Font style
    keywordstyle=\color{blue}, % Keyword color
    commentstyle=\color{green!50!black}, % Comment color
    stringstyle=\color{red!70!black}, % String color
    numbers=left, % Line numbers on the left
    numberstyle=\tiny\color{gray}, % Style for line numbers
    frame=single, % Frame around the listing
    breaklines=true, % Allow line breaking
    captionpos=b, % Caption at the bottom
    tabsize=4 % Tab size
}

% ------------------------------------------------------------------------------
% Use the beautiful metropolis beamer template
% ------------------------------------------------------------------------------
\usepackage[T1]{fontenc}
\usepackage[utf8]{inputenc}
\usepackage{fontawesome}
\usepackage{FiraSans} 
\mode<presentation>
{
  \usetheme[progressbar=foot,background=light]{metropolis} 
  \usecolortheme{default} % or try albatross, beaver, crane, ...
  \usefonttheme{default}  % or try default, serif, structurebold, ...
  \setbeamertemplate{navigation symbols}{}
  \setbeamertemplate{caption}[numbered]
  %\setbeamertemplate{frame footer}{My custom footer}
} 

\newenvironment{stepenumerate}{\begin{enumerate}[<+->]}{\end{enumerate}}
\newenvironment{stepitemize}{\begin{itemize}[<+->]}{\end{itemize} }
\newenvironment{stepenumeratewithalert}{\begin{enumerate}[<+-| alert@+>]}{\end{enumerate}}
\newenvironment{stepitemizewithalert}{\begin{itemize}[<+-| alert@+>]}{\end{itemize} }

\newtheorem{question}{Question}
\newtheorem{claim}{Claim}
\newtheorem{proposition}{Proposition}
\newtheorem{remark}{Remark}
\newtheorem{conjecture}{Conjecture}

\definecolor{metrop}{RGB}{29, 44, 44}
\colorlet{GrayLight}{black!15}
\colorlet{GrayMedium}{black!30}
\colorlet{ForestGreen}{green!60!black}

\newenvironment{transitionframe}{
  \setbeamercolor{background canvas}{bg=black!80}
  \begin{frame}}{
    \end{frame}
}

\newcommand{\br}{

\bigskip

}

\newcommand{\pd}{\partial}
\newcommand{\RR}{\mathbb{R}}

\newcommand*\hugme[1]{\tikz[baseline=(char.base)]{\node[shape=ellipse,draw,inner sep=0pt] (char) {#1};}}

\newcounter{saveenumi}
\newcommand{\seti}{\setcounter{saveenumi}{\value{enumi}}}
\newcommand{\conti}{\setcounter{enumi}{\value{saveenumi}}}
\resetcounteronoverlays{saveenumi}



\newcommand\dotprod[2]{\langle #1 , #2 \rangle}
\newcommand{\ft}[1]{\widehat #1}
\newcommand{\qabove}[1]{\overset{\text{\large \textbf ?}}{#1}}
\newcommand{\eqae}{\overset{\text{a.e.}}{=}}
\newcommand{\calp}{\mathcal{P}}
\newcommand{\calg}{\mathcal{G}}
\newcommand{\calb}{\mathcal{B}}
\newcommand{\textd}{\text{d}}
\newcommand{\bbr}{\mathbb{R}}
\newcommand{\binm}{\mathbin{M}}
\newcommand{\binc}{\mathbin{C}}
\newcommand{\binb}{\mathbin{B}}
\newcommand{\calc}{\mathcal{C}}
\newcommand{\calh}{\mathcal{H}}
\newcommand{\bfone}{\mathbf{1}}
\newcommand{\bbe}{\mathbb{E}}
\newcommand{\bfle}{\mathbf{e}}
\newcommand{\calf}{\mathcal{F}}
\newcommand{\cala}{\mathcal{A}}
\newcommand{\cale}{\mathcal{E}}
\newcommand{\bbn}{\mathbb{N}}
\newcommand{\cantor}{\calc}
\newcommand{\calY}{\mathcal{Y}}
\newcommand{\textb}{\text{B}}
\newcommand{\calm}{\mathcal{M}}
\newcommand{\bint}{\mathbin{T}}
\newcommand{\ep}{\epsilon}
\newcommand{\bbq}{\mathbb{Q}}
\newcommand{\bbp}{\mathbb{P}}
\newcommand{\cals}{\mathcal{S}}
\newcommand{\emptysequence}{e}
\newcommand{\bbz}{\mathbb{Z}}
\newcommand{\fraka}{\frak{A}}
\newcommand{\frakb}{\frak{B}}
\newcommand{\length}{\text{length}}
\newcommand{\bfn}{\mathbf{N}}
\newcommand{\support}{\text{support}}
\DeclareMathOperator*{\argmax}{arg\,max}
\newcommand{\dom}{\mbox{dom}}
\def\ut{\underline t}
\def\um{\underline m}
\def\PP{\mathbb{P}}
\def\EE{\mathbb{E}}
\def\RR{\mathbb{R}}










\begin{document}

% Title page info
\title[Vernon's Precepts]{ExpEcon Methods:\\Vernon's Precepts}
\author[ECON 8877]{ECON 8877\\P.J. Healy} \color{metrop}
\institute[OSU]{}
\date[]{\vfill {\tiny Updated \today\ at\ \DTMcurrenttime}}

\frame{\maketitle}

\begin{frame}{Vernon Smith}
\begin{columns}
\begin{column}{0.5\textwidth}
    \begin{center}
        \includegraphics[height=2in]{graphics/history/SmithMug.png} 
    \end{center}
\end{column}
\begin{column}{0.5\textwidth}  %%<--- here
    \begin{center}
    \vspace{0.7in}
     Vernon Smith\\
     2002 Nobel Prize
     \end{center}
\end{column}
\end{columns}
\br
\vfill
Other early pioneers: Plott, Kagel, Battalio, Williams... many
\end{frame}

\begin{frame}{Why Vernon?}
The 2002 Nobel Prize
\begin{itemize}
    \item Vernon Smith \& Charlie Plott: pioneered market experiments
    \begin{itemize}
        \item $\rightarrow$ ``experimental economics''
    \end{itemize}
    \item Kahneman \& Tversky: Prospect Theory 
    \begin{itemize}
        \item $\rightarrow$ ``behavioral economics''
        \item Tversky passed away in 1996
    \end{itemize}
\end{itemize}
\br
\begin{itemize}
    \item Smith elucidated the theory of incentivized experiments
    \begin{itemize}
        \item ``Experimental Economics: Induced Value Theory'' (1976)
        \item ``Microeconomic Systems as an Experimental Science'' (1982)
    \end{itemize}
\end{itemize}    
\end{frame}

\begin{frame}{Induced Value Theory (1976)}
    Experiments are important because
    \begin{enumerate}
        \item They are a \textit{pretest} of economic theory
        \begin{itemize}
            \item \emph{Prior} to the use of field data
            \item Using field data to modify models is flawed since it's in-sample. ``Any test of significance now becomes hopelessly confused''
            \begin{itemize}
                \item Discuss: Do you agree?
            \end{itemize}
            \item Model $\rightarrow$ expm'nt $\rightarrow$ new model $\rightarrow$ expm'nt $\rightarrow$ new model $\rightarrow$ ...
        \end{itemize}
        \item (Presumed) parallelism between lab and field
        \begin{itemize}
            \item Physics: ``As far as we can tell, the same physical laws prevail everywhere.''
            \item Smith: ``The laboratory becomes a place where real people earn real money for making real decisions about abstract claims that are just as ``real'' as a share of General Motors.
            \begin{itemize}
                \item Discuss: Do you agree?
            \end{itemize}
        \end{itemize}
    \end{enumerate}
\end{frame}

\begin{frame}{Micro Systems as Experimental Science (1982)}
    The precepts of induced value theory:
    \begin{enumerate}
        \item ``Non-satiation'' (monotonically increasing utility)
        \begin{itemize}
            \item Monetary reward: $M(a)$ for action $a$
            \item Utility for money: $u(M)$, $u'>0$
            \item $\arg\max_a u(M(a)) = \arg\max_a M(a)$
            \item If multiple actions (eg, $a=(x,y)$), MRS is the same:
            $$
            \frac{\pd u(M(x,y))/\pd x}{\pd u(M(x,y))/\pd y} = \frac{u'}{u'} \frac{\pd M/\pd x}{\pd M/\pd y} = \frac{\pd M/\pd x}{\pd M/\pd y}
            $$
            \item Assumes choice is costless
            \item Assumes selfish money maximization
            \begin{itemize}
                \item Vernon was really just focused on market expmnts
                \item Vernon is a libertarian...
            \end{itemize}
        \end{itemize}
        \seti
    \end{enumerate}
\end{frame}

\begin{frame}{Micro Systems as Experimental Science (1982)}
    The precepts of induced value theory:
    \begin{enumerate}
        \conti
        \item ``Saliency'' (actions map to rewards)
        \begin{itemize}
            \item Action profiles $a$ map into rewards $M(a)$
            \item This mapping is known and understood
            \item Example: If you win an auction, you earn $v_i$
            \item Example: Show-up fee is not salient
            \begin{itemize}
                \item Warning: Not the same as Shliefer \textit{et al.} notion of salience
            \end{itemize}
        \end{itemize}
        \seti
    \end{enumerate}
\end{frame}

\begin{frame}{Micro Systems as Experimental Science (1982)}
    The precepts of induced value theory:
    \begin{enumerate}
        \conti
        \item ``Dominance'' (sufficiently large rewards)
        \begin{itemize}
            \item The reward structure dominates any subjective costs or values
            \item Example: Cognitive costs, effort
            \item Example: Paying a commission for each transaction\\
            to offset effort costs
            \item Discussion: Do we want to rule out other-regarding preferences?
        \end{itemize}
        \seti
    \end{enumerate}
\end{frame}

\begin{frame}{Micro Systems as Experimental Science (1982)}
    The precepts of induced value theory:
    \begin{enumerate}
        \conti
        \item ``Privacy'' (only know your own payoff)
        \begin{itemize}
            \item Subjects can only see their own payoffs
            \item Goal: own-reward maximizers
            \item Removes social preferences
            \item Removes ``tournament incentives'' (desire to come in first)
            \item Discussion: social preferences?
        \end{itemize}
        \seti
    \end{enumerate}
\end{frame}

\begin{frame}{Micro Systems as Experimental Science (1982)}
    The precepts of induced value theory:
    \begin{enumerate}
        \conti
        \item ``Parallelism'' (aka ``external validity'')
        \begin{itemize}
            \item Lab results apply to non-lab settings
            \item Not required if only testing theories in the abstract
            \item ``the same physical laws prevail everywhere'' (Harlow Shapley 1964)
            \item How to test parallelism? Comparison studies
            \item Who has the burden of proof?
            \item My view (following Roth):
            \begin{itemize}
                \item Experiments provide provocative examples
                \item We can't guarantee parallelism, but our results are at least worth considering in the field
            \end{itemize}
        \end{itemize}
        \seti
    \end{enumerate}
\end{frame}

\begin{frame}{Micro Systems as Experimental Science (1982)}
    What else this paper does:
    \begin{itemize}
        \item Experiments as game forms $\Rightarrow$ mechanism design
        \begin{itemize}
            \item Mount-Reiter diagram (Fig. 1)
        \end{itemize}
        \item Classifications of experiments
        \begin{itemize}
            \item Vary environment vs. vary institution
            \item Methodological:
            \begin{enumerate}
                \item Establishing laws of behavior
                \item Heuristic/exploratory experiments
                \item Boundary/extreme experiments
            \end{enumerate}
        \end{itemize}
        \item List of ``stylized facts'' (robust results)
    \end{itemize}
\end{frame}

\end{document}