\documentclass[10pt,t,english]{beamer}
\usepackage{fontawesome}
\usepackage{graphicx}
\usepackage{array}
\usepackage[normalem]{ulem}
\usepackage{amsfonts,amsmath,amssymb,bm,bbm}
\usepackage{mathrsfs}
\usepackage{sgame}
\usepackage{graphicx,pstricks}
\usepackage{xcolor}
\usepackage{colortbl}
\usepackage{makecell}
\usepackage{tikz,tikzsymbols,gnuplottex}
\usetikzlibrary{decorations.pathreplacing,shapes}
\usepackage[english]{babel}
\usepackage[utf8]{inputenc}
\usepackage{appendixnumberbeamer}
\usepackage{datetime2}
\usepackage{booktabs}
\usepackage{setspace}
\usepackage{rotating}
\usepackage{natbib}
\usepackage{listings}

%\usepackage{matlab-prettifier} % For enhanced MATLAB highlighting
\lstset{
    language=matlab,
    % Other options for styling (optional)
    basicstyle=\ttfamily\footnotesize, % Font style
    keywordstyle=\color{blue}, % Keyword color
    commentstyle=\color{green!50!black}, % Comment color
    stringstyle=\color{red!70!black}, % String color
    numbers=left, % Line numbers on the left
    numberstyle=\tiny\color{gray}, % Style for line numbers
    frame=single, % Frame around the listing
    breaklines=true, % Allow line breaking
    captionpos=b, % Caption at the bottom
    tabsize=4 % Tab size
}

% ------------------------------------------------------------------------------
% Use the beautiful metropolis beamer template
% ------------------------------------------------------------------------------
\usepackage[T1]{fontenc}
\usepackage[utf8]{inputenc}
\usepackage{fontawesome}
\usepackage{FiraSans} 
\mode<presentation>
{
  \usetheme[progressbar=foot,background=light]{metropolis} 
  \usecolortheme{default} % or try albatross, beaver, crane, ...
  \usefonttheme{default}  % or try default, serif, structurebold, ...
  \setbeamertemplate{navigation symbols}{}
  \setbeamertemplate{caption}[numbered]
  %\setbeamertemplate{frame footer}{My custom footer}
} 

\newenvironment{stepenumerate}{\begin{enumerate}[<+->]}{\end{enumerate}}
\newenvironment{stepitemize}{\begin{itemize}[<+->]}{\end{itemize} }
\newenvironment{stepenumeratewithalert}{\begin{enumerate}[<+-| alert@+>]}{\end{enumerate}}
\newenvironment{stepitemizewithalert}{\begin{itemize}[<+-| alert@+>]}{\end{itemize} }

\newtheorem{question}{Question}
\newtheorem{claim}{Claim}
\newtheorem{proposition}{Proposition}
\newtheorem{remark}{Remark}
\newtheorem{conjecture}{Conjecture}

\definecolor{metrop}{RGB}{29, 44, 44}
\colorlet{GrayLight}{black!15}
\colorlet{GrayMedium}{black!30}
\colorlet{ForestGreen}{green!60!black}

\newenvironment{transitionframe}{
  \setbeamercolor{background canvas}{bg=black!80}
  \begin{frame}}{
    \end{frame}
}

\newcommand{\br}{

\bigskip

}

\newcommand{\pd}{\partial}
\newcommand{\RR}{\mathbb{R}}

\newcommand*\hugme[1]{\tikz[baseline=(char.base)]{\node[shape=ellipse,draw,inner sep=0pt] (char) {#1};}}

\newcounter{saveenumi}
\newcommand{\seti}{\setcounter{saveenumi}{\value{enumi}}}
\newcommand{\conti}{\setcounter{enumi}{\value{saveenumi}}}
\resetcounteronoverlays{saveenumi}



\newcommand\dotprod[2]{\langle #1 , #2 \rangle}
\newcommand{\ft}[1]{\widehat #1}
\newcommand{\qabove}[1]{\overset{\text{\large \textbf ?}}{#1}}
\newcommand{\eqae}{\overset{\text{a.e.}}{=}}
\newcommand{\calp}{\mathcal{P}}
\newcommand{\calg}{\mathcal{G}}
\newcommand{\calb}{\mathcal{B}}
\newcommand{\textd}{\text{d}}
\newcommand{\bbr}{\mathbb{R}}
\newcommand{\binm}{\mathbin{M}}
\newcommand{\binc}{\mathbin{C}}
\newcommand{\binb}{\mathbin{B}}
\newcommand{\calc}{\mathcal{C}}
\newcommand{\calh}{\mathcal{H}}
\newcommand{\bfone}{\mathbf{1}}
\newcommand{\bbe}{\mathbb{E}}
\newcommand{\bfle}{\mathbf{e}}
\newcommand{\calf}{\mathcal{F}}
\newcommand{\cala}{\mathcal{A}}
\newcommand{\cale}{\mathcal{E}}
\newcommand{\bbn}{\mathbb{N}}
\newcommand{\cantor}{\calc}
\newcommand{\calY}{\mathcal{Y}}
\newcommand{\textb}{\text{B}}
\newcommand{\calm}{\mathcal{M}}
\newcommand{\bint}{\mathbin{T}}
\newcommand{\ep}{\epsilon}
\newcommand{\bbq}{\mathbb{Q}}
\newcommand{\bbp}{\mathbb{P}}
\newcommand{\cals}{\mathcal{S}}
\newcommand{\emptysequence}{e}
\newcommand{\bbz}{\mathbb{Z}}
\newcommand{\fraka}{\frak{A}}
\newcommand{\frakb}{\frak{B}}
\newcommand{\length}{\text{length}}
\newcommand{\bfn}{\mathbf{N}}
\newcommand{\support}{\text{support}}
\DeclareMathOperator*{\argmax}{arg\,max}
\newcommand{\dom}{\mbox{dom}}
\def\ut{\underline t}
\def\um{\underline m}
\def\PP{\mathbb{P}}
\def\EE{\mathbb{E}}
\def\RR{\mathbb{R}}

\begin{document}

% Title page info
\title[MLE]{ExpEcon Methods:\\
MLE, Finite Mixture Models, \& Model Selection}
\author[ECON 8877]{ECON 8877\\P.J. Healy\\First version thanks to Hyoeun Park} \color{metrop}
\institute[OSU]{}
\date[]{\vfill {\tiny Updated \today}}

\frame{\maketitle}

\begin{frame}{Contents}
\begin{enumerate}
\item Model estimation via MLE: how to code it Finite mixture models
\item Model selection: Cross-validation vs. BIC vs. AIC
\end{enumerate}
\end{frame}

\begin{frame}{Maximum Likelihood Estimation}
Likelihood function\\
\begin{itemize}
    \item $y$: random variable, $\pmb{\theta}$: set of parameters
    \item $f(\pmb{y}|\pmb{\theta})$: pdf, $\theta$ identifies possible DGPs (true models)
    \item The joint density of $n$ i.i.d. observations from this process
    \begin{equation*}
    f(y_1 \dots y_n|\pmb{\theta})=\prod_{i=1}^n f(y_i|\pmb{\theta})\onslide<2->{=L(\pmb{\theta}|\pmb{y})}
    \end{equation*}
    \item \onslide<2->{$L(\pmb{\theta}|\pmb{y})$: function of the unknown parameter vector, $\pmb{\theta}$, \\
    \hspace{1.2cm}given observed data $\pmb{y}$}
\end{itemize}
\end{frame}

\begin{frame}{Maximum Likelihood Estimation}
Example: Behavioral game theory models
\begin{itemize}
    \item 2-player, 3$\times$3 game
    \item $S_i$ set of strategies, $s_i \in S_i$
    \item $\sigma_i(s_i)$: $i$'s probability of playing $s_i$ in mixed strategy $\sigma_i(\cdot)$
    \item $u_i(\sigma_1, \sigma_2)=\sum_{(s_1,s_2)}\sigma_1(s_1)\sigma_2(s_2)u_i(s_1,s_2)$
    \item Some models use deterministic best response:
    $$BR_i(\sigma_j)=\arg\max_{s_i}u_i(s_i,\sigma_j)$$
    (assume unique for simplicity here)
    \item Others assume noisy behavior, like logistic response:
    \[
    LR_i(\sigma_j|\lambda_i)(s_i) = \frac{\exp(\lambda u_i(s_i,\sigma_j))}{\sum_{s_i' \in S_i}\exp(\lambda u_i(s_i',\sigma_j))}
    \]
\end{itemize}
\end{frame}

\begin{frame}{Maximum Likelihood Estimation}
Model 1: Level-$k$ with logistic trembles
\begin{itemize}
    \item Observed data: a played strategy
    \item 3 parameters:\\ $k$ (hierarchy level), $\epsilon$ (prob. tremble), $\lambda$ (precision parameter)
    \item Base model ($k$): $\sigma_i^{LK}(\cdot|k=0)=U[S_i]$ is uniform, then $\forall k > 0$
    $$\sigma_i^{LK}(s_i|k)=1 \text{ iff } s_i = BR_i\left(\sigma_j^{LK}(\cdot|k-1)\right)$$
    \item Problem: deterministic model. Zero likelihood possible.
    \item Solution ($\epsilon,\lambda$): modify the model with logistic trembles:
        \begin{equation*}
        f^{LK}(s_i|\epsilon,\lambda,k)=
        (1-\epsilon)\,\mathbbm{1}_{\left\{s_i = BR_i\left(\sigma_j^{LK}(\cdot|k-1)\right)\right\}} + \epsilon \cdot LR_i(\sigma_j^{LK}(\cdot|(k-1))|\lambda)(s_i)
        \end{equation*}
\end{itemize}

\end{frame}


\begin{frame}{Maximum Likelihood Estimation}
Example\\
QRE
\begin{itemize}
    \item One parameter: $\lambda$
    \item Model: defined by fixed point. For each $i$,
        \begin{align*}
        \sigma_i^{QRE}(s_i|\lambda) &=
        LR_i(\sigma_j^{QRE}(\cdot|\lambda)|\lambda,S_i)(s_i)\\
        f^{QRE}(s_i|\lambda) &= \sigma_i^{QRE}(s_i|\lambda)
        \end{align*}
\end{itemize}

\bigskip
Data: $s_i^g$ for games $g\in \mathbf{G}=\{1,2,3 \dots ,G\}$. 
\begin{equation*}
f^{LK}(\pmb{s}|\epsilon,\lambda,k)=\prod_{g\in \mathbf{G}}f^{LK}(s_i^g|\epsilon,\lambda,k) = L^{LK}(\epsilon,\lambda,k|\pmb {s})
\end{equation*}
and
\begin{equation*}
f^{QRE}(\pmb{s}|\lambda)=\prod_{g\in \mathbf{G}}f^{QRE}(s_i^g|\lambda) =L^{QRE}(\lambda|\pmb {s})
\end{equation*}
\end{frame}




\begin{frame}{Maximum Likelihood Estimation}
Maximum Likelihood Estimation of a single model (eg, noisy LK)
\begin{enumerate}
    \item Let's allow different parameters for each subject. So, fix $i$
    \item Set a 3D grid of $(\epsilon,\lambda,k)$ values
    \item For each point on that grid calculate
    \begin{align*}
        f^{LK}(\mathbf{s}|\epsilon,\lambda,k) &= \prod_{g\in \mathbf{G}}f^{LK}(s_i^g|\epsilon,\lambda,k) \\
        &= \sum_{g\in \mathbf{G}}log(f^{LK}(s_i^g|\epsilon,\lambda,k))
    \end{align*}
    \item MLE estimate: $(\epsilon^*,\lambda^*,k^*)$ with highest value
\end{enumerate}
Model Selection:
\begin{itemize}
    \item Which model has higher likelihood at its MLE parameter?
    \item Problem: unfair advantage having more parameters
    \item Solutions: penalize MLE by subtracting param. penalty
    \begin{itemize}
        \item Akaike Info Criterion (AIC) vs. Bayesian Info Criterion (BIC)
    \end{itemize}
\end{itemize}
\end{frame}


\begin{frame}{Maximum Likelihood Estimation}
Example
\begin{figure}
    \centering
\includegraphics[width=0.9\linewidth]{LectureSlides/graphics/mle/games.png}
\end{figure}
\small{
\begin{itemize}
    \item Suppose that a subject plays M, T, B
    \item Let $\lambda \in \{0.01,0.05,1 \}$
    \item $log(L^{QRE}(0.01|(M,T,B)))=log(0.3628)+log(0.3914)+log(0.3311)$
    \item $log(L^{QRE}(0.05|(M,T,B)))=log(0.5391)+log(0.5355)+log(0.3518)$
    \item $log(L^{QRE}(1|(M,T,B)))=log(1)+log(0.7024)+log(0.9999)$
    \item Thus, in this example $\hat \lambda =1$
\end{itemize}
    }
\end{frame}


\begin{frame}{Finite Mixture Model}
\begin{itemize}
    \item So far, only one model for one likelihood function \pause
    \item Is it a valid approach?
    \begin{itemize}
    \pause
        \item For example, Georganas et al., (2015) show that a cognitive hierarchy is not persistent across classes of games
        \item Suggests that estimating with \textit{one} hierarchy cannot be valid.
        \pause
        \item Another example at the population level, people might have different risk preferences, etc.,
    \end{itemize}
\end{itemize}


\end{frame}


\begin{frame}{Finite Mixture Model}
``Mix" models
\begin{itemize}
    \item $m=1,2, \dots, M$ denotes model
    \item $f(\pmb{y}|\pmb{\psi})=\sum_{m=1}^M \pi_m f_m(\pmb{y}|\pmb{\theta_m})$, where $\pmb{\psi}=(\{\pmb{\theta_m}\}_{m=1}^M,\pi_1,\pi_2,\dots,\pi_m)$
    \item Usually, $f_m(\pmb{y},\pmb{\theta_m})$ (called component density) are taken to belong to the same parametric family.
    \begin{itemize}
        \item There are special cases where component densities are taken to be different (nonstandard mixture)
    \end{itemize}
    \item posterior probability that data is drawn from model $m$, given observed data $\pmb{y}$ is \hspace{0.2cm}$\pi_m \cdot \frac{f_m(\pmb{y}|\pmb{\theta}_m)}{f(\pmb{y}|\pmb{\psi})}$
    \item A parametric family of densities is primitive. Each component has distinct values
    \item Using MLE to fitting mixture distributions $\pi$ (Most commonly used way)
\end{itemize}
\end{frame}

\begin{frame}{Model Selection}
Going back to Level-$k$ and QRE Example

\begin{itemize}
    \item Suppose that an experimenter wants to compare which model better explains data
    \item They can horse-race models
    \item Using MLE?
    \begin{itemize}
        \item For level-$k$, find the ML estimates $(\hat{\epsilon}, \hat{\lambda}, \hat{k})$\\
        plug in those values to the level-$k$ model's likelihood function
        \item For QRE, find the ML estimate $\hat{\lambda}$\\plug in $\hat\lambda$ to the QRE's likelihood function
        \item Compare the likelihood values of two models and pick the model that gives the higher value
    \end{itemize}
\end{itemize}

\end{frame}

\begin{frame}{Model Selection}
    What is the problem with ML approach?
   \begin{itemize}
       \item level-$k$ has three parameters, while QRE has only one parameter
       \item level-$k$ has more ``flexibility"
       \item Consider a weird model with $\infty$ numbers of parameters
       \begin{itemize}
           \item $\infty$ flexibility
           \item Can explain any behavior in the data
           \item Then, this model ``wins" just because it has more flexibility, not because it is true DGP.
       \end{itemize}
       \item Need for fixing the problem of over-fitting due to large \# of parameters
   \end{itemize}
\end{frame}

\begin{frame}{Model Selection}
    How to penalize over-fitting due to large numbers of parameters?
    \begin{itemize}
        \item AIC (Akaike information criterion)
        \item BIC (Bayesian information criterion)
        \item Cross-Validation
    \end{itemize}
\end{frame}

\begin{frame}{Model Selection: AIC, BIC}
\begin{itemize}
    \item AIC =$2k\ln(n)-2\ln({\widehat {L}})$
    \item BIC=$k\ln(n)-2\ln({\widehat {L}})$
\end{itemize}
where
\begin{itemize}
    \item $\widehat{L}$=the maximized value of the likelihood function of the model
    \begin{itemize}
        \item From observed data, get ML estimates, and plug into the ML function
    \end{itemize}
    \item $n$= number of observed data point
    \begin{itemize}
        \item In our example, the number of games subjects played
    \end{itemize}
    \item $k$ = number of parameters
    \begin{itemize}
        \item In level-$k$ model, $k$=3 \\ In QRE, $k=1$
    \end{itemize}
\end{itemize}
\end{frame}

\begin{frame}{Model Selection: AIC, BIC}
\begin{itemize}
    \item AIC =$2k\ln(n)-2\ln({\widehat {L}})$
    \item BIC=$k\ln(n)-2\ln({\widehat {L}})$
\end{itemize}

\bigskip
\begin{itemize}
    \item The preferred model is the one with the minimum AIC/BIC value
    \item The second term gives benefits to the model with goodness-of-fit
    \item The first term gives a penalty to the number of parameters
    \item Those can be only used for linear-models
    \item Can be used only when $n>>k$
\end{itemize}
\end{frame}

\begin{frame}{Model Selection: Cross-Validation}
\begin{itemize}
    \item $k$-fold cross-validations
    \item Divide data into $k$ sub-samples
    \begin{itemize}
        \item For example, 12 data points, 4 sub-samples that include 3 data points each
    \end{itemize}
    \item $k-1$ sub-samples = training data
    \begin{itemize}
        \item Fit the data to a model (MLE, MSE ...)
    \end{itemize}
    \item one sub-sample = testing data
    \begin{itemize}
        \item Using the fitted parameters from training data, test the model i.e., Plug in the estimated parameter to the goodness-of-fit function used for training (MLE, MSE, ...)
    \end{itemize}
    \item Repeat this for $K$ times
    \item Extreme case of $K$o fold cross-validation is leave-one-out cross-validation that $K=n$, where $n$= number of data points
\end{itemize}
\end{frame}

\begin{frame}{Model Selection: Cross-Validation}
    How does Cross-Validation penalize the number of parameters?

    \bigskip
    Consider the following example..
    \begin{itemize}
        \item Suppose that subjects played four games
        \item For three games, a subject's choices coincide with the level-$k$ level-1's predictions
        \item Then $\hat \epsilon = 0$ and log-likelihood function value is 0.
        \item Suppose that the subject did not play level-1's predicted strategy.
        \item Then for the testing data (fourth game), $\hat \epsilon=0$ results in $-\infty$
        \item For QRE, less likely to over-fit (since it has only one parameter). Less likely to have $-\infty$ for testing data
    \end{itemize}
\end{frame}
\begin{frame}{Model Selection: Cross-Validation}
    \begin{itemize}
        \item Cross-Validation penalizes the number of parameters internally.
        \item Over-fitting due to a higher number of parameters penalizes deviation from the prediction a lot in the testing data
        \item No restriction for models being tested; does not have to be linear
    \end{itemize}
\end{frame}

\begin{frame}{Model Selection}
    \begin{itemize}
        \item SO, is Cross-Validation a perfect solution for model selection?
        \pause
        \item NO! \pause
        \item The penalization can be ``too" severe \pause
        \item See Healy \& Park (2023) for suggestions :)
    \end{itemize}
\end{frame}



\end{document}