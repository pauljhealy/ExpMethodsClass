\documentclass[10pt,t,english]{beamer}
\usepackage{fontawesome}
\usepackage{graphicx}
\usepackage{array}
\usepackage[normalem]{ulem}
\usepackage{amsfonts,amsmath,amssymb,bm,bbm}
\usepackage{mathrsfs}
\usepackage{sgame}
\usepackage{graphicx,pstricks}
\usepackage{xcolor}
\usepackage{colortbl}
\usepackage{makecell}
\usepackage{tikz,tikzsymbols,gnuplottex}
\usetikzlibrary{decorations.pathreplacing,shapes}
\usepackage[english]{babel}
\usepackage[utf8]{inputenc}
\usepackage{appendixnumberbeamer}
\usepackage{datetime2}
\usepackage{booktabs}
\usepackage{setspace}
\usepackage{rotating}
\usepackage{natbib}
\usepackage{listings}

%\usepackage{matlab-prettifier} % For enhanced MATLAB highlighting
\lstset{
    language=matlab,
    % Other options for styling (optional)
    basicstyle=\ttfamily\footnotesize, % Font style
    keywordstyle=\color{blue}, % Keyword color
    commentstyle=\color{green!50!black}, % Comment color
    stringstyle=\color{red!70!black}, % String color
    numbers=left, % Line numbers on the left
    numberstyle=\tiny\color{gray}, % Style for line numbers
    frame=single, % Frame around the listing
    breaklines=true, % Allow line breaking
    captionpos=b, % Caption at the bottom
    tabsize=4 % Tab size
}

% ------------------------------------------------------------------------------
% Use the beautiful metropolis beamer template
% ------------------------------------------------------------------------------
\usepackage[T1]{fontenc}
\usepackage[utf8]{inputenc}
\usepackage{fontawesome}
\usepackage{FiraSans} 
\mode<presentation>
{
  \usetheme[progressbar=foot,background=light]{metropolis} 
  \usecolortheme{default} % or try albatross, beaver, crane, ...
  \usefonttheme{default}  % or try default, serif, structurebold, ...
  \setbeamertemplate{navigation symbols}{}
  \setbeamertemplate{caption}[numbered]
  %\setbeamertemplate{frame footer}{My custom footer}
} 

\newenvironment{stepenumerate}{\begin{enumerate}[<+->]}{\end{enumerate}}
\newenvironment{stepitemize}{\begin{itemize}[<+->]}{\end{itemize} }
\newenvironment{stepenumeratewithalert}{\begin{enumerate}[<+-| alert@+>]}{\end{enumerate}}
\newenvironment{stepitemizewithalert}{\begin{itemize}[<+-| alert@+>]}{\end{itemize} }

\newtheorem{question}{Question}
\newtheorem{claim}{Claim}
\newtheorem{proposition}{Proposition}
\newtheorem{remark}{Remark}
\newtheorem{conjecture}{Conjecture}

\definecolor{metrop}{RGB}{29, 44, 44}
\colorlet{GrayLight}{black!15}
\colorlet{GrayMedium}{black!30}
\colorlet{ForestGreen}{green!60!black}

\newenvironment{transitionframe}{
  \setbeamercolor{background canvas}{bg=black!80}
  \begin{frame}}{
    \end{frame}
}

\newcommand{\br}{

\bigskip

}

\newcommand{\pd}{\partial}
\newcommand{\RR}{\mathbb{R}}

\newcommand*\hugme[1]{\tikz[baseline=(char.base)]{\node[shape=ellipse,draw,inner sep=0pt] (char) {#1};}}

\newcounter{saveenumi}
\newcommand{\seti}{\setcounter{saveenumi}{\value{enumi}}}
\newcommand{\conti}{\setcounter{enumi}{\value{saveenumi}}}
\resetcounteronoverlays{saveenumi}



\newcommand\dotprod[2]{\langle #1 , #2 \rangle}
\newcommand{\ft}[1]{\widehat #1}
\newcommand{\qabove}[1]{\overset{\text{\large \textbf ?}}{#1}}
\newcommand{\eqae}{\overset{\text{a.e.}}{=}}
\newcommand{\calp}{\mathcal{P}}
\newcommand{\calg}{\mathcal{G}}
\newcommand{\calb}{\mathcal{B}}
\newcommand{\textd}{\text{d}}
\newcommand{\bbr}{\mathbb{R}}
\newcommand{\binm}{\mathbin{M}}
\newcommand{\binc}{\mathbin{C}}
\newcommand{\binb}{\mathbin{B}}
\newcommand{\calc}{\mathcal{C}}
\newcommand{\calh}{\mathcal{H}}
\newcommand{\bfone}{\mathbf{1}}
\newcommand{\bbe}{\mathbb{E}}
\newcommand{\bfle}{\mathbf{e}}
\newcommand{\calf}{\mathcal{F}}
\newcommand{\cala}{\mathcal{A}}
\newcommand{\cale}{\mathcal{E}}
\newcommand{\bbn}{\mathbb{N}}
\newcommand{\cantor}{\calc}
\newcommand{\calY}{\mathcal{Y}}
\newcommand{\textb}{\text{B}}
\newcommand{\calm}{\mathcal{M}}
\newcommand{\bint}{\mathbin{T}}
\newcommand{\ep}{\epsilon}
\newcommand{\bbq}{\mathbb{Q}}
\newcommand{\bbp}{\mathbb{P}}
\newcommand{\cals}{\mathcal{S}}
\newcommand{\emptysequence}{e}
\newcommand{\bbz}{\mathbb{Z}}
\newcommand{\fraka}{\frak{A}}
\newcommand{\frakb}{\frak{B}}
\newcommand{\length}{\text{length}}
\newcommand{\bfn}{\mathbf{N}}
\newcommand{\support}{\text{support}}
\DeclareMathOperator*{\argmax}{arg\,max}
\newcommand{\dom}{\mbox{dom}}
\def\ut{\underline t}
\def\um{\underline m}
\def\PP{\mathbb{P}}
\def\EE{\mathbb{E}}
\def\RR{\mathbb{R}}
\usepackage{natbib}
\begin{document}

% Title page info
\title[Incentives: History]{ExpEcon Methods:\\Why Incentivize??}
\author[ECON 8877]{ECON 8877\\P.J. Healy} \color{metrop}
\institute[OSU]{}
\date[]{\vfill {\tiny Updated \today\ at\ \DTMcurrenttime}}

\frame{\maketitle}

\begin{frame}{Why Pay?}
    Is there \emph{really} a reason to pay subjects?
    \begin{itemize}
        \item I \emph{still} get asked this pretty frequently.
        \item Ned Augenblick: ``Why are we fetishizing incentives?''
        \item Danz Vesterlund Wilson (2022): Pay but don't explain
        \item Enke-Graeber: Unincentivized measure of decision confidence
    \end{itemize}
    \br
    What do the data say?? This should be an empirical question...
\end{frame}

\begin{frame}{The Effect of Incentives}
\citet{CamererHogarth1999} remains the classic reference\\
They compare hypothetical, low, high payments.
\begin{enumerate}
    \item Modal results don't change
    \item $\uparrow$ payments reduce noise
    \item $\uparrow$ payments induce more effort, performance
    \item $\uparrow$ payments reduce desirability bias (generosity, risk-seeking)
    \item Cognitive capital and costs are important, too
    \item Rationality violations still persist with $\uparrow$ payments
\end{enumerate}
\end{frame}

\begin{frame}{The Effect of Incentives}
\citet{GneezyRustichini2000} test various payment levels
\begin{enumerate}
    \item IQ task
    \begin{itemize}
        \item U-shaped performance. ``Pay enough or not at all''
    \end{itemize}
    \item Hire HS students to soliciting money for charity
    \begin{itemize}
        \item U-shaped performance
        \item No pay > high pay > low pay
    \end{itemize}
\end{enumerate}
Are these tasks similar to typical experiment tasks?
\end{frame}

\begin{frame}{The Effect of Incentives}
\citet{Branas-GarzaEtAl2021}: donate $x$\% of your lottery winnings.\\
High stakes $\uparrow$ total giving, but $\downarrow$ fraction, $\downarrow$ 100\% giving
\br
Ultimatum game:
\begin{itemize}
    \item \citet{SlonimRoth1998}
    \item \citet{AndersenEtAl2011}
\end{itemize}
\end{frame}

\begin{frame}{Hypothetical Incentives \& Beliefs}
    Why pay for beliefs? The mechanisms are complex \citet{DanzEtAl2022}\\
    Arguments in favor:
    \begin{enumerate}
        \item Induces subjects to take time to report truthfully
        \item Might improve beliefs if belief formation is costly
        \begin{itemize}
            \item But do we want that?? Discuss.
        \end{itemize}
        \item Smith's dominance \citep{Wilde1981,Smith1982}
        \begin{itemize}
            \item Stated beliefs used to justify selfish behavior \citep{BlancoEtAl2010}
            \item Wanting to appear more confident than they are
            \item Example: salesperson
        \end{itemize}
    \end{enumerate}
\end{frame}

\begin{frame}{Hypothetical Incentives \& Beliefs}
    Arguments against paying:
    \begin{enumerate}
        \item Not needed; people don't like to lie \citep{Gneezy2005,FischbacherFollmi-Heusi2013}
        \item Mechanism not IC for actual people
        \begin{itemize}
            \item Complex mechanism w/ flat maximum can crowd out intrinsic motive to report truthfully.
            \item \citet{DanzEtAl2022}: calculator screws up responses
        \end{itemize}
    \end{enumerate}
\end{frame}

\begin{frame}{Hypothetical Incentives \& Beliefs}
OK but what do the data say? This is a science...\\
Studies that show incentives improve beliefs:
\begin{itemize}
    \item Posteriors closer to Bayes \citep{PhillipsEdwards1966,Grether1980,WrightAnderson1989}
    \item \cite{BurfurdWilkening2022}: 
    \begin{itemize}
        \item People w/ basic grasp of Bayes's Rule: $\downarrow$ errors
        \item People w/out grasp of Bayes: Update required: no difference\\
        No update required (uninformative signal): incentives are worse!
    \end{itemize}
    \item \cite{WrightAboul-Ezz1988}: beliefs closer to truth (eg, average GMAT scores) 
    \item More accurate beliefs in games \citep{GachterRenner2010,Wang2011} 
    \item \cite{Harrison2014} complex patterns of hypothetical bias
    \begin{itemize}
        \item Paying a flat fee largely fixes it!!
    \end{itemize}
\end{itemize}
\end{frame}

\begin{frame}{Hypothetical Incentives \& Beliefs}
Studies that show incentives improve beliefs:
\begin{itemize}
    \item Incentives improve belief formation \nocite{RutstromWilcox2009}
    \begin{itemize}
        \item No incentives $\Rightarrow$ default/focal values (50\% or 100\% \citep{MassoniEtAl2014,BurfurdWilkening2022}
        \item And $E\succ E^C$ yet $p(E)<1/2$ \citep{Grether1992}
    \end{itemize}
    \item Incentives reduce noise
    \begin{itemize}
        \item \citet{CamererHogarth1999}, \citet{GachterRenner2010}, and \citet{TrautmannVanDeKuilen2015}. Paying for power!
    \end{itemize}
    \item Higher incentives reduce overconfidence
    \begin{itemize}
        \item \cite{BloomEtAl2025}: firms guess future revenue
        \item Paid $\$x$ if guess is within $\pm 10\%$ (what does that elicit?)
    \end{itemize}
\end{itemize}
\begin{center}
    \includegraphics[height=1.1in,width=2in]{graphics/history/OverconfidenceByIncentivesBloomEtAl.png}
\end{center}
\end{frame}


\begin{frame}{Hypothetical Incentives \& Beliefs}
Studies that show no or even negative effect of incentives:
\begin{itemize}
    \item \citet{SonnemansOfferman2001} and \citet{TrautmannVanDeKuilen2015}
    \item BDM vs. Unincentivized
    \begin{itemize}
        \item \cite{MassoniEtAl2014}: tie
        \item \cite{HollardEtAl2016}: BDM $\succ$ no pay
    \end{itemize}
    \item \citet{ArmantierTreich2013} incentives are worse, but could be due to risk aversion
    \item \citet{TrautmannVanDeKuilen2015}: look at $p(E)+p(E^C)=1$\\
    More often true \textit{without} incentives.
\end{itemize}
\end{frame}



\begin{frame}[allowframebreaks]
    \frametitle{References:}
    \small
    \bibliographystyle{plainnat}
    %\bibliographystyle{elsarticle-harv}
    \bibliography{pjzotero}
\end{frame}
\end{document}