\documentclass[10pt,t,english]{beamer}
\usepackage{fontawesome}
\usepackage{graphicx}
\usepackage{array}
\usepackage[normalem]{ulem}
\usepackage{amsfonts,amsmath,amssymb,bm,bbm}
\usepackage{mathrsfs}
\usepackage{sgame}
\usepackage{graphicx,pstricks}
\usepackage{xcolor}
\usepackage{colortbl}
\usepackage{makecell}
\usepackage{tikz,tikzsymbols,gnuplottex}
\usetikzlibrary{decorations.pathreplacing,shapes}
\usepackage[english]{babel}
\usepackage[utf8]{inputenc}
\usepackage{appendixnumberbeamer}
\usepackage{datetime2}
\usepackage{booktabs}
\usepackage{setspace}
\usepackage{rotating}
\usepackage{natbib}
\usepackage{listings}

%\usepackage{matlab-prettifier} % For enhanced MATLAB highlighting
\lstset{
    language=matlab,
    % Other options for styling (optional)
    basicstyle=\ttfamily\footnotesize, % Font style
    keywordstyle=\color{blue}, % Keyword color
    commentstyle=\color{green!50!black}, % Comment color
    stringstyle=\color{red!70!black}, % String color
    numbers=left, % Line numbers on the left
    numberstyle=\tiny\color{gray}, % Style for line numbers
    frame=single, % Frame around the listing
    breaklines=true, % Allow line breaking
    captionpos=b, % Caption at the bottom
    tabsize=4 % Tab size
}

% ------------------------------------------------------------------------------
% Use the beautiful metropolis beamer template
% ------------------------------------------------------------------------------
\usepackage[T1]{fontenc}
\usepackage[utf8]{inputenc}
\usepackage{fontawesome}
\usepackage{FiraSans} 
\mode<presentation>
{
  \usetheme[progressbar=foot,background=light]{metropolis} 
  \usecolortheme{default} % or try albatross, beaver, crane, ...
  \usefonttheme{default}  % or try default, serif, structurebold, ...
  \setbeamertemplate{navigation symbols}{}
  \setbeamertemplate{caption}[numbered]
  %\setbeamertemplate{frame footer}{My custom footer}
} 

\newenvironment{stepenumerate}{\begin{enumerate}[<+->]}{\end{enumerate}}
\newenvironment{stepitemize}{\begin{itemize}[<+->]}{\end{itemize} }
\newenvironment{stepenumeratewithalert}{\begin{enumerate}[<+-| alert@+>]}{\end{enumerate}}
\newenvironment{stepitemizewithalert}{\begin{itemize}[<+-| alert@+>]}{\end{itemize} }

\newtheorem{question}{Question}
\newtheorem{claim}{Claim}
\newtheorem{proposition}{Proposition}
\newtheorem{remark}{Remark}
\newtheorem{conjecture}{Conjecture}

\definecolor{metrop}{RGB}{29, 44, 44}
\colorlet{GrayLight}{black!15}
\colorlet{GrayMedium}{black!30}
\colorlet{ForestGreen}{green!60!black}

\newenvironment{transitionframe}{
  \setbeamercolor{background canvas}{bg=black!80}
  \begin{frame}}{
    \end{frame}
}

\newcommand{\br}{

\bigskip

}

\newcommand{\pd}{\partial}
\newcommand{\RR}{\mathbb{R}}

\newcommand*\hugme[1]{\tikz[baseline=(char.base)]{\node[shape=ellipse,draw,inner sep=0pt] (char) {#1};}}

\newcounter{saveenumi}
\newcommand{\seti}{\setcounter{saveenumi}{\value{enumi}}}
\newcommand{\conti}{\setcounter{enumi}{\value{saveenumi}}}
\resetcounteronoverlays{saveenumi}



\newcommand\dotprod[2]{\langle #1 , #2 \rangle}
\newcommand{\ft}[1]{\widehat #1}
\newcommand{\qabove}[1]{\overset{\text{\large \textbf ?}}{#1}}
\newcommand{\eqae}{\overset{\text{a.e.}}{=}}
\newcommand{\calp}{\mathcal{P}}
\newcommand{\calg}{\mathcal{G}}
\newcommand{\calb}{\mathcal{B}}
\newcommand{\textd}{\text{d}}
\newcommand{\bbr}{\mathbb{R}}
\newcommand{\binm}{\mathbin{M}}
\newcommand{\binc}{\mathbin{C}}
\newcommand{\binb}{\mathbin{B}}
\newcommand{\calc}{\mathcal{C}}
\newcommand{\calh}{\mathcal{H}}
\newcommand{\bfone}{\mathbf{1}}
\newcommand{\bbe}{\mathbb{E}}
\newcommand{\bfle}{\mathbf{e}}
\newcommand{\calf}{\mathcal{F}}
\newcommand{\cala}{\mathcal{A}}
\newcommand{\cale}{\mathcal{E}}
\newcommand{\bbn}{\mathbb{N}}
\newcommand{\cantor}{\calc}
\newcommand{\calY}{\mathcal{Y}}
\newcommand{\textb}{\text{B}}
\newcommand{\calm}{\mathcal{M}}
\newcommand{\bint}{\mathbin{T}}
\newcommand{\ep}{\epsilon}
\newcommand{\bbq}{\mathbb{Q}}
\newcommand{\bbp}{\mathbb{P}}
\newcommand{\cals}{\mathcal{S}}
\newcommand{\emptysequence}{e}
\newcommand{\bbz}{\mathbb{Z}}
\newcommand{\fraka}{\frak{A}}
\newcommand{\frakb}{\frak{B}}
\newcommand{\length}{\text{length}}
\newcommand{\bfn}{\mathbf{N}}
\newcommand{\support}{\text{support}}
\DeclareMathOperator*{\argmax}{arg\,max}
\newcommand{\dom}{\mbox{dom}}
\def\ut{\underline t}
\def\um{\underline m}
\def\PP{\mathbb{P}}
\def\EE{\mathbb{E}}
\def\RR{\mathbb{R}}
\newcommand{\eps}{\varepsilon}

\begin{document}

% Title page info
\title[Regressions 2]{ExpEcon Methods:\\Measurement Error \& Attenuation Bias in OLS}
\author[ECON 8877]{ECON 8877\\P.J. Healy\\First version thanks to Changkuk Im} \color{metrop}
\institute[OSU]{}
\date[]{\vfill {\tiny Updated \today}}

\frame{\maketitle}


\begin{frame}
    \huge{\color{blue}Instrumental Variable (IV)}\\
    \vspace{8mm}
    \center\large{\color{blue}to deal with}\\
    \vspace{5mm}
    \hfill\huge{\color{blue}Measurement Error (ME)}
    \br
    \br
    But aren't IVs for applied folks???
\end{frame}

\begin{frame}{Gillen, Snowberg \& Yariv (2019)}
\begin{itemize}
    \item \textbf{Measurement errors (ME)} in the lab
    \begin{itemize}
        \item Participant's attention and focus
        \item Rounding due to finite choice menus
        \item Measures of risk/ambiguity aversion
    \end{itemize}
    \vspace{3mm}
    \item This paper illustrates the issue and proposes a mix of statistical tools (duplicate elicitations and IV approach) and design recommendations
    \begin{itemize}
        \item Other ways to solve: improve elicitation techniques, multiple rounds, ...
    \end{itemize}
\end{itemize}
\end{frame}

\begin{frame}{Outline}
    \begin{itemize}
        \item Gender gap in competition \citep{niederle2007women}
        \item[]<2-> {\color{blue}$\implies$ risk attitudes and overconfidence account for the gap}
        \begin{itemize}
            \item \color{blue}Linearly include controls
            \item \color{blue}Principal component analysis
            \item \color{blue}Instrumental variables
        \end{itemize}
        \item Low correlation between different methods of measuring risk \citep{friedman2014risky}
        \item[]<3-> {\color{blue}$\implies$ measures of risk attitudes are highly correlated}
        \begin{itemize}
            \item \color{blue}Obviously related instrumental variables (ORIV)
        \end{itemize}
        \item Compound risk and ambiguity are separate phenomena \citep{halevy2007ellsberg, epstein2010paradox, ahn2014estimating}
        \item[]<4-> {\color{blue}$\implies$ very little difference btw the two attitudes}
        \begin{itemize}
            \item \color{blue}Obviously related instrumental variables (ORIV)
        \end{itemize}
    \end{itemize}
\end{frame}

\begin{frame}{Measurement Error}
    \begin{definition}
        The model $X = X^{*} + \nu_{X}$ with $X^{*}$ and $\nu_{X}$ independent and $\mathrm{E}[\nu_{X}]=0$ is known as \textbf{classical measurement error}.
    \end{definition}
    \vspace{3mm}
    \begin{definition}
        We say that there is \textbf{endogeneity} in the linear model $Y=\beta X + \varepsilon$
        \\if $\beta$ is the parameter of interest and $\mathrm{E}[X \eps] \neq 0$.
    \end{definition}
\end{frame}

\begin{frame}{Measurement Error}
    Suppose that we are interested in to estimate the \textit{relationship} between the two variables, $Y^{*}$ and $X^{*}$.\\
    \vspace{3mm}
    But we can only observe variables measured with independent and identically distributed error,
    \[
    Y=Y^{*} + \nu_{Y} \quad \text{and} \quad X = X^{*} + \nu_{X}
    \]
     with $\text{E}[\nu_{k}]=0$, $\text{Var}[\nu_{k}] = \sigma^{2}_{\nu_{k}}$, and $\text{E}[\nu_{Y}\nu_{X}] = 0$.
\end{frame}

\begin{frame}{Measurement Error}
    The ideal regression model would be
    \[
    Y^{*} = \alpha^{*} + \beta^{*} X^{*} + \eps^{*}.
    \]
    Instead, we can only estimate
    \[
    Y = \alpha + \beta X + \eps
    \]
    where $\alpha$ is a constant and $\eps$ is a mean-zero random noise.\\
    \vspace{3mm}
    In this case, we have an \textbf{endogeneity problem}.
\end{frame}

\begin{frame}{Measurement Error}
Why?
    \begin{equation*}\begin{split}
    Y^{*} = \alpha^{*} + \beta^{*} X^{*} + \eps^{*}
    &\implies Y-\nu_{Y} = \alpha^{*} + \beta^{*} (X-\nu_{X}) + \eps^{*} \\
    &\implies Y = \alpha^{*} + \beta^{*} X + \underbrace{(\eps^{*} - \beta^{*}\nu_{X} + \nu_{Y})}_{=\eps}
\end{split}\end{equation*}
Hence
\begin{equation*}\begin{split}
    \text{E}[X\eps]
    = \text{E}[(X^{*}+\nu_{X}) (\eps^{*} - \beta^{*}\nu_{X} + \nu_{Y})]
    = -\beta^{*} \sigma^{2}_{\nu_{X}} \neq 0
\end{split}
\end{equation*}
if $\beta^{*}\neq 0$ and $\sigma^{2}_{\nu_{X}} \neq 0$.
\end{frame}

\begin{frame}{Measurement Error}
    Annotating finite-sample estimates with hats and population moments without hats, we have
\begin{equation*}
    \hat{\beta}
    = \frac{\widehat{\text{Cov}}[Y,X]}{\widehat{\text{Var}}[X]}
    = \frac{\widehat{\text{Cov}}[\alpha^{*} + \beta^{*} X^{*} + \eps^{*} + \nu_{Y}, X^{*}+\nu_{X}]}{\widehat{\text{Var}}[X^{*}+\nu_{X}]}
\end{equation*}
and
\begin{equation*}
    \text{E}[\hat{\beta}] = \text{plim}_{n\rightarrow \infty} \hat{\beta}
    = \beta^{*} \underbrace{\left( \frac{\sigma^{2}_{X^{*}}}{\sigma^{2}_{X^{*}} + \sigma^{2}_{\nu_{X}}} \right)}_{<1} < \beta^{*}.
\end{equation*}
This is called \textbf{measurement error bias} or \textbf{attenuation bias}.\\
\vspace{3mm}
{\color{blue}\phantom{Q. Against false positive... Is this a big problem?}}
\end{frame}

\begin{frame}[noframenumbering]
\frametitle{Measurement Error}
    Annotating finite-sample estimates with hats and population moments without hats, we have
\begin{equation*}
    \hat{\beta}
    = \frac{\widehat{\text{Cov}}[Y,X]}{\widehat{\text{Var}}[X]}
    = \frac{\widehat{\text{Cov}}[\alpha^{*} + \beta^{*} X^{*} + \eps^{*} + \nu_{Y}, X^{*}+\nu_{X}]}{\widehat{\text{Var}}[X^{*}+\nu_{X}]}
\end{equation*}
and
\begin{equation*}
    \text{E}[\hat{\beta}] = \text{plim}_{n\rightarrow \infty} \hat{\beta}
    = \beta^{*} \underbrace{\left( \frac{\sigma^{2}_{X^{*}}}{\sigma^{2}_{X^{*}} + \sigma^{2}_{\nu_{X}}} \right)}_{<1} < \beta^{*}.
\end{equation*}
This is called \textbf{measurement error bias} or \textbf{attenuation bias}.\\
\vspace{3mm}
{\color{blue}Q. Against false positive... Is this a big problem?}
\end{frame}

\begin{frame}{Simulated Example}
    \begin{table}[]
\centering
\label{tab:risk}
\scalebox{0.84}{
\begin{tabular}{lcccccc}
\hline
                                                                     & \multicolumn{6}{c}{Error as a percentage of $\text{Var}[X]$ and $\text{Var}[Y]$}    \\ \cline{2-7}
\multicolumn{1}{c}{}                                                 & 0\%      & 10\%         & 20\%         & 30\%         & 40\%         & 50\%         \\ \hline
$\widehat{\text{Corr}}[X,Y]$                                         & $1.00$   & $0.90^{***}$ & $0.80^{***}$ & $0.70^{***}$ & $0.60^{***}$ & $0.50^{***}$ \\
                                                                     & $(0.00)$ & $(0.02)$     & $(0.04)$     & $(0.05)$     & $(0.06)$     & $(0.08)$     \\
\multicolumn{1}{c}{$\widehat{\text{Corr}}[\text{E}[X],\text{E}[Y]]$} & $1.00$   & $0.95^{***}$ & $0.89^{***}$ & $0.82^{***}$ & $0.75^{***}$ & $0.66^{***}$ \\
                                                                     & $(0.00)$ & $(0.01)$     & $(0.02)$     & $(0.03)$     & $(0.04)$     & $(0.06)$     \\
ORIV $\widehat{\text{Corr}}[X,Y]$                                    & $1.00$   & $1.00$       & $1.00$       & $1.00$       & $1.00$       & $1.00$       \\
                                                                     & $(0.00)$ & $(0.01)$     & $(0.02)$     & $(0.04)$     & $(0.06)$     & $(0.10)$     \\ \hline
\end{tabular}
}
\end{table}
\small{\color{blue}* Coefficients and standard errors are averages from 10,000 simulated regressions ($N=100$).}
\begin{itemize}
    \item[$\implies$] Even a bit of ME causes significant deviations from the true correlation of 1, i.e., $\text{Corr}[X^{*},Y^{*}]=1$.
\end{itemize}
\end{frame}

\begin{frame}{Two Replicated Measures}
    Suppose that we elicit two replicated measures of $X^{*}$, i.e.,
    \[
    X^{{\color{red}a}}=X^{*}+\nu_{X}^{{\color{red}a}} \quad \text{and} \quad X^{{\color{blue}b}}=X^{*}+\nu_{X}^{{\color{blue}b}}
    \]
    with $\nu_{X}^{a}$, $\nu_{X}^{b}$ i.i.d. random variables, and ${\color{blue}\text{E}[\nu_{X}^{a}\nu_{X}^{b}]=0}$.
\end{frame}

\begin{frame}{Two-Stage Least Squares}
    Apply two-stage least squares (2SLS) to instrument $X^{\color{red}a}$ with $X^{\color{blue}b}$,
    \[
    X^{\color{red}a} = \pi_{0} + \pi_{1}X^{\color{blue}b} + \eps_{X}
    \implies \hat{\pi}_{1} = \frac{\widehat{\text{Cov}}[X^{a},X^{b}]}{\widehat{\text{Var}}[X^{b}]}
    \approx \frac{\widehat{\text{Var}}[X^{*}]}{\widehat{\text{Var}}[X^{b}]}.
    \]
    Then estimate $Y=\alpha + \beta (\hat{\pi}_{0} + \hat{\pi}_{1}X^{b}) + \eps_{Y}$.
    \begin{equation*}\begin{split}
    \hat{\beta}
    %&= \frac{\widehat{\text{Cov}}[Y,\hat{\pi}_{0} + \hat{\pi}_{1}X^{b}]}{\widehat{\text{Var}}[\hat{\pi}_{0} + \hat{\pi}_{1}X^{b}]} \\
    &= \frac{\widehat{\text{Cov}}[\alpha^{*}+\beta^{*}X^{*} + \eps^{*} + \nu_{Y},\hat{\pi}_{0} + \hat{\pi}_{1}X^{b}]}{\widehat{\text{Var}}[\hat{\pi}_{0} + \hat{\pi}_{1}X^{b}]}
    \approx
    \frac{\beta^{*} \hat{\pi}_{1} \widehat{\text{Var}}[X^{*}]}{(\hat{\pi}_{1})^{2} \widehat{\text{Var}}[X^{b}]}
    \rightarrow_{p} \beta^{*}.
    \end{split}
    \end{equation*}
    Thus, $\hat{\beta}$ is a consistent estimate of $\beta^{*}$.
\end{frame}

\begin{frame}{Instrumentation strategies}
    \begin{itemize}
        \item[Q.] Do we instrument $X^{\color{red}a}$ with $X^{\color{blue}b}$, or $X^{\color{blue}b}$ with $X^{\color{red}a}$?
        \begin{itemize}
            \item They may produce different results (see Table 5 in the paper).
        \end{itemize}
        \item[]
        \item[A.] The \textbf{obviously related IV (ORIV)} estimator consolidates the information from these different formulations.
        %\begin{itemize}
            %\item More efficient estimator.
            %\item GMM
        %\end{itemize}
    \end{itemize}
\end{frame}

\begin{frame}{ORIV}{Errors in Explanatory Variables}
The ORIV regressions estimates a \textit{stacked model} to consolidate the information from the two available instrumentation strategies:
    \begin{equation*}
        \begin{bmatrix}
        Y \\ Y \end{bmatrix}
        =
        \begin{bmatrix}
            \alpha_{1} \\ \alpha_{2}
        \end{bmatrix}
        + \beta
        \begin{bmatrix}
            X^{\color{red}a} \\ X^{\color{blue}b}
        \end{bmatrix}
        + \eps,
    \end{equation*}
    instrumenting
    \begin{equation*}
    \begin{bmatrix}
        X^{\color{red}a} \\ X^{\color{blue}b}
    \end{bmatrix}
    \text{with } W=
    \begin{bmatrix}
        X^{\color{blue}b} & 0_{N} \\ 0_{N} & X^{\color{red}a}
    \end{bmatrix}
    \end{equation*}
    \\
    \vspace{3mm}
    where $N$ is the number of participants and $0_{N}$ is an $N\times 1$ zero matrix.
\end{frame}

\begin{frame}{ORIV}
    This is equivalent to estimating a first stage for both instrumentation strategies, then estimating
    \begin{equation*}
        \begin{bmatrix}
        Y \\ Y \end{bmatrix}
        =
        \begin{bmatrix}
            \alpha_{1} \\ \alpha_{2}
        \end{bmatrix}
        + \beta
        \begin{bmatrix}
            \hat{X}^{\color{red}a} \\ \hat{X}^{\color{blue}b}
        \end{bmatrix}
        + \eps.
    \end{equation*}
The stacked regression will produce an estimated of $\beta^{*}$ that is the \textit{average} of the estimates from the two instrumentation approaches.
\end{frame}

\begin{frame}{ORIV}
    \begin{block}{Proposition 1}
        ORIV produces consistent estimates of $\beta^{*}$.
    \end{block}
    \vspace{3mm}
    \begin{block}{Proposition 2}
        The ORIV estimator satisfies asymptotic normality under standard conditions. The estimated standard errors, when clustered by participant, are consistent estimates of the asymptotic standard errors.
    \end{block}
\end{frame}

\begin{frame}{Errors in Outcome and Explanatory Variables}
Suppose that $Y^{a} = Y^{*} + \nu_{Y}^{a}$, $Y^{b} = Y^{*} + \nu_{Y}^{b}$, with $\text{E}[\nu_{Y}^{a}] = \text{E}[\nu_{Y}^{b}] = 0$.
    \begin{equation*}\begin{split}
        \begin{bmatrix}
        Y^{\color{red}a} \\ Y^{\color{red}a} \\ Y^{\color{blue}b} \\ Y^{\color{blue}b} \end{bmatrix}
        =
        \begin{bmatrix}
            \alpha_{1} \\ \alpha_{2} \\ \alpha_{3} \\ \alpha_{4}
        \end{bmatrix}
        + \beta
        \begin{bmatrix}
            X^{\color{red}a} \\ X^{\color{blue}b} \\ X^{\color{red}a} \\ X^{\color{blue}b}
        \end{bmatrix}
        + \eps, \text{ with } W=
    \begin{bmatrix}
        X^{\color{blue}b} & 0_{N} & 0_{N} & 0_{N} \\
        0_{N} & X^{\color{red}a} & 0_{N} & 0_{N} \\
        0_{N} & 0_{N} & X^{\color{blue}b} & 0_{N} \\
        0_{N} & 0_{N} & 0_{N} & X^{\color{red}a} \\
    \end{bmatrix}.
    \end{split}
    \end{equation*}
\small{\color{blue}* The existence of ME in $Y$ does not change propositions 1 and 2, although estimated standard errors will increase.}
\end{frame}

\begin{frame}{Estimating Correlations}
    Note that
    \begin{equation*}
        \hat{\beta} = \frac{\widehat{\text{Cov}}[X,Y]}{\widehat{\text{Var}}[X]}
        \implies
        \hat{\rho}_{XY} = \hat{\beta} \sqrt{\frac{\widehat{\text{Var}}[X]}{\widehat{\text{Var}}[Y]}}.
    \end{equation*}
    We cannot use $\text{Var}[X] = \text{Var}[X^{*}] + \text{Var}[\nu_{X}]$.\\
    \vspace{3mm}
    Instead, use $\text{Cov}[X^{\color{red}a},X^{\color{blue}b}] = \text{Cov}[X^{*}+\nu^{\color{red}a}_{X}, X^{*}+\nu^{\color{blue}b}_{X}] = \text{Var}[X^{*}]$.\\
    Thus,
    \begin{equation*}
        \hat{\rho}^{*}_{XY} = \hat{\beta}^{*} \sqrt{\frac{\widehat{\text{Cov}}[X^{\color{red}a},X^{\color{blue}b}]}{\widehat{\text{Cov}}[Y^{\color{red}a},Y^{\color{blue}b}]}}.
    \end{equation*}
\end{frame}

\begin{frame}{Estimating Correlations}
    \begin{block}{Proposition 3}
        $\hat{\rho}_{XY}^{*}$ is consistent with an asymptotically normal distribution, where standard errors can be derived using the delta method. \\
        \vspace{3mm}
        These standard errors can be consistently estimated using a bootstrap to construct confidence intervals.
    \end{block}
\end{frame}

\begin{frame}{Designing Experiments for ORIV}
    \textbf{Assumption.} MEs are {\color{red}\textit{independent}} across elicitations.\\
    \vspace{3mm}
    {\color{blue}Q. How to design an experiment to achieve this?}\\
    \vspace{3mm}
    A. The paper suggests:
    \begin{enumerate}
        \item Duplicated elicitations should use different numerical values.
        \item When using an MPL, the response grid should be constructed so that implied values are not the same.
        \item Duplicated items should be placed in different parts of the study.
    \end{enumerate}
\end{frame}

\begin{frame}{Measures of Risk}{Caltech Cohort Study}\label{slide:cohort_risk}
    \begin{itemize}
        \item \textbf{Project}
        \begin{itemize}
            \item Allocate 100 or 200 tokens btw a safe option and a project (e.g., returning 3 tokens w.p. 0.4 or nothing otherwise).
            \item \cite{gneezy1997experiment}
        \end{itemize}
        \item \textbf{Qualitative}
        \begin{itemize}
            \item Self-rate, on a scale of $0-10$, in terms of willingness to take risk.
            \item \cite{dohmen2011individual}
        \end{itemize}
        \item \textbf{Lottery menu}
        \begin{itemize}
            \item Choose btw six $50/50$ lotteries with different stakes.
            \item \cite{eckel2002sex}
        \end{itemize}
        \item \textbf{MPLs}
        \begin{itemize}
            \item E.g., 100 tokens w.p. $\frac{10}{20}$; 150 tokens w.p. $\frac{15}{30}$.
        \end{itemize}
    \end{itemize}
    \hyperlink{app:cohort_over}{\beamerskipbutton{Overconfidence}}\\
    \hyperlink{app:cohort_mpl}{\beamerskipbutton{Compound and Ambiguity}}
\end{frame}

\begin{frame}{Correlation Results}
\begin{table}[]
\centering
\label{tab:risk}
\scalebox{0.84}{
\begin{tabular}{lccclccc}
\hline
                     & \multicolumn{3}{c}{Raw Correlation}        &                      & \multicolumn{3}{c}{Corrected for ME}       \\ \cline{2-4} \cline{6-8}
                     & Project      & Qualitative  & Lottery      &                      & Project      & Qualitative  & Lottery      \\ \hline
Qualitative          & $0.26^{***}$ &              &              &                      & $0.40^{***}$ &              &              \\
\multicolumn{1}{c}{} & $(0.029)$    &              &              & \multicolumn{1}{c}{} & $(0.043)$    &              &              \\
Lottery              & $0.47^{***}$ & $0.25^{***}$ &              &                      & $0.71^{***}$ & $0.40^{***}$ &              \\
\multicolumn{1}{c}{} & $(0.029)$    & $(0.032)$    &              & \multicolumn{1}{c}{} & $(0.046)$    & $(0.052)$    &              \\
Risk MPL             & $0.19^{***}$ & $0.13^{***}$ & $0.22^{***}$ &                      & $0.30^{***}$ & $0.19^{***}$ & $0.38^{***}$ \\
\multicolumn{1}{c}{} & $(0.032)$    & $(0.033)$    & $(0.030)$    & \multicolumn{1}{c}{} & $(0.048)$    & $(0.047)$    & $(0.053)$    \\ \hline
\end{tabular}
}
\end{table}
\begin{enumerate}
    \item The corrected correlations are substantially higher.
    \item Some measures are noticeably more correlated.
    \begin{itemize}
        \item Project is most correlated with others (e.g., Project \& Lottery).
        \item MPL \& Qualitative are least correlated.
    \end{itemize}
\end{enumerate}
\end{frame}

\begin{frame}\label{slide:NV}
\frametitle{Misspecified Controls and ME}
\framesubtitle{\cite{niederle2007women}}
    $Y$: Competition\\
    $D$: Gender\\
    $X$: Controls for risk aversion and overconfidence\\
    \vspace{3mm}
    Consider a regression model:
    \begin{equation*}
        Y = \alpha D + X'\beta + \eps.
    \end{equation*}
    Consider the model:
    \begin{itemize}
        \item $Y^{*}=X^{*}$;
        \item $D$ and $X^{*}$ are correlated;
        \begin{itemize}
            \item Overconfidence \& gender \citep{moore2008trouble}
            \item Risk aversion \& gender \citep{charness2013experimental, holt2014assessment}
        \end{itemize}
        \item $X=X^{*}+{\color{red}\nu}$.
    \end{itemize}
    \vspace{3mm}
    \color{blue}We may have an erroneous conclusion that $Y$ and $D$ are correlated, even when controlling for $X$. \hyperlink{app:sim_control1}{\beamerskipbutton{e.g., simulation}}
\end{frame}

\begin{frame}{Replication Results}
    \begin{table}[]
\centering
%\caption{Gender, Competition, and Controls}
\label{tab:gender1}
\scalebox{0.90}{
\begin{tabular}{lccc}
\hline
                           & \multicolumn{3}{c}{Chose to Compete $(N=783)$} \\ \cline{2-4}
                           & (1)           & (2)           & (3)            \\ \hline
Male                       & $0.19^{***}$   & {\color{red}$0.11^{***}$}   & {\color{red}$0.048$}         \\
%                           & $(.034)$      & $(.031)$      & $(.033)$       \\
Risk aversion: MPL \#1     &               & $0.042^{***}$  &                \\
%                           &               & $(.015)$      &                \\
Overplacement: CRT         &               & $0.026^{***}$  &                \\
%                           &               & $(.015)$      &                \\
Risk aversion: project \#2 &               &               & $0.067^{***}$   \\
%                           &               &               & $(.016)$       \\
Perceived performance: CRT &               &               & $-0.042^{***}$  \\
%                           &               &               & $(.016)$
\hline
\end{tabular}
}
\end{table}
\small{\color{blue}* Guessed tournament rank, Tournament performance, Performance difference are controlled in (2) and (3).}
\vspace{3mm}
\begin{itemize}
    \item (1) and (2) replicate \cite{niederle2007women}.
    \item A different set of controls in (3) provides different result.
    \begin{itemize}
        \item Statistical significance of controls is not a good indicator of whether a trait is fully controlled for.
    \end{itemize}
\end{itemize}
\end{frame}

\begin{frame}{Three Approaches to Solve}
    \begin{enumerate}
        \item Include \textbf{multiple measures} for each of the possible controls $X$.
        \begin{itemize}
            %\item Cost of too many degrees of freedom.
            \item Cannot eliminate the effects of ME without a large enough number of controls. But how many?
        \end{itemize}
        \item Include \textbf{principal components} of the multiple controls.
        \item \textbf{Instrument} each control with a duplicate.
        \begin{itemize}
            \item Two controls are enough.
        \end{itemize}
    \end{enumerate}
    \vspace{3mm}
    The paper suggests that the {\color{blue}IV approach} is preferable whenever feasible.
\end{frame}

\begin{frame}{Regression Results}
\begin{table}[]
\centering
%\caption{Gender, Competition, and Controls}
\label{tab:gender2}
\scalebox{0.90}{
\begin{tabular}{lcccc}
\hline
                                & \multicolumn{4}{c}{Chose to Compete $(N=783)$}      \\ \cline{2-5}
                                & (1)         & (Sol 1)   & (Sol 2)   & (Sol 3)             \\ \hline
Male                            & $0.19^{***}$ & \color{red}$0.050$ & \color{red}$0.041$ & \color{red}$0.0063$          \\
6 risk aversion controls      &             & $F=4.9$ &         &                   \\
12 overconfidence controls     &             & $F=1.8$ &         &                   \\
Five Principal components &             &         & $F=37$  &                   \\
Instrumental variables          &             &         &         & $\chi^{2}_{7}=24$ \\ \hline
\end{tabular}
}
\end{table}
\small{\color{blue}* (Sol 1) has 76 controls in total including Guessed tournament rank, Tournament performance, Performance difference.}
\begin{itemize}
    \item[$\implies$] Estimated coefficients of gender variable is no more statistically significant.
\end{itemize}
\end{frame}

\begin{frame}
    \huge{\color{blue}Analysis of Variance}\\
    \huge{\color{blue}(ANOVA)}
\end{frame}

\begin{frame}{Motivation}
\begin{figure}[ht]
        \centering
        \includegraphics[scale=0.37]{graphics/reg2/screenshots/example_anova.png}
        \label{fig:anova}
    \end{figure}
    \small{\color{blue}Figure 4 from \cite{liguori2018operating}}
\end{frame}

\begin{frame}{Assumptions}
Suppose that there are $k$ groups of interest.\\
\vspace{3mm}
\textbf{Assumptions} \citep{verma2013one}:
    \begin{enumerate}
        \item The data must be measured either on interval or ratio scale.
        \item The samples must be independent.
        \item The dependent variable must be normally distributed.
        \item The population from which the samples have been drawn must be normally distributed.
        \item The variances of the population must be equal ($\sigma_{1}^{2}=\dots=\sigma_{k}^{2}=\sigma^{2}$).
        \item The errors are independent and normally distributed.
    \end{enumerate}
    \vspace{3mm}
    \textbf{Hypothesis:}
    \begin{equation*}\begin{split}
        &H_{0}: \mu_{1}=\dots=\mu_{k}. \\
        &H_{a}: \text{At least one $\mu_{i}$ is different.}
    \end{split}
    \end{equation*}
\end{frame}

\begin{frame}[noframenumbering]
\frametitle{Assumptions}
Suppose that there are $k$ groups of interest.\\
\vspace{3mm}
\textbf{Assumptions} \citep{verma2013one}:
    \begin{enumerate}
        \item The data must be measured either on interval or ratio scale.
        \item The samples must be {\color{red}independent}.
        \item The dependent variable must be {\color{red}normally distributed}.
        \item The population from which the samples have been drawn must be {\color{red}normally distributed}.
        \item The variances of the population must be equal {\color{red}($\sigma_{1}^{2}=\dots=\sigma_{k}^{2}=\sigma^{2}$).}
        \item The errors are independent and normally distributed.
    \end{enumerate}
    \vspace{3mm}
    \textbf{Hypothesis:}
    \begin{equation*}\begin{split}
        &H_{0}: \mu_{1}=\dots=\mu_{k}. \\
        &H_{a}: \text{At least one $\mu_{i}$ is different.}
    \end{split}
    \end{equation*}
\end{frame}

\begin{frame}{One-Way ANOVA}{\cite{lee2011statistical, verma2013one}}
Consider a \textit{between-subject} design ($1\sim k$ treatments).\\
Each treatment $j$ has $n_{j}$ number of (i.i.d.) observations.\\
Let $n=\sum_{j=1}^{k} n_{j}$ be the total number of observations.\\
\vspace{3mm}
Define
    \begin{itemize}
        \item[]<2-> $X_{ij}$: $i$-th observation in treatment $j$
        \item[]<2-> $\bar{X}_{j} = \frac{\sum_{i=1}^{n_{j}}X_{ij}}{ n_{j}}$: sample mean in treatment $j \; (j=1,\dots,k)$.
        \item[]<2-> $\bar{X} = \frac{\sum_{j=1}^{k}\sum_{i=1}^{n_{j}}X_{ij}}{n}$: grand sample mean.
        \vspace{3mm}
        \item[]<3-> $TSS = \sum_{j=1}^{k} \sum_{i=1}^{n_{j}} (X_{ij}-\bar{X})^{2}$: Total sum of squares.
        \item[]<3-> $SS_{b} = \sum_{j=1}^{k} n_{j} ({\color{red}\bar{X}_{j}}-{\color{red}\bar{X}})^{2}$: Sum of squares between groups.
        \item[]<3-> $SS_{w} = \sum_{j=1}^{k}\sum_{i=1}^{n_{j}} ({\color{red}X_{ij}}-{\color{red}\bar{X}_{j}})^{2}$: Sum of squares within groups.
        \begin{itemize}
            \item<4-> $TSS = SS_{b} + SS_{w}$
        \end{itemize}
    \end{itemize}
\end{frame}

\begin{frame}{One-Way ANOVA}{\cite{lee2011statistical, verma2013one}}
    \begin{itemize}
        \item[] $TSS = \sum_{j=1}^{k} \sum_{i=1}^{n_{j}} (X_{ij}-\bar{X})^{2}$: Total sum of squares .
        \item[] $SS_{b} = \sum_{j=1}^{k} n_{j} (\bar{X}_{j}-\bar{X})^{2}$: Sum of squares between groups.
        \item[] $SS_{w} = \sum_{j=1}^{k}\sum_{i=1}^{n_{j}} (X_{ij}-\bar{X}_{j})^{2}$: Sum of squares within groups.
    \end{itemize}
    \vspace{3mm}
    \begin{itemize}
        \item $MSS_{b} = \frac{SS_{b}}{\color{red}k-1}$: Mean sum of squares for between groups.
        \item $MSS_{w} = \frac{SS_{w}}{\color{red}n-k}$: Mean sum of squares for within groups.
    \end{itemize}
    \vspace{3mm}
    \textbf{Test statistic:}
    \[
    F = \frac{MSS_{b}}{MSS_{w}} \sim F_{k-1,n-k}.
    \]
\end{frame}

\begin{frame}
\frametitle{One-Way ANOVA}\label{slide:anova}
    Under the null, both $MSS_{b}$ and $MSS_{w}$ are unbiased estimator of $\sigma^{2}$.\\
    \vspace{3mm}
    If $H_{0}$ does not hold, then
    \begin{itemize}
        \item $MSS_{b}$ is NOT an unbiased estimator of $\sigma^{2}$ \textit{(biased upward)}.
        \item $MSS_{w}$ is an unbiased estimator of $\sigma^{2}$.
    \end{itemize}
    \hyperlink{app:Fstatistic}{\beamerskipbutton{Details}}
\end{frame}

\begin{frame}{ANOVA Table}
    \begin{table}[]
\centering
%\caption{}
\label{tab:anova}
\begin{tabular}{lllll}
\hline
Sources of variation & SS       & df    & MSS                          & $F$-value                   \\ \hline
Between groups       & $SS_{b}$ & $k-1$ & $MSS_{b}=\frac{SS_{b}}{k-1}$ & $F=\frac{MSS_{b}}{MSS_{w}}$ \\
Within groups        & $SS_{w}$ & $n-k$ & $MSS_{w}=\frac{SS_{w}}{n-k}$  &                             \\
Total                & $TSS$    & $n-1$ &                              &                             \\ \hline
\end{tabular}
\end{table}
\end{frame}

\begin{frame}
\frametitle{Example}
\framesubtitle{Stata screenshot}
Switch points from three different MPLs (MPL 0, MPL 1, MPL 2):
\begin{figure}[ht]
        \centering
        \includegraphics[scale=0.68]{graphics/reg2/screenshots/anova.png}
        \label{fig:anova}
    \end{figure}
$\implies$ Switch points from at least one MPL are different.
\end{frame}

\begin{frame}\label{anova_beyond}
\frametitle{Beyond One-Way ANOVA}
    \begin{itemize}
        \item \textbf{Two-Way ANOVA}: Compare more than one group
        \begin{itemize}
            \item E.g., (stata) \texttt{anova switch\char`_average treatment gender}
            \\
            \hyperlink{app:anova_twoway}{\beamerskipbutton{Screenshot}}
        \end{itemize}
        \vspace{3mm}
        \item \textbf{Analysis of covariance (ANCOVA)}: Controls for covariates
    \end{itemize}
\end{frame}

\begin{frame}{Regression with Dummy Variables}
    Regress $Y$ on \textbf{dummy variables} can do the same thing!\\
    \vspace{3mm}
    For example,
    \begin{equation*}
        Switch_{i} = \beta_{0} + \delta_{1} {\color{red}D_{i,1}} + \delta_{2} {\color{red}D_{i,2}} + \eps_{i}
    \end{equation*}
    where \begin{equation*}D_{i,k}=
    \begin{cases}
        0 \text{ if MPL 0}\\
        1 \text{ if MPL $k$}
    \end{cases}.
    \end{equation*}
\end{frame}

\begin{frame}{Example}{Stata screenshot}
\begin{figure}[ht]
        \centering
        \includegraphics[scale=0.62]{graphics/reg2/screenshots/reg.png}
        \label{fig:anova}
    \end{figure}
    \begin{enumerate}
        \item Switch points btwn MPL0 and MPL1 are significantly different.
        \item Switch points btwn MPL0 and MPL2 are not sig. different.
    \end{enumerate}
\end{frame}

\begin{frame}{Discussion}
    \begin{itemize}
        \item[1.] We can put more structures when regressing w/ dummy variables.
        \begin{itemize}
            \item Control variables, cluster se, nonlinear regression, ...
        \end{itemize}
        \item[]
        \item[2.] When regressing w/ dummy variables, we need a \textit{control} treatment.
        \begin{itemize}
            \item E.g., MPL1 vs MPL2?
            \item If we want to compare two groups, maybe do pairwise comparison with corrections?
        \end{itemize}
    \end{itemize}
\end{frame}

\begin{frame}{ANOVA vs Regression w/ Dummy Variables}
My takeaways are...
    \begin{itemize}
        \item To show the \textit{overall} difference across multiple groups, use \textbf{ANOVA}.
        \vspace{3mm}
        %\item To investigate difference between \textit{two specific groups}, regress with \textbf{dummy variables}.
        \vspace{3mm}
        \item To put more \textit{structures} (control variables, cluster se, nonlinear regression), regress with \textbf{dummy variables} with a proper specification.
        \begin{itemize}
            \item E.g., we often do $t$-test to show the overall difference between two groups, and then run regressions to have further analysis.
        \end{itemize}
    \end{itemize}
\end{frame}

\begin{frame}[noframenumbering]
    \huge{\color{blue}The End}
\end{frame}

\begin{frame}[noframenumbering]
\frametitle{Appendix}\label{app:cohort_over}
\framesubtitle{Caltech Cohort Study: Overconfidence}
\begin{itemize}
    \item \textbf{Overestimation and overplacement}
    \begin{itemize}
        \item How many they think they answered correctly.
        \item Where they think they are in the performance distribution of all participants
    \end{itemize}
    \item \textbf{Overprecision}
    \begin{itemize}
        \item How confident they are of their guess (six-point qualitative scale).
    \end{itemize}
    \item \textbf{Perception of academic performance}
    \begin{itemize}
        \item Where in the grade distribution of their entering cohort they belieive they would fall over the next year.
    \end{itemize}
\end{itemize}
\hyperlink{slide:cohort_risk}{\beamerskipbutton{Caltech Cohort Study: Risk measures}}
\end{frame}

\begin{frame}[noframenumbering]
\frametitle{Appendix}\label{app:cohort_mpl}
\framesubtitle{Caltech Cohort Study: Ambiguous and Compound Lotteries}
\begin{itemize}
    \item \textbf{Compound MPL}
    \begin{itemize}
        \item Same as MPL except that the number of balls is uniformly drawn.
    \end{itemize}
    \item \textbf{Ambiguous MPL}
    \begin{itemize}
        \item Same as MPL except that the composition of the urn was chosen by the dean of undergraduate students of Caltech.
    \end{itemize}
\end{itemize}
\hyperlink{slide:cohort_risk}{\beamerskipbutton{Caltech Cohort Study: Risk measures}}
\end{frame}

\begin{frame}[noframenumbering]
\frametitle{Appendix: Simulated Example}\label{app:sim_control1}
$Y$: Participation in dangerous sports\\
$D$: Gambling\\
$X$: Risk attitude (experimentally measured)\\
\vspace{3mm}
Consider a model:
    \begin{itemize}
        \item $Y^{*}=X^{*}$;
        \begin{itemize}
            \item $X^{*}\sim N[0,1]$ and $Y=Y^{*} + \zeta$ where $\zeta \sim N[0,1]$.
        \end{itemize}
        \item $D = 0.5 \cdot X^{*} + \eta$\quad where $\eta \sim N [0,0.9]$;
        \item $X = X^{*} + \nu$\quad where $\nu \sim N[0,{\color{red}\sigma^{2}_{\nu}}]$.
    \end{itemize}
    \vspace{3mm}
    Consider a regression model:
    \[
    Y = \alpha D + \beta X + \eps
    \]
    \hyperlink{slide:NV}{\beamerskipbutton{Model}}
\end{frame}

\begin{frame}[noframenumbering]
\frametitle{Appendix: Simulated Example}\label{app:sim_control2}
    \begin{table}[]
\centering
\caption{Simulated Regressions}
\label{tab:sim}
\begin{tabular}{lcccccc}
\hline
               & \multicolumn{6}{c}{ME as a Percent of Var$[X]$, i.e., $\frac{\sigma^{2}_{\nu}}{\sigma^{2}_{\nu}+\sigma^{2}_{X^{*}}}$}                                                                                                                                                                                                                                                                                                                                          \\ \cline{2-7}
               & 0\%                                                            & 10\%                                                          & 20\%                                                          & 30\%                                                           & {\color{red}40\%}                                                          & {\color{red}50\%}                                                          \\ \hline
$\hat{\alpha}$ & \begin{tabular}[c]{@{}c@{}}$.00$\\ $(.11)$\end{tabular}        & \begin{tabular}[c]{@{}c@{}}$.06$\\ $(.11)$\end{tabular}       & \begin{tabular}[c]{@{}c@{}}$.11$\\ $(.12)$\end{tabular}       & \begin{tabular}[c]{@{}c@{}}$.16$\\ $(.12)$\end{tabular}        & \begin{tabular}[c]{@{}c@{}}$.21^{*}$\\ $(.12)$\end{tabular}   & \begin{tabular}[c]{@{}c@{}}$.26^{***}$\\ $(.12)$\end{tabular} \\
$\hat{\beta}$  & \begin{tabular}[c]{@{}c@{}}$1.00^{***}$\\ $(.12)$\end{tabular} & \begin{tabular}[c]{@{}c@{}}$.87^{***}$\\ $(.11)$\end{tabular} & \begin{tabular}[c]{@{}c@{}}$.75^{***}$\\ $(.11)$\end{tabular} & \begin{tabular}[c]{@{}c@{}}$.64^{***}$\\ $(0.10)$\end{tabular} & \begin{tabular}[c]{@{}c@{}}$.54^{***}$\\ $(.10)$\end{tabular} & \begin{tabular}[c]{@{}c@{}}$.44^{***}$\\ $(.09)$\end{tabular} \\ \hline
\end{tabular}
\end{table}
True model: $\alpha=0$ and $\beta=1$.\\
\hyperlink{slide:NV}{\beamerskipbutton{Model}}
\end{frame}

\begin{frame}[noframenumbering]
\frametitle{Appendix: F-statistic for One-Way ANOVA}\label{app:Fstatistic}
    Consider $X_{ij}=\mu_{j} + \eps_{ij}$, $\eps_{ij} \sim N(0,\sigma^{2})$ and independent.\\
    We can rewrite it as
    \[
    X_{ij}=\mu + \alpha_{j} + \eps_{ij}
    \]
  where $\mu=\frac{1}{k} \sum_{j=1}^{k} \mu_{j}$, and $\alpha_{j}$ is called sample effect.\\
  %It is easy to show that $\sum_{i=1}^{k} \alpha_{i} = 0$ and $\text{E}(X_{ij})=\mu+\alpha_{i}$.\\
  Under {\color{red}$n_{j}=m$}, we have
  \[
    \text{E}[SS_{b}] = (k-1)\sigma^{2} + m\sum_{j=1}^{k} {\color{red}\alpha_{j}^{2}}
    \quad \text{and} \quad
  \text{E}[SS_{w}] = k(m-1)\sigma^{2}.
  \]
  Hence, $MSS_{b}$ is an unbiased estimator of $\sigma^{2}$ only if the null ($\alpha_{1}=\dots=\alpha_{k}=0$) is true, while $MSS_{w}$ is an unbiased estimator regardless of the null.\\
  In particular, $MSS_{b}$ gets larger as $\alpha_{j}$ increases.\\
  \vspace{3mm}
  Since $MSS_{b}/\sigma^{2} \sim \chi^{2}_{k-1}$, $MSS_{w}/\sigma^{2} \sim \chi^{2}_{n-k}$, and they are independent (by Cochran theorem), $F=\frac{MSS_{b}}{MSS_{w}} \sim F_{k-1,n-k}$. \hyperlink{slide:anova}{\beamerskipbutton{Slide: One-Way ANOVA}}
\end{frame}

\begin{frame}[noframenumbering]
\label{app:anova_twoway}
\frametitle{Appendix: Two-Way ANOVA}
\framesubtitle{Screenshot}
\begin{figure}[ht]
        \centering
        \includegraphics[scale=0.68]{graphics/reg2/screenshots/anova_2.png}
        \label{fig:anova2}
    \end{figure}
    \hyperlink{anova_beyond}{\beamerskipbutton{Slide: Beyond One-Way ANOVA}}
\end{frame}

\begin{frame}[noframenumbering, allowframebreaks]
\frametitle{References}
\bibliographystyle{elsarticle-harv}
\bibliography{Lecture_Slides/Econometrics/ref_reg2}
\end{frame}

\end{document}